\documentclass[12pt]{article}
\usepackage{amsmath}
\usepackage{graphicx}
\usepackage{hyperref}
\usepackage[utf8]{inputenc}
\usepackage{mathrsfs}
\usepackage{amssymb}

\title{Introduction to Topological Groups}
\author{Dikran Dikranjan}

\begin{document}

\section{Pontryagin-van Kampen duality}

\subsection{The Dual Group}


In the sequel we shall write the circle additively as $(\mathbb{T}, +)$ and we denote by $q_0 : \mathbb{R} \to \mathbb{T} = \mathbb{R}/\mathbb{Z}$ the canonical
projection. For every $k \in \mathbb{N}_+$ let $\wedge_k = q_0((\frac{-1}{3k},\frac{1}{3k}))$. Then ${\wedge_k : k \in \mathbb{N}_+}$ is a base of the neighborhoods of 0
in $\mathbb{T}$, because ${(\frac{-1}{3k}, \frac{1}{3k}) : k \in \mathbb{N}_+}$ is a base of the neighborhoods of 0 in $\mathbb{R}$.


    For every abelian group $G^* = Hom (G,\mathbb{T})$. For a subset $K$ of $G$ and a subset $U$ of $\mathbb{T}$ let


        $W_{G^*} (K, U) = \{\chi \in G : \chi(K) \subseteq U\}$.


    For any subgroup $H$ of $G^*$ we abbreviate $H \cap W(K, U)$ to $W_H (K, U)$. When there is no danger of confusion
we shall write only $W(K, U)$ in place of $W_{G^*} (K, U)$. The group $G^*$ will be considered only with one topology,
namely the induced from ${T}^G$ compact topology (see Remark 4.1).


    If $G$ is a topological abelian group, $\hat{G}$ will denote the subgroup of $G^*$ consisting of continuous characters.


    The group $\hat{G}$ will carry the compact open topology that has as basic neighborhoods of 0 the sets $W_{\hat{G}}(K, U)$,
where $K$ is a compact subset of $G$ and $U$ is neighborhood of 0 in $\mathbb{T}$. We shall see below that when $U \subseteq \wedge_1$,
then $W_{\hat{G}}(K, U)$ coincides with $W_{G^*} (K, U)$ in case $K$ is a neighborhood of 0 in $G$. Therefore we shall use mainly
the notation $W(K, U)$ when the group $G$ is clear from the context.


    Let us start with an easy example.


\textbf{Example 7.1.} Let $G$  be an abelian topological group.


    \begin{enumerate}

        \item If $G$ is compact, then $\hat{G}$ is discrete.

        \item If $G$ is discrete, then $\hat{G}$ is compact.

    \end{enumerate}


    Indeed, to prove (1) it is sufficient to note that $W_{\hat{g}}(G,\wedge_1) = {0}$ as $\wedge_1$ contains no subgroup of $\mathbb{T}$ beyond 0


    (2) Suppose that $G$ is discrete. Then $\hat{G} = Hom(G,\mathbb{T})$ is a subgroup of the compact group $T^G$. The 
compactopen topology of $\hat{G}$ coincides with the topology inherited from $\mathbb{T}^G$: let $F$ be a finite subset of $G$ and $U$ an open
neighborhood of 0 in $\mathbb{T}$, then


    $\bigcap_{x \in F} π^{-1}_x (U) \cap Hom (G,T) = {\chi \in Hom(G,\mathbb{T}) : \pi_x \in U \text{ for every } x \in F}$


        $= {\chi \in Hom (G,\mathbb{T}) : \chi(x) \in U \text{ for every } x \in F} = W(F, U)$.


    Moreover $Hom (G,\mathbb{T})$ is closed in the compact product $\mathbb{T}^G$ by Remark 4.1 and we can conclude that $\hat{G}$ is
compact.


    Now we prove that the dual group is always a topological group. If the group $G$ is locally compact, then its
dual is locally compact too. This is the first step of the Pontryagin-van Kampen duality theorem.


\textbf{Theorem 7.2}. For an abelian topological group $G$ the following assertions hold true:


    \begin{itemize}

        \item (a) if $x \in \mathbb{T}$ and $k \in \mathbb{N}_+$, then $x \in \wedge_k$ if and only if $x, 2x, . . . , kx \in \wedge_1$;
        
        \item (b) $\chi \in Hom (G,T)$ is continuous if and only if $\chi^{-1} (\wedge_1)$ is a neighborhood of 0 in $G$;
        
        \item (c) $\{W_{\hat{G}}(K,\wedge_1) : K \text{ compact } \subseteq G\}$ is a base of the neighborhoods of 0 in $\hat{G}$, in particular $\hat{G}$ is a topological
        group.        
        
        \item (d) $W_{\hat{G}}(A,\wedge_s) + W_{\hat{G}}(A,\wedge_s) \subseteq W_{\hat{G}}(A,\wedge_{s-1})$ and $W_{\hat{G}}(\bar{A},\wedge_s) + W_{\hat{G}}(\bar{A},\wedge_s) ⊆ W_{\hat{G}}(\bar{A},\wedge_{s-1})$ for every $A \subseteq G$
        and $s > 1$.        
        
        \item (e) if $F$ is a closed subset of $\mathbb{T}$, then for every $K \subseteq G$ the subset $W_{G^*}(K, F)$ of $G^*$ is closed (hence, compact);        
        
        \item (f) if $U$ is neighborhood of 0 in $G$, then
            
            \item (f_1) $W_{G^*}(\bar{U}, V) = W_{G^*} (\bar{U}, V)$ for every neighborhood of 0 $V \subseteq \wedge_1$ in $T$;
            
            \item (f_2) $W(\bar{U},\wedge_4)$ has compact closure;
            
            \item (f_3)  if $U$ has compact closure, then $W(\bar{U},\wedge_4)$ is a neighborhood of 0 in $\hat{G}$ with compact closure, so $\hat{G}$ is
            locally compact.

    \end{itemize}


    \emph{Proof.} (a) Note that for $s \in \mathbb{N}$, $sx \in \wedge_1$ if and only if $x \in A_{s,t} = \wedge_s+\pi_{\mathbb{T}}(\frac{t}{s})$ for some integer $t$ with $0 \leq t \leq s$. On
the other hand, $A_s, 0 = \wedge_s$ and $\wedge_s \cap A_{s+1,t}$ is non-empty if and only if $t = 0$. Hence, if $x \in \wedge_s$ and $(s+ 1)x \in \wedge_1$,
then $x \in \wedge_{s+1}$ and this holds in particular for $1 \leq s < k$. This proves that $sx \in \wedge_1$ for $s = 1, . . . , k$ if and only
if $x \in \wedge_k$.


    (b) Suppose that $\chi^{-1}(\wedge_1)$ is a neighborhood of 0 in $G$. So there exists an open neighborhood $U$ of 0 in $G$
such that $U \subseteq \chi_{-1}(\wedge_1)$. Moreover, there exists an other neighborhood $V$ of 0 in $G$ with $\underbrace{V + · · · + V}_k \subseteq U$ where
$k \in \mathbb{N}_+$. Now $\chi(y) \in \wedge_1$ for every $y \in V$ and $s = 1, . . . , k$. By item (a) $\chi(y) \in \wedge_k$ and so $\chi(V) \subseteq \wedge_k$.


    (c) Let $k \in \mathbb{N}_+$ and $K$ be a compact subset of $G$. Define $L = \underbrace{K + · · · + K}_k$, which is a compact subset of
$G$ because it is a continuous image of the compact subset $K^k$ of $G^k$. Take $\chi \in W(L,\wedge_1)$. For every $x \in K$ we
have $s\chi(x) \in \wedge_1$ for $s = 1, . . . , k$ and so $\chi(x) \in \wedge_k$ by item (a). Hence $W(L,\wedge_1) \subseteq W(K,\wedge_k)$.


    (d) obvious.


    (e) If $\pi_x : \mathbb{T}^G \to \mathbb{T}$ is the projection defined by the evaluation at $x$, for $x \in G$, then obviously


    $W_{G^*} (K, F) = \bigcap_{x \in K} \{\chi \in G^*: \chi(x) \in F\} = \bigcap_{x \in K}\pi^{-1}_{x}(F)$


is cloased as each $\pi^{-1}_{x}(F)$ is closed in $G^*$.


    (f1) follows immediately from item (c).


    (f2) To prove that the closure of $W0 = W(U,Λ4)$ is compact it is sufficient to note that $W_0 \subseteq W_1 := W(\bar{U},\bar{\wedge_4})$
and prove that $W_1$ is compact. Let $\tau_s$ denote the subspace topology of W1 in Gb. We prove in the sequel that
(W1, τs) is compact.


    Consider on the set $W_1$ also the weaker topology $\tau$ induced from $G^*$ and consequently from $\mathbb{T}^G$. By (e)
$(W_1, \tau)$ is compact.


    It remains to show that both topologies $\tau_s$ and $\tau$ of $W_1$ coincide. Since $\tau_s$ is finer than $\tau$ , it suffices to show
that if $\alpha \in W_1$ and $K$ is a compact subset of $G$, then $(\alpha + W(K,\wedge_1)) \cap W_1$ is also a neighborhood of $\alpha$ in
$(W_1, \tau)$


    Since $\bigcup{a + U : a \in K} \supseteq K$ and $K$ is compact, $K \subseteq F + U$, where $F$ is a finite subset of $K$. We prove
now that


    $(\alpha + W(F,\wedge_2)) \cap W_1 \subseteq (\alpha + W(K,\wedge_1)) \cap W_1$. (*) 


    Let $\xi \in W(F,\wedge_2)$, so that $\alpha + \xi' \in W_1 = W(\bar{U},\bar{\wedge_4})$. As $\alpha \in W_1$ as well, we deduce from items (c) and (d) that
$\xi = (\alpha + \xi') - \alpha \in W_1 - W_1$. Hence $\xi(\bar{U}) \subseteq Λ_2$ and consequently
    

    $\xi(K) \subseteq \xi(F + U) \subseteq \wedge_2 + \bar{\wedge_2} \subseteq \wedge_1.$


This proves $\xi \in W(K,\wedge_1)$ and (*).


    (f3) Follows obviously from (f2) and the definition of the compact open topology


    The above proof shows another relevant fact. The neighborhood $W(\bar{U},\wedge_4)$ of 0 in the dual group $\hat{G}$ carries
the same topology in $\hat{G}$ and $G^*$, nevertheless the inclusion map $j : \hat{G} , \hookrightarrow G^*$ need not be an embedding:
    

\textbf{Corollary 7.3}. For a locally compact abelian group $G$ the following are equivalent:


    \begin{itemize}

        \item the inclusion map $j : \hat{G} , b \to G^*$ is an embedding;

        \item $G$ is discrete;

        \item $\hat{G} = G^*$ is compact

    \end{itemize}


Proof. Since $G^*$ is compact, $j$ can be an embedding iff $\hat{G}$ itself is compact. According to Example 7.1 this
occurs precisely when $G$ is discrete. In that case $\hat{G} = G^*$ is compact.


    Actually, it can be proved, once the duality theorem is available, that $j : \hat{G}, \hookrightarrow G^*$ need not be even a local
homeomorphism. (If $j$ is a local homeomorphism, then the topological subgroup $j(\hat{G})$ of $G^*$ will be locally
compact, hence closed in $G^*$. This would yield that $j(G^*)$ is compact. On the other hand, the topology of
$j(G^*)$ is precisely the initial topology of all projections $p_x$ restricted to $\hat{G}$. By the Pontryagin duality theorem,
these projections form the group of all continuous characters of $\hat{G}$. So this topology coincides with $T_{\hat{\hat{G}}}$. By a
general theorem of Glicksberg, a locally compact abelian groups $H$ and $(H, \mathcal{T}_{\hat{H}})$ have the same compact sets.
In particular, compactness of $(H, \mathcal{T}_{\hat{H}})$ yields compactness of $H$. This proves that if $j : \hat{G}, \hookrightarrow G^*$ is a local
homeomorphism, then $\hat{G}$ is compact and consequently $G$ is discrete.)


\subsection{Computation of some dual groups}


In the next proprosition we show, roughly speaking, that the projective order between continuous surjective
open homomorphisms with the same domain corresponds to the order by inclusion of their kernels.


\textbf{Proposition 7.4} Let $G$, $H_1$ and $H_2$ be topological abelian groups and let $\chi_i: G \to H_i, i = 1, 2$, be continuous
surjective open homomorphisms. Then there exists a continuous homomorphism $\iota : H_1 \to H_2$ such that $\chi_2 = \iota \circ \chi_1 $
iff ker $\chi_1 \leq ker \chi_2$. If ker $\chi_1 = ker \chi_2$ then $\iota$ will be a topological isomorphism.


    Proof. The necessity is obvious. So assume that ker $\chi_1 \leq ker \chi_2$ holds. By the homomorphism theorem applied
to $\chi_i$ there exists a topological isomorphisms $j_i: G/ ker \chi_i \to H_i$ such that $\chi_i = j_i \circ q_i$, where $q_i: G \to G/ ker \chi_i$
is the canonical homomorphism for $i = 1, 2$. As $ker \chi_1 \leq ker \chi_2$ we get a continuous homomorphism $t$ that
makes commutative the following diagram

%Note from Skarlet: This diagram is a little difficult to type out so I had to describe it a bit, feel free to contact me for clarification
At the top of the diagram, there is $G$, and $G$ points to all of the elements in the line below like so $G \to^{\chi_1} H_1$, $G \to^{q_1} G / ker \chi_1$, $G \to^{q_2} G / ker \chi_2$, $G \to^{\chi_2} H_2$
$H_1 \leftarrow^{j_1} G / ker \chi_1 \dashrightarrow^t G / ker \chi_2 \rightarrow^{j_2} H_2$
Along the bottom of the diagram, there is an arrow arching from one end to the other $H_1 \to_\iota H_2$


Obviously $\iota = j_2 \circ t \circ j^{-1}_1$ works. If $ker \chi_1 = ker \chi_2$, then t is a topological isomorphism, hence $\iota$ will be a
topological isomorphism as well.


    In the sequel we denote by $k · id_G$ the endomorphism of an abelian group $G$ obtained by the map $x \mapsto kx$,
for a fixed $k \in \mathbb{Z}$. The next lemma will be used for the computation of the dual groups in Example 7.7.


\prrof{Proof.} We prove first that the only topological isomorphisms $\chi : \mathbb{T} \to \mathbb{T}$ are $\pm id_{\mathbb{T}}$. The proof will exploit the
fact that the arcs are the only connected sets of $\mathbb{T}$. Hence $\chi$ sends any arc of $\mathbb{T}$ to an arc, sending end points to
end points. Denote by $\varphi$ the canonical homomorphism $\mathbb{R} \to \mapsto{T}$ and for $n \in \mathbb{N}$ let $c_n = \varphi(1/2^n)$ be the generators
of the Pr¨ufer subgroup $\mathbb{Z}(2^{\infty})$ of $\mathbb{T}$. Then, $c_1$ is the only element of $\mathbb{T}$ of order 2, hence $g(c_1) = c_1$. Therefore,
the arc $A_1 = \varphi([0, 1/2])$ either goes onto itself, or goes onto its symmetric image $-A_1$. Let us consider the first
case. Clearly, either $g(c_2) = c_2$ or $g(c_2) = -c_2$ as $o(g(c_2)) = 4$ and being $\pm c_2$ the only elements of order 4 of $\mathbb{T}$.
By our assumption $g(A_1) = A_1$ we have $g(c_2) = c_2$ since $c_2$ is the only element of order 4 on the arc $A_1$. Now
the arc $A_2 = [0, c_2]$ goes onto itself, hence for $c_3$ we must have $g(c_3) = c_3$ as the only element of order 8 on the
arc $A_2$, etc. We see in the same way that $g(c_n) = c_n$. Hence $g$ is identical on the whole subgroup $\mathbb{Z}(2^{\infty})$. As
this subgroup is dense in $\mathbb{T}$, we conclude that $g$ coincides with $id_{\mathbb{T}}$. In the case $g(A_1) = -A_1$ we replace $g$ by
$-g$ and the previous proof gives $-g = id_{\mathbb{T}}$, i.e., $g = -id_{\mathbb{T}}$.


    For $k \in \mathbb{N}_+$ let $\pi_k = k · id_{\mathbb{T}}$. Then $ker \pi_k = \mathbb{Z}_k$ and $\pi_k$ is surjective. Let now $χ : \mathbb{T} \to \mathbb{T}$ be a non-trivial
continuous homomorphism. Then $ker \chi$ is a closed proper subgroup of $\mathbb{T}$, hence $ker \chi = \mathbb{Z}_k$ for some $k \in N_+$.
Moreover, $\chi(\mathbb{T})$ is a connected non-trivial subgroup of $\mathbb{T}$, hence $\chi(\mathbb{T}) = \mathbb{T}$. By Proposition 7.4 $\chi = \pm \pi_k$.


    Obviously, $\chi = \pm \xi$ for characters $\chi, \xi : G \to \mathbb{T}$ implies $ker \chi = ker \xi$ and $\chi (G) = \xi (G)$. More generally,
if $\chi = k · \xi$ for some $k \in \mathbb{Z}$, then $ker \chi \geq ker \xi$ and $\chi (G) \leq \xi (G)$. Now we see that this implication can be
(partially) inverted under appropriate hypotheses.


\textbf{Corollary 7.6.} Let $G$ be a $\sigma$-compact locally compact abelian group and let $\chi, \xi : G \hookrightarrow T$ be continuous
characters such that $ker \chi \geq ker \xi$ and $\chi (G) \leq \xi (G)$.


\begin{itemize}

    \item (a) If $\chi(G) = \xi(G) \mathbb{T}$ then $\chi = k · \xi$ for some $k \in \mathbb{Z}$; moreover, $ker \chi = ker \xi$ iff $\chi = \pm \xi$

    \item (b) If $G$ is compact and $|\xi (G)| = m$ for some $m \in \mathbb{N}_+$, then $\chi = k \xi$ for some $k \in \mathbb{Z}$; moreover, $ker \chi = ker \xi$
    iff $\chi(G) = \xi(G)$, in such a case $k$ must be coprime to $m$.

    \item (c) If $ker \xi = ker \xi$ is open and H = χ(G) = ξ(G), then χ = ι ◦ ξ, where ι : H → H is an arbitrary
    automorphism of the subgroup H of T equipped with the discrete topology.    

\end{itemize}


\emph{Proof.} (a) As $\chi (G) = \xi (G) = T$ and $G$ is $\sigma$-compact, we can apply Lemma 7.4 and observe that the only $\iota$ given
by the lemma can be $k · id_{\mathbb{T}}$ for some $k \in \mathbb{Z}$ in view of the previous lemma. The same lemma yields $k = \pm 1$
when ker $\chi = ker \xi$.


    (b) If $G$ is compact and $|\xi (G)| = m$ for some $m \in \mathbb{N}_+$, $\xi (G)$ is a cyclic subgroup of $\mathbb{T}$ of order $m$. Note that
$\mathbb{T}$ has a unique such cyclic subgroup. By Proposition 7.4 there exists a homomorphism $\iota : \xi (G) \to \chi (G)$ such
that $\chi = \iota \circ \xi$. The hypothesis $\chi (G) \leq \xi (G)$ implies that there such a $\iota$ must by the multiplication by some
$k \in \mathbb{Z}$. In case $\chi (G) = \xi (G)$ this $k$ is coprime to $m$.


    (c) Obvious.


\textbf{Example 7.7} Let $p$ be a prime. Then $\hat{Z(p^{\infty})} \cong \mathbb{J}_p, \hat{\mathbb{J}_p} \cong \mathbb{Z}(p^{\infty}), \hat{\mathbb{T}} \cong \mathbb{Z}, \hat{\mathbb{Z}} \cong \hat{\mathbb{T}}$ and $\hat{\mathbb{R}} \cong \mathbb{R}$.


Proof. The first isomorphism $\hat{\mathbb{Z}(p^{\infty})} = \mathbb{J}_p$ follows from our definition $\mathbb{J}_p = End(\mathbb{Z}(p^{\infty})) = Hom(\mathbb{Z}(p^{\infty}),\mathbb{T}) = \hat{\mathbb{Z}(p^{\infty})}$.


    To verify the isomorphism $\hat{\mathbb{J}_p} \cong Z(p^{\infty})$ consider first the quotient homomorphism $\eta_n : \mathbb{J}_p \to \mathbb{J}_p / p^n \mathbb{J}_p\cong \mathbb{Z}_{p^n} \leq \mathbb{T}$.
With this identifications we consider $\eta_n \in \hat{\mathbb{J}_p}$. It is easy to see that under this identification $p\eta_n = \eta_{n-1}$.
Therefore, the subgroup $H$ of $\hat{\mathbb{J}_p}$ generated by the characters $\eta_n$ is isomorphic to $\mathbb{Z}(p^\infty)$. Let us see that $H = \hat{\mathbb{J}_p}$.
Indeed, take any non-trivial character $\chi : \mathbb{J}_p \to \mathbb{T}$. Then $\mathbb{N} = ker \chi$ is a closed proper subgroup of $\mathbb{J}_p$. Moreover,
$N \neq 0$ as $\mathbb{J}_p$ is not isomorphic to a subgroup of $\mathbb{T}$ by Exercise 4.49. Thus $N = p^n \mathbb{J}_p$ for some $n \in \mathbb{N}_+$. Since
N = ker ηn, we conclude with (b) of Corollary 7.6 that χ = kηn for some k ∈ Z. This proves that χ = H and
consequently $\hat{\mathbb{J}_p} \cong \mathbb{Z}(p^{\infty})$.


    The isomorphism $g : \mathbb{\hat{Z}} \to \mathbb{T}$ is obtained by setting $g(\chi) := \chi(1)$ for every $\chi : \mathbb{Z} \to \mathbb{T}$. It is easy to check that
this isomorphism is topological.


    According to 7.5 every $\chi \in \mathbb{T}$ has the form $\chi = k · id_{\mathbb{T}}$ for some $k \in \mathbb{Z}$. This gives a homomorphism $\hat{\mathbb{T}} \to \mathbb{Z}$
assigning $\chi \mapsto k$. It is obviously injective and surjective. This proves $\hat{\mathbb{T}} \cong \mathbb{Z}$ since both groups are discrete.


    To prove $\hat{\mathbb{R}} \cong \mathbb{R}$ consider the character $\chi_1 : \mathbb{R} \to \mathbb{T}$ obtained simply by the canonical map $\mathbb{R} \to \mathbb{R} / \mathbb{Z}$. For
every non-zero $r \in \mathbb{R}$ consider the map $\rho r : \mathbb{R} \to \mathbb{R}$ defined by $\rho r(x) = rx$. Then its composition $\chi_r = \chi_1 \circ \rho_r$
with $\chi_1$ gives a continuous character of $\mathbb{R}$ that is surjective and $ker \chi_r = \langle 1/r \rangle$. Now consider any continuous
non-trivial character $\chi \in \hat{\mathbb{R}}$. Then $\chi$ is surjective and $N = ker \chi$ is a proper closed subgroup of $\mathbb{R}$. Hence $N$
is cyclic by Exercise 3.20. Let $N = \langle 1/r \rangle$. Then $ker \chi = ker \chi_r$, so that Corollary 7.6 yields $\chi = \pm \chi_r$. The
assignment $\chi \mapsto \pm r$ defines a homomorphism $\hat{\mathbb{R}} \to \mathbb{R}$ that is obviously injective and surjective. Its continuity
immediately follows from the definition of the compact-open topology of $\hat{\mathbb{R}}$. As $\mathbb{R}$ is $\sigma$-compact, this isomorphism
is also open by the open mapping theorem.


\textbf{Exercise 7.8.} Let $G$ be an abelian group and $p$ be a prime. Prove that


\begin{itemize}

    \item (a) $\chi \in p \hat{G}$ iff $\chi(G[p]) = 0$.

    \item (b) $p \chi = 0$ in $\hat{G}$ iff $\chi (pG) = 0$.

\end{itemize}


Conclude that


\begin{itemize}

    \item a discrete abelian group $G$ is divisible (resp., torsion-free) iff $\hat{G}$ is torsion-free (resp., divisible)

    \item  the groups $\hat{\mathbb{Q}}$ and $\hat{\mathbb{Q}_p}$ are torsion-free and divisible.

\end{itemize}


\textbf{Exercise 7.9.} Let $G$ be a totally disconnected locally compact abelian group. Prove that $ker \chi$ is an open
subgroup of $G$ for every $\chi \in \hat{G}$.


    (Hint. Use the fact that by the continuity of $\chi$ and the total disconnectedness of $G$ there exists an open
subgroup $O$ of $G$ such that $\chi(O) \subseteq \wedge_1$.)


\textbf{Exercise 7.10.} Let $p$ be a prime. Prove that $\hat{\mathbb{Q}_p} \cong \mathbb{Q}_p$, where $\mathbb{Q}_p$ denotes the field of all $p$-adic numbers.


    (Hint. Fix $N = {\chi \in \hat{\mathbb{Q}_p} : ker \chi \geq \mathbb{J}_p}$. By the compactness of $\mathbb{J}_p$, conclude that $N$ is an open subgroup of $\mathbb{Q}_p$
topologically isomorphic to $\mathbb{J}_p$ using Exercise 7.9 and Corollary 7.6 (c). For every $n \in \mathbb{N}_+$ let $\xi_n : \mathbb{Q}_p \to \mathbb{Q}_p/p^n \mathbb{J}_p$
be the canonical homomorphism. As $\mathbb{Q}_p/p^{-n} \mathbb{J}_p \cong \mathbb{Z}(p^{\infty}) \leq \mathbb{T}$, we can consider $\xi_n \in \hat{\mathbb{Q}_p}$. Show that $p \xi_{n+1} = \xi_n$
for $n \in \mathbb{N}_+$ and $p \xi_1 \in N$. The subgroup of $\hat{\mathbb{Q}_p}$ generated by $N$ and $(\xi_n)$ is isomorphic to $\mathbb{Q}_p$. Using Corollary
7.6 (c) and Exercise 7.9 deduce that it coincides with the whole group $\hat{\mathbb{Q}_p}$.)


\textbf{Exercise 7.11.} Let $H$ be a subgroup of $\mathbb{R}^n$. Prove that every $\chi \in \hat{H}$ extends to a continuous character of $\mathbb{R}^n$.


\subsection{Some general properties of the dual}
We prove next that the dual group of a finite product of abelian topological groups is the product of the dual
groups of each group.


\textbf{Lemma 7.12.} If $G$ and $H$ are topological abelian groups, then $\hat{G \times H}$ is isomorphic to $\hat{G} \times \hat{H}$.


\emph{Proof.} Define $\Phi : \hat{G} \times \hat{H} \to \hat{G \tiems H}$ by $\Phi(\chi_1, \chi_2)(\chi_1, x_2) = \chi_1(x_1) + \chi_2(x_2)$ for every $(\chi_1, \chi_2) \in \hat{G} \times \hat{H}$ and
$(x_1, x_2) \in G \times H$. Then $\Phi$ is a homomorphism, in fact $\Phi(\chi_1+\psi_1, \chi_2+\psi_2)(x_1, x_2) = (\chi_1+\psi_1)(x_1)+(\chi_2+\psi_2)(x_2) =$
$\chi_1 (x_1) + \psi_1 (x_1) + \chi_2 (x_2) + \psi_2 (x_2) = \Phi (\chi_1, \chi_2)(x_1, x_2) + \Phi (\psi_1, \psi_2)(x_1, x_2)$.


    Moreover $\Phi$ is injective, because


        $ker \Phi = {(\chi, \psi) \in \hat{G} \times \hat{H} : \Phi (\chi, \psi) = 0}$ 
            $ = {(\chi, \psi) \in \hat{G} \times \hat{H} : \Phi (\chi, \psi)(x, y) = 0 \text{ for every } (x, y) \in G \times H}$
            $ = {(\chi, \psi) \in \hat{G} \times \hat{H} : \chi(x) + \psi(y) = 0 \text{ for every } (x, y) \in G \times H}$
            $ = {(\chi, \psi) \in \hat{G} \times \hat{H} : \chi(x) = 0 \text{ and } \psi(x) = 0 \text{ for every } (x, y) \in G \times H}$
            $ = {(0, 0)}$.


    To prove that $\Phi$ is surjective, take $\psi \in \hat{G \times H}$ and note that $\psi(x_1, x_2) = \psi(x_1, 0) + \psi(0, x_2)$. Now define
$\psi_1(x_1) = \psi(x_1, 0)$ for every $x_1 \in G$ and $\psi_2(x_2) = \psi(0, x_2)$ for every $x_2 \in H$. Hence $\psi_1 \in \hat{G}$, $\psi_2 \in \hat{H}$ and
$\psi = \Phi(\psi_1, \psi_2)$.


    Now we show that $\Phi$ is continuous. Let $W(K, U)$ be an open neighborhood of $0$ in $\hat{G \times H}$ ($K$ is a compact
subset of $G \times H$ and $U$ is an open neighborhood of 0 in $\mathbb{T}$). Since the projections $\pi_G$ and $\pi_H$ of $G \times H$ onto
$G$ and $H$ are continuous, $K_G = \pi_G (K)$ and $K_H = \pi_H (K)$ are compact in $G$ and in $H$ respectively. Taking an
open symmetric neighborhood $V$ of 0 in $\mathbb{T}$, it follows $\Phi(W(K_G, V ) \times W(K_H, V )) \subseteq W(K, U)$.


    It remains to prove that Φ is open. Consider two open neighborhoods $W(K_G, U_G)$ of 0 in $\hat{G}$ and $W(K_H, U_H)$
of 0 in $\hat{H}$, where $K_G \subseteq G$ and $K_H \subseteq H$ are compact and $U_G, U_H$ are open neighborhoods of 0 in $\mathbb{T}$. Then
$K = (K_G \cup \{0\}) \times (K_H \cup \{0\})$ is a compact subset of $G \times H$ and $U = U_G \cap U_H$ is an open neighborhood of
0 in $\mathbb{T}$. Thus $W(K, U) \subseteq \Phi(W(K_G, U_G) \times W(K_H, U_H))$, because if $\chi \in W(K, U)$ then $\chi = \Phi(\chi_1, \chi_2)$, where
$χ1(x1) = \chi(x_1, 0) \in U \subseteq U_G$ for every $x_1 \in G$ and $\chi_2(x_2) = \chi(0, x_2) \in U \subseteq U_H$ for every $x_2 \in H$.


    It follows from Proposition 7.7 that the groups $\mathbb{T}, \mathbb{Z}, \mathbb{Z}(p^{\infty}), \mathbb{J}_p e \mathbb{R}$ satisfy $\hat{\hat{G}} \cong G$, namely the 
Pontryaginvan Kampen duality theorem. Using the next theorem this propertiy extends to all finite direct products of
these groups.


    Call a topological abelian group $G$ autodual, if $G$ satisfies $\hat{G} \cong G$. We have seen already that $\mathbb{R}$ and $\mathbb{Q}_p$ are
autodual. By Lemma 7.12 finite direct products of autodual groups are autodual. Now using this observation
and Lemma 7.12 we provide a large supply of groups for which the Pontryagin-van Kampen duality holds true.


\textbf{Proposition 7.13.} Let $P_1, P_2$ and $P_3$ be finite sets of primes, $m, n, k, k_p \in N (p \in P_3)$ and
$n_p, m_p \in \mathbb{N}_+ (p \in P_1 \cup P_2)$. Then every group of the form


    $G = \mathbb{T}^n \times \mathbb{Z}^m \times \mathbb{R}^k \times F \times \Pi_{p \in P_1} \mathbb{Z} (p^{\infty})^{n_p} \times \Pi_{p \in P_2} \mathbb{J}_{p}^{m_p} \times \Pi_{j \in P_3} \mathbb{Q}_{p}^{k_p}$


where $F$ is a finite abelian group, satisfies $\hat{\hat{G}} \cong G$


Moreover, such a group is autodual iff $n = m$, $P_1 = P_2$ and $n_p = m_p$ for all $p \in P_1 = P_2$. In particular,
$\hat{\hat{G}} \cong G$ holds true for all elementary locally compact abelian groups.


\emph{Proof.} Let us start by proving $\hat{F} = F^* \cong F$. Recall that $F$ has the form $F \cong \mathbb{Z}_{n_1} \times . . . \times \mathbb{Z}_{n_m}$. So applying
Theorem 7.14 we are left with the proof of the isomorphism $\mathbb{Z}^*_n \cong \mathbb{Z}_n$ for every $n \in \mathbb{N}_+$. The elements $x$ of
$T$ satisfying $nx = 0$ are precisely those of the unique cyclic subgroup of order $n$ of $\mathbb{T}$, we shall denote that
subgroup by $\mathbb{Z}_n$. Therefore, the group $Hom(\mathbb{Z}_n,\mathbb{Z}_n)$ of all homomorphisms $\mathbb{Z}_n \to \mathbb{Z}_n$ is isomorphic to $\mathbb{Z}_n$.


It follows easily from Lemma 7.12 that if $\hat{\hat{G}}_i \cong G_i$ (resp., $\hat{G}_i \cong G_i$) for a finite family $\{G_i\}^n_{i=1}$ of topological
abelian groups, then also $G = \Pi^n_{i=1} G_i$ satisfies $\hat{\hat{G}}_i \cong G$ (resp., $\hat{G}_i \cong G$). Therefore, it suffices to verify that the
groups $\mathbb{T}, \mathbb{Z}, \mathbb{Z}(p^{\infty})$, and $\mathbb{J}_p e$ satisfy $\hat{\hat{G}} \cong G$, while $\hat{\mathbb{R}} \cong \mathbb{R}$, $\hat{\mathbb{Q}_p} \cong \mathbb{Q}_p$ were already checked.


    It follows from Proposition 7.7 that $\hat{\mathbb{Z}} \cong \mathbb{T}$ and $\hat{\mathbb{T}} \cong \mathbb{Z}$, hence $\mathbb{Z} \cong \hat{\hat{\mathbb{Z}}}$ and $\mathbb{T} \cong \hat{\hat{\mathbb{T}}}$. Analogously, $\hat{\mathbb{Z} (p^{\infty})} \cong \mathbb{J}_p$
and $\hat{\mathbb{J}_p} \cong \mathbb{Z}(p^{\infty})$ yield $\mathbb{Z}(p^{\infty}) \cong \hat{\hat{\mathbb{Z}(p^{\infty})}}$ and $\mathbb{J}_p \cong \hat{\hat{\mathbb{J}_p}}$.


    The problem of characterizing all autodual locally compact abelian groups is still open [47, 48].


\textbf{Theorem t.14.} Let $\{D_i\}_{i \in I}$ be a family of discrete abelian groups and let $\{G_i\}_{i \in I}$ be a family of compact
abelian groups. Then


    $\hat{\bigoplus D_i}_{i \in \I} \cong \Pi_{i \in I} \hat{D_i}$ and $\hat{\Pi_{i \in I} G_i} \cong \bigoplus_{i \in I} \hat{G_i}$ (5)


\emph{Proof.} Let $\chi : \oplus_{i \in I} D_i \to \mathbb{T}$ be a character and let $\chi_i: D_i \to \mathbb{T}$ be its restriction to $D_i$. Then $\chi \mapsto (\chi_i) \in \Pi_{i \in I} \hat{D_i}$
is the first isomorphism in (5).


    Let $\chi : \Pi_{i \in I} G_i \to \mathbb{T}$ be a continuous character. Pick a neighborhood $U$ of 0 containing no non-trivial
subgroups of $\mathbb{T}$. Then there exists a neighborhood $V$ of 0 in $G = \Pi_{i \in I} G_i$ with $\chi(V) \subseteq U$. By the definition of
the Tychonov topology there exists a finite subset $F \subseteq I$ such that $V$ contains the subproduct $B = Q_{i \in I/F} G_i$.
Being $\chi(B)$ a subgroup of $\mathbb{T}$, we conclude that $\chi(B) = 0$ by the choice of $U$. Hence $\chi$ factorizes through the
projection $p : G \to \Pi_{i \in F} G_i = G/B$; so there exists a character $\chi' : \Pi_{i \in F} G_i \to \mathbb{T}$ such that $\chi = \chi' \circ p$.
Obviously, $\chi' \in \oplus_{i \in I} \hat{G_i}$. Then $\chi \mapsto \chi'$ is the second isomorphism in (5).


    In order to extend the isomorphism (5) to the general case of locally compact abelian groups one has to
consider a specific topology on the direct sum.


    Algebraic properties of the dual group $\hat{G}$ of a compact abelian group $G$ can be described in terms of
topological properties of the group $G$. We saw in Corollary 6.22 that $\hat{G}$ is torsion precisely when $G$ is totally
disconnected. Here is the counterpart of this property in the connected case:


\textbf{Proposition 7.15.} Let $G$ be a topological abelian group.


\begin{itemize}

    \item (a) If $G$ is connected, then the dual group $\hat{G}$ is torsion-free.

    \item (b) If $G$ is compact, then the dual group $\hat{G}$ is torsion-free iff $G$ is connected.

\end{itemize}


\emph{Proof.} (a) Since for every non-zero continuous character $\chi : G \to \mathbb{T}$ the image $\chi(G)$ is a non-trivial connected
subgroup of $\mathbb{T}$, we deduce that $\chi(G) = \mathbb{T}$ for every non-zero $\chi \in \hat{G}$. Hence $\hat{G}$ is torsion-free.


    (b) If the group $G$ is compact and disconnected, then by Theorem 4.19 there exists a proper open subgroup
$N$ of $G$. Take any non-zero character $\xi$ of the finite group $G/N$. Then $m \xi = 0$ for some positive integer $m$.
Now the composition $\chi$ of $\xi$ and the canonical homomorphism $G \to G/N$ satisfies $m\chi = 0$ as well. So $\hat{G}$ haas a
non-zero torsion character. This proves the implication left open by item (a).


    Let $G$ and $H$ be abelian topological groups. If $f : G \to H$ is a continuous homomorphism, define $\hat{f}: \hat{H} \to \hat{G}$
putting $\hat{f}(\chi) = \chi \circ f$ for every $\chi \in \hat{H}$.

\maketitle\end{document}
