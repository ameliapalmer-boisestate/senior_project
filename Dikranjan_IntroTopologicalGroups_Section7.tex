\documentclass[12pt]{article}
\usepackage{amsmath}
\usepackage{graphicx}
\usepackage{hyperref}
\usepackage[utf8]{inputenc}
\usepackage{mathrsfs}
\usepackage{amssymb}

\title{Introduction to Topological Groups}
\author{Dikran Dikranjan}

\begin{document}

\section{Pontryagin-van Kampen duality}

\subsection{The Dual Group}


In the sequel we shall write the circle additively as $(\mathbb{T}, +)$ and we denote by $q_0 : \mathbb{R} \to \mathbb{T} = \mathbb{R}/\mathbb{Z}$ the canonical
projection. For every $k \in \mathbb{N}_+$ let $\wedge_k = q_0((\frac{-1}{3k},\frac{1}{3k}))$. Then ${\wedge_k : k \in \mathbb{N}_+}$ is a base of the neighborhoods of 0
in $\mathbb{T}$, because ${(\frac{-1}{3k}, \frac{1}{3k}) : k \in \mathbb{N}_+}$ is a base of the neighborhoods of 0 in $\mathbb{R}$.


    For every abelian group $G^* = Hom (G,\mathbb{T})$. For a subset $K$ of $G$ and a subset $U$ of $\mathbb{T}$ let


        $W_{G^*} (K, U) = \{\chi \in G : \chi(K) \subseteq U\}$.


    For any subgroup $H$ of $G^*$ we abbreviate $H \cap W(K, U)$ to $W_H (K, U)$. When there is no danger of confusion
we shall write only $W(K, U)$ in place of $W_{G^*} (K, U)$. The group $G^*$ will be considered only with one topology,
namely the induced from ${T}^G$ compact topology (see Remark 4.1).


    If $G$ is a topological abelian group, $\hat{G}$ will denote the subgroup of $G^*$ consisting of continuous characters.


    The group $\hat{G}$ will carry the compact open topology that has as basic neighborhoods of 0 the sets $W_{\hat{G}}(K, U)$,
where $K$ is a compact subset of $G$ and $U$ is neighborhood of 0 in $\mathbb{T}$. We shall see below that when $U \subseteq \wedge_1$,
then $W_{\hat{G}}(K, U)$ coincides with $W_{G^*} (K, U)$ in case $K$ is a neighborhood of 0 in $G$. Therefore we shall use mainly
the notation $W(K, U)$ when the group $G$ is clear from the context.


    Let us start with an easy example.


\textbf{Example 7.1.} Let $G$  be an abelian topological group.


    \begin{enumerate}

        \item If $G$ is compact, then $\hat{G}$ is discrete.

        \item If $G$ is discrete, then $\hat{G}$ is compact.

    \end{enumerate}


    Indeed, to prove (1) it is sufficient to note that $W_{\hat{g}}(G,\wedge_1) = {0}$ as $\wedge_1$ contains no subgroup of $\mathbb{T}$ beyond 0


    (2) Suppose that $G$ is discrete. Then $\hat{G} = Hom(G,\mathbb{T})$ is a subgroup of the compact group $T^G$. The 
compactopen topology of $\hat{G}$ coincides with the topology inherited from $\mathbb{T}^G$: let $F$ be a finite subset of $G$ and $U$ an open
neighborhood of 0 in $\mathbb{T}$, then


    $\bigcap_{x \in F} π^{-1}_x (U) \cap Hom (G,T) = {\chi \in Hom(G,\mathbb{T}) : \pi_x \in U \text{ for every } x \in F}$


        $= {\chi \in Hom (G,\mathbb{T}) : \chi(x) \in U \text{ for every } x \in F} = W(F, U)$.


    Moreover $Hom (G,\mathbb{T})$ is closed in the compact product $\mathbb{T}^G$ by Remark 4.1 and we can conclude that $\hat{G}$ is
compact.


    Now we prove that the dual group is always a topological group. If the group $G$ is locally compact, then its
dual is locally compact too. This is the first step of the Pontryagin-van Kampen duality theorem.


\textbf{Theorem 7.2}. For an abelian topological group $G$ the following assertions hold true:


    \begin{itemize}

        \item (a) if $x \in \mathbb{T}$ and $k \in \mathbb{N}_+$, then $x \in \wedge_k$ if and only if $x, 2x, . . . , kx \in \wedge_1$;
        
        \item (b) $\chi \in Hom (G,T)$ is continuous if and only if $\chi^{-1} (\wedge_1)$ is a neighborhood of 0 in $G$;
        
        \item (c) $\{W_{\hat{G}}(K,\wedge_1) : K \text{ compact } \subseteq G\}$ is a base of the neighborhoods of 0 in $\hat{G}$, in particular $\hat{G}$ is a topological
        group.        
        
        \item (d) $W_{\hat{G}}(A,\wedge_s) + W_{\hat{G}}(A,\wedge_s) \subseteq W_{\hat{G}}(A,\wedge_{s-1})$ and $W_{\hat{G}}(\bar{A},\wedge_s) + W_{\hat{G}}(\bar{A},\wedge_s) ⊆ W_{\hat{G}}(\bar{A},\wedge_{s-1})$ for every $A \subseteq G$
        and $s > 1$.        
        
        \item (e) if $F$ is a closed subset of $\mathbb{T}$, then for every $K \subseteq G$ the subset $W_{G^*}(K, F)$ of $G^*$ is closed (hence, compact);        
        
        \item (f) if $U$ is neighborhood of 0 in $G$, then
            
            \item (f_1) $W_{G^*}(\bar{U}, V) = W_{G^*} (\bar{U}, V)$ for every neighborhood of 0 $V \subseteq \wedge_1$ in $T$;
            
            \item (f_2) $W(\bar{U},\wedge_4)$ has compact closure;
            
            \item (f_3)  if $U$ has compact closure, then $W(\bar{U},\wedge_4)$ is a neighborhood of 0 in $\hat{G}$ with compact closure, so $\hat{G}$ is
            locally compact.

    \end{itemize}


    \emph{Proof.} (a) Note that for $s \in \mathbb{N}$, $sx \in \wedge_1$ if and only if $x \in A_{s,t} = \wedge_s+\pi_{\mathbb{T}}(\frac{t}{s})$ for some integer $t$ with $0 \leq t \leq s$. On
the other hand, $A_s, 0 = \wedge_s$ and $\wedge_s \cap A_{s+1,t}$ is non-empty if and only if $t = 0$. Hence, if $x \in \wedge_s$ and $(s+ 1)x \in \wedge_1$,
then $x \in \wedge_{s+1}$ and this holds in particular for $1 \leq s < k$. This proves that $sx \in \wedge_1$ for $s = 1, . . . , k$ if and only
if $x \in \wedge_k$.


    (b) Suppose that $\chi^{-1}(\wedge_1)$ is a neighborhood of 0 in $G$. So there exists an open neighborhood $U$ of 0 in $G$
such that $U \subseteq \chi_{-1}(\wedge_1)$. Moreover, there exists an other neighborhood $V$ of 0 in $G$ with $\underbrace{V + · · · + V}_k \subseteq U$ where
$k \in \mathbb{N}_+$. Now $\chi(y) \in \wedge_1$ for every $y \in V$ and $s = 1, . . . , k$. By item (a) $\chi(y) \in \wedge_k$ and so $\chi(V) \subseteq \wedge_k$.


    (c) Let $k \in \mathbb{N}_+$ and $K$ be a compact subset of $G$. Define $L = \underbrace{K + · · · + K}_k$, which is a compact subset of
$G$ because it is a continuous image of the compact subset $K^k$ of $G^k$. Take $\chi \in W(L,\wedge_1)$. For every $x \in K$ we
have $s\chi(x) \in \wedge_1$ for $s = 1, . . . , k$ and so $\chi(x) \in \wedge_k$ by item (a). Hence $W(L,\wedge_1) \subseteq W(K,\wedge_k)$.


    (d) obvious.


    (e) If $\pi_x : \mathbb{T}^G \to \mathbb{T}$ is the projection defined by the evaluation at $x$, for $x \in G$, then obviously


    $W_{G^*} (K, F) = \bigcap_{x \in K} \{\chi \in G^*: \chi(x) \in F\} = \bigcap_{x \in K}\pi^{-1}_{x}(F)$


is cloased as each $\pi^{-1}_{x}(F)$ is closed in $G^*$.


    (f1) follows immediately from item (c).


    (f2) To prove that the closure of $W0 = W(U,Λ4)$ is compact it is sufficient to note that $W_0 \subseteq W_1 := W(\bar{U},\bar{\wedge_4})$
and prove that $W_1$ is compact. Let $\tau_s$ denote the subspace topology of W1 in Gb. We prove in the sequel that
(W1, τs) is compact.


    Consider on the set $W_1$ also the weaker topology $\tau$ induced from $G^*$ and consequently from $\mathbb{T}^G$. By (e)
$(W_1, \tau)$ is compact.


    It remains to show that both topologies $\tau_s$ and $\tau$ of $W_1$ coincide. Since $\tau_s$ is finer than $\tau$ , it suffices to show
that if $\alpha \in W_1$ and $K$ is a compact subset of $G$, then $(\alpha + W(K,\wedge_1)) \cap W_1$ is also a neighborhood of $\alpha$ in
$(W_1, \tau)$


    Since $\bigcup{a + U : a \in K} \supseteq K$ and $K$ is compact, $K \subseteq F + U$, where $F$ is a finite subset of $K$. We prove
now that


    $(\alpha + W(F,\wedge_2)) \cap W_1 \subseteq (\alpha + W(K,\wedge_1)) \cap W_1$. (*) 


    Let $\xi \in W(F,\wedge_2)$, so that $\alpha + \xi' \in W_1 = W(\bar{U},\bar{\wedge_4})$. As $\alpha \in W_1$ as well, we deduce from items (c) and (d) that
$\xi = (\alpha + \xi') - \alpha \in W_1 - W_1$. Hence $\xi(\bar{U}) \subseteq Λ_2$ and consequently
    

    $\xi(K) \subseteq \xi(F + U) \subseteq \wedge_2 + \bar{\wedge_2} \subseteq \wedge_1.$


This proves $\xi \in W(K,\wedge_1)$ and (*).


    (f3) Follows obviously from (f2) and the definition of the compact open topology


    The above proof shows another relevant fact. The neighborhood $W(\bar{U},\wedge_4)$ of 0 in the dual group $\hat{G}$ carries
the same topology in $\hat{G}$ and $G^*$, nevertheless the inclusion map $j : \hat{G} , \hookrightarrow G^*$ need not be an embedding:
    

\textbf{Corollary 7.3}. For a locally compact abelian group $G$ the following are equivalent:


    \begin{itemize}

        \item the inclusion map $j : \hat{G} , b \to G^*$ is an embedding;

        \item $G$ is discrete;

        \item $\hat{G} = G^*$ is compact

    \end{itemize}


Proof. Since $G^*$ is compact, $j$ can be an embedding iff $\hat{G}$ itself is compact. According to Example 7.1 this
occurs precisely when $G$ is discrete. In that case $\hat{G} = G^*$ is compact.


    Actually, it can be proved, once the duality theorem is available, that $j : \hat{G}, \hookrightarrow G^*$ need not be even a local
homeomorphism. (If $j$ is a local homeomorphism, then the topological subgroup $j(\hat{G})$ of $G^*$ will be locally
compact, hence closed in $G^*$. This would yield that $j(G^*)$ is compact. On the other hand, the topology of
$j(G^*)$ is precisely the initial topology of all projections $p_x$ restricted to $\hat{G}$. By the Pontryagin duality theorem,
these projections form the group of all continuous characters of $\hat{G}$. So this topology coincides with $T_{\hat{\hat{G}}}$. By a
general theorem of Glicksberg, a locally compact abelian groups $H$ and $(H, \mathcal{T}_{\hat{H}})$ have the same compact sets.
In particular, compactness of $(H, \mathcal{T}_{\hat{H}})$ yields compactness of $H$. This proves that if $j : \hat{G}, \hookrightarrow G^*$ is a local
homeomorphism, then $\hat{G}$ is compact and consequently $G$ is discrete.)


\subsection{Computation of some dual groups}


In the next proprosition we show, roughly speaking, that the projective order between continuous surjective
open homomorphisms with the same domain corresponds to the order by inclusion of their kernels.


\textbf{Proposition 7.4} Let $G$, $H_1$ and $H_2$ be topological abelian groups and let $\chi_i: G \to H_i, i = 1, 2$, be continuous
surjective open homomorphisms. Then there exists a continuous homomorphism $\iota : H_1 \to H_2$ such that $\chi_2 = \iota \circ \chi_1 $
iff ker $\chi_1 \leq ker \chi_2$. If ker $\chi_1 = ker \chi_2$ then $\iota$ will be a topological isomorphism.


    Proof. The necessity is obvious. So assume that ker $\chi_1 \leq ker \chi_2$ holds. By the homomorphism theorem applied
to $\chi_i$ there exists a topological isomorphisms $j_i: G/ ker \chi_i \to H_i$ such that $\chi_i = j_i \circ q_i$, where $q_i: G \to G/ ker \chi_i$
is the canonical homomorphism for $i = 1, 2$. As $ker \chi_1 \leq ker \chi_2$ we get a continuous homomorphism $t$ that
makes commutative the following diagram

%Note from Skarlet: This diagram is a little difficult to type out so I had to describe it a bit, feel free to contact me for clarification
At the top of the diagram, there is $G$, and $G$ points to all of the elements in the line below like so $G \to^{\chi_1} H_1$, $G \to^{q_1} G / ker \chi_1$, $G \to^{q_2} G / ker \chi_2$, $G \to^{\chi_2} H_2$
$H_1 \leftarrow^{j_1} G / ker \chi_1 \dashrightarrow^t G / ker \chi_2 \rightarrow^{j_2} H_2$
Along the bottom of the diagram, there is an arrow arching from one end to the other $H_1 \to_\iota H_2$


Obviously $\iota = j_2 \circ t \circ j^{-1}_1$ works. If $ker \chi_1 = ker \chi_2$, then t is a topological isomorphism, hence $\iota$ will be a
topological isomorphism as well.


    In the sequel we denote by $k · id_G$ the endomorphism of an abelian group $G$ obtained by the map $x \mapsto kx$,
for a fixed $k \in \mathbb{Z}$. The next lemma will be used for the computation of the dual groups in Example 7.7.


\prrof{Proof.} We prove first that the only topological isomorphisms $\chi : \mathbb{T} \to \mathbb{T}$ are $\pm id_{\mathbb{T}}$. The proof will exploit the
fact that the arcs are the only connected sets of $\mathbb{T}$. Hence $\chi$ sends any arc of $\mathbb{T}$ to an arc, sending end points to
end points. Denote by $\varphi$ the canonical homomorphism $\mathbb{R} \to \mapsto{T}$ and for $n \in \mathbb{N}$ let $c_n = \varphi(1/2^n)$ be the generators
of the Pr¨ufer subgroup $\mathbb{Z}(2^{\infty})$ of $\mathbb{T}$. Then, $c_1$ is the only element of $\mathbb{T}$ of order 2, hence $g(c_1) = c_1$. Therefore,
the arc $A_1 = \varphi([0, 1/2])$ either goes onto itself, or goes onto its symmetric image $-A_1$. Let us consider the first
case. Clearly, either $g(c_2) = c_2$ or $g(c_2) = -c_2$ as $o(g(c_2)) = 4$ and being $\pm c_2$ the only elements of order 4 of $\mathbb{T}$.
By our assumption $g(A_1) = A_1$ we have $g(c_2) = c_2$ since $c_2$ is the only element of order 4 on the arc $A_1$. Now
the arc $A_2 = [0, c_2]$ goes onto itself, hence for $c_3$ we must have $g(c_3) = c_3$ as the only element of order 8 on the
arc $A_2$, etc. We see in the same way that $g(c_n) = c_n$. Hence $g$ is identical on the whole subgroup $\mathbb{Z}(2^{\infty})$. As
this subgroup is dense in $\mathbb{T}$, we conclude that $g$ coincides with $id_{\mathbb{T}}$. In the case $g(A_1) = -A_1$ we replace $g$ by
$-g$ and the previous proof gives $-g = id_{\mathbb{T}}$, i.e., $g = -id_{\mathbb{T}}$.


    For $k \in \mathbb{N}_+$ let $\pi_k = k · id_{\mathbb{T}}$. Then $ker \pi_k = \mathbb{Z}_k$ and $\pi_k$ is surjective. Let now $χ : \mathbb{T} \to \mathbb{T}$ be a non-trivial
continuous homomorphism. Then $ker \chi$ is a closed proper subgroup of $\mathbb{T}$, hence $ker \chi = \mathbb{Z}_k$ for some $k \in N_+$.
Moreover, $\chi(\mathbb{T})$ is a connected non-trivial subgroup of $\mathbb{T}$, hence $\chi(\mathbb{T}) = \mathbb{T}$. By Proposition 7.4 $\chi = \pm \pi_k$.


    Obviously, $\chi = \pm \xi$ for characters $\chi, \xi : G \to \mathbb{T}$ implies $ker \chi = ker \xi$ and $\chi (G) = \xi (G)$. More generally,
if $\chi = k · \xi$ for some $k \in \mathbb{Z}$, then $ker \chi \geq ker \xi$ and $\chi (G) \leq \xi (G)$. Now we see that this implication can be
(partially) inverted under appropriate hypotheses.


\textbf{Corollary 7.6.} Let $G$ be a $\sigma$-compact locally compact abelian group and let $\chi, \xi : G \hookrightarrow T$ be continuous
characters such that $ker \chi \geq ker \xi$ and $\chi (G) \leq \xi (G)$.


\begin{itemize}

    \item (a) If $\chi(G) = \xi(G) \mathbb{T}$ then $\chi = k · \xi$ for some $k \in \mathbb{Z}$; moreover, $ker \chi = ker \xi$ iff $\chi = \pm \xi$

    \item (b) If $G$ is compact and $|\xi (G)| = m$ for some $m \in \mathbb{N}_+$, then $\chi = k \xi$ for some $k \in \mathbb{Z}$; moreover, $ker \chi = ker \xi$
    iff $\chi(G) = \xi(G)$, in such a case $k$ must be coprime to $m$.

    \item (c) If $ker \xi = ker \xi$ is open and H = χ(G) = ξ(G), then χ = ι ◦ ξ, where ι : H → H is an arbitrary
    automorphism of the subgroup H of T equipped with the discrete topology.    

\end{itemize}


\emph{Proof.} (a) As $\chi (G) = \xi (G) = T$ and $G$ is $\sigma$-compact, we can apply Lemma 7.4 and observe that the only $\iota$ given
by the lemma can be $k · id_{\mathbb{T}}$ for some $k \in \mathbb{Z}$ in view of the previous lemma. The same lemma yields $k = \pm 1$
when ker $\chi = ker \xi$.


    (b) If $G$ is compact and $|\xi (G)| = m$ for some $m \in \mathbb{N}_+$, $\xi (G)$ is a cyclic subgroup of $\mathbb{T}$ of order $m$. Note that
$\mathbb{T}$ has a unique such cyclic subgroup. By Proposition 7.4 there exists a homomorphism $\iota : \xi (G) \to \chi (G)$ such
that $\chi = \iota \circ \xi$. The hypothesis $\chi (G) \leq \xi (G)$ implies that there such a $\iota$ must by the multiplication by some
$k \in \mathbb{Z}$. In case $\chi (G) = \xi (G)$ this $k$ is coprime to $m$.


    (c) Obvious.


\textbf{Example 7.7} Let $p$ be a prime. Then $\hat{Z(p^{\infty})} \cong \mathbb{J}_p, \hat{\mathbb{J}_p} \cong \mathbb{Z}(p^{\infty}), \hat{\mathbb{T}} \cong \mathbb{Z}, \hat{\mathbb{Z}} \cong \hat{\mathbb{T}}$ and $\hat{\mathbb{R}} \cong \mathbb{R}$.


Proof. The first isomorphism $\hat{\mathbb{Z}(p^{\infty})} = \mathbb{J}_p$ follows from our definition $\mathbb{J}_p = End(\mathbb{Z}(p^{\infty})) = Hom(\mathbb{Z}(p^{\infty}),\mathbb{T}) = \hat{\mathbb{Z}(p^{\infty})}$.


    To verify the isomorphism $\hat{\mathbb{J}_p} \cong Z(p^{\infty})$ consider first the quotient homomorphism $\eta_n : \mathbb{J}_p \to \mathbb{J}_p / p^n \mathbb{J}_p\cong \mathbb{Z}_{p^n} \leq \mathbb{T}$.
With this identifications we consider $\eta_n \in \hat{\mathbb{J}_p}$. It is easy to see that under this identification $p\eta_n = \eta_{n-1}$.
Therefore, the subgroup $H$ of $\hat{\mathbb{J}_p}$ generated by the characters $\eta_n$ is isomorphic to $\mathbb{Z}(p^\infty)$. Let us see that $H = \hat{\mathbb{J}_p}$.
Indeed, take any non-trivial character $\chi : \mathbb{J}_p \to \mathbb{T}$. Then $\mathbb{N} = ker \chi$ is a closed proper subgroup of $\mathbb{J}_p$. Moreover,
$N \neq 0$ as $\mathbb{J}_p$ is not isomorphic to a subgroup of $\mathbb{T}$ by Exercise 4.49. Thus $N = p^n \mathbb{J}_p$ for some $n \in \mathbb{N}_+$. Since
N = ker ηn, we conclude with (b) of Corollary 7.6 that χ = kηn for some k ∈ Z. This proves that χ = H and
consequently $\hat{\mathbb{J}_p} \cong \mathbb{Z}(p^{\infty})$.


    The isomorphism $g : \mathbb{\hat{Z}} \to \mathbb{T}$ is obtained by setting $g(\chi) := \chi(1)$ for every $\chi : \mathbb{Z} \to \mathbb{T}$. It is easy to check that
this isomorphism is topological.


    According to 7.5 every $\chi \in \mathbb{T}$ has the form $\chi = k · id_{\mathbb{T}}$ for some $k \in \mathbb{Z}$. This gives a homomorphism $\hat{\mathbb{T}} \to \mathbb{Z}$
assigning $\chi \mapsto k$. It is obviously injective and surjective. This proves $\hat{\mathbb{T}} \cong \mathbb{Z}$ since both groups are discrete.


    To prove $\hat{\mathbb{R}} \cong \mathbb{R}$ consider the character $\chi_1 : \mathbb{R} \to \mathbb{T}$ obtained simply by the canonical map $\mathbb{R} \to \mathbb{R} / \mathbb{Z}$. For
every non-zero $r \in \mathbb{R}$ consider the map $\rho r : \mathbb{R} \to \mathbb{R}$ defined by $\rho r(x) = rx$. Then its composition $\chi_r = \chi_1 \circ \rho_r$
with $\chi_1$ gives a continuous character of $\mathbb{R}$ that is surjective and $ker \chi_r = \langle 1/r \rangle$. Now consider any continuous
non-trivial character $\chi \in \hat{\mathbb{R}}$. Then $\chi$ is surjective and $N = ker \chi$ is a proper closed subgroup of $\mathbb{R}$. Hence $N$
is cyclic by Exercise 3.20. Let $N = \langle 1/r \rangle$. Then $ker \chi = ker \chi_r$, so that Corollary 7.6 yields $\chi = \pm \chi_r$. The
assignment $\chi \mapsto \pm r$ defines a homomorphism $\hat{\mathbb{R}} \to \mathbb{R}$ that is obviously injective and surjective. Its continuity
immediately follows from the definition of the compact-open topology of $\hat{\mathbb{R}}$. As $\mathbb{R}$ is $\sigma$-compact, this isomorphism
is also open by the open mapping theorem.


\textbf{Exercise 7.8.} Let $G$ be an abelian group and $p$ be a prime. Prove that


\begin{itemize}

    \item (a) $\chi \in p \hat{G}$ iff $\chi(G[p]) = 0$.

    \item (b) $p \chi = 0$ in $\hat{G}$ iff $\chi (pG) = 0$.

\end{itemize}


Conclude that


\begin{itemize}

    \item a discrete abelian group $G$ is divisible (resp., torsion-free) iff $\hat{G}$ is torsion-free (resp., divisible)

    \item  the groups $\hat{\mathbb{Q}}$ and $\hat{\mathbb{Q}_p}$ are torsion-free and divisible.

\end{itemize}


\textbf{Exercise 7.9.} Let $G$ be a totally disconnected locally compact abelian group. Prove that $ker \chi$ is an open
subgroup of $G$ for every $\chi \in \hat{G}$.


    (Hint. Use the fact that by the continuity of $\chi$ and the total disconnectedness of $G$ there exists an open
subgroup $O$ of $G$ such that $\chi(O) \subseteq \wedge_1$.)


\textbf{Exercise 7.10.} Let $p$ be a prime. Prove that $\hat{\mathbb{Q}_p} \cong \mathbb{Q}_p$, where $\mathbb{Q}_p$ denotes the field of all $p$-adic numbers.


    (Hint. Fix $N = {\chi \in \hat{\mathbb{Q}_p} : ker \chi \geq \mathbb{J}_p}$. By the compactness of $\mathbb{J}_p$, conclude that $N$ is an open subgroup of $\mathbb{Q}_p$
topologically isomorphic to $\mathbb{J}_p$ using Exercise 7.9 and Corollary 7.6 (c). For every $n \in \mathbb{N}_+$ let $\xi_n : \mathbb{Q}_p \to \mathbb{Q}_p/p^n \mathbb{J}_p$
be the canonical homomorphism. As $\mathbb{Q}_p/p^{-n} \mathbb{J}_p \cong \mathbb{Z}(p^{\infty}) \leq \mathbb{T}$, we can consider $\xi_n \in \hat{\mathbb{Q}_p}$. Show that $p \xi_{n+1} = \xi_n$
for $n \in \mathbb{N}_+$ and $p \xi_1 \in N$. The subgroup of $\hat{\mathbb{Q}_p}$ generated by $N$ and $(\xi_n)$ is isomorphic to $\mathbb{Q}_p$. Using Corollary
7.6 (c) and Exercise 7.9 deduce that it coincides with the whole group $\hat{\mathbb{Q}_p}$.)


\textbf{Exercise 7.11.} Let $H$ be a subgroup of $\mathbb{R}^n$. Prove that every $\chi \in \hat{H}$ extends to a continuous character of $\mathbb{R}^n$.


\subsection{Some general properties of the dual}
We prove next that the dual group of a finite product of abelian topological groups is the product of the dual
groups of each group.


\textbf{Lemma 7.12.} If $G$ and $H$ are topological abelian groups, then $\hat{G \times H}$ is isomorphic to $\hat{G} \times \hat{H}$.


\emph{Proof.} Define $\Phi : \hat{G} \times \hat{H} \to \hat{G \tiems H}$ by $\Phi(\chi_1, \chi_2)(\chi_1, x_2) = \chi_1(x_1) + \chi_2(x_2)$ for every $(\chi_1, \chi_2) \in \hat{G} \times \hat{H}$ and
$(x_1, x_2) \in G \times H$. Then $\Phi$ is a homomorphism, in fact $\Phi(\chi_1+\psi_1, \chi_2+\psi_2)(x_1, x_2) = (\chi_1+\psi_1)(x_1)+(\chi_2+\psi_2)(x_2) =$
$\chi_1 (x_1) + \psi_1 (x_1) + \chi_2 (x_2) + \psi_2 (x_2) = \Phi (\chi_1, \chi_2)(x_1, x_2) + \Phi (\psi_1, \psi_2)(x_1, x_2)$.


    Moreover $\Phi$ is injective, because


        $ker \Phi = {(\chi, \psi) \in \hat{G} \times \hat{H} : \Phi (\chi, \psi) = 0}$ 
            $ = {(\chi, \psi) \in \hat{G} \times \hat{H} : \Phi (\chi, \psi)(x, y) = 0 \text{ for every } (x, y) \in G \times H}$
            $ = {(\chi, \psi) \in \hat{G} \times \hat{H} : \chi(x) + \psi(y) = 0 \text{ for every } (x, y) \in G \times H}$
            $ = {(\chi, \psi) \in \hat{G} \times \hat{H} : \chi(x) = 0 \text{ and } \psi(x) = 0 \text{ for every } (x, y) \in G \times H}$
            $ = {(0, 0)}$.


    To prove that $\Phi$ is surjective, take $\psi \in \hat{G \times H}$ and note that $\psi(x_1, x_2) = \psi(x_1, 0) + \psi(0, x_2)$. Now define
$\psi_1(x_1) = \psi(x_1, 0)$ for every $x_1 \in G$ and $\psi_2(x_2) = \psi(0, x_2)$ for every $x_2 \in H$. Hence $\psi_1 \in \hat{G}$, $\psi_2 \in \hat{H}$ and
$\psi = \Phi(\psi_1, \psi_2)$.


    Now we show that $\Phi$ is continuous. Let $W(K, U)$ be an open neighborhood of $0$ in $\hat{G \times H}$ ($K$ is a compact
subset of $G \times H$ and $U$ is an open neighborhood of 0 in $\mathbb{T}$). Since the projections $\pi_G$ and $\pi_H$ of $G \times H$ onto
$G$ and $H$ are continuous, $K_G = \pi_G (K)$ and $K_H = \pi_H (K)$ are compact in $G$ and in $H$ respectively. Taking an
open symmetric neighborhood $V$ of 0 in $\mathbb{T}$, it follows $\Phi(W(K_G, V ) \times W(K_H, V )) \subseteq W(K, U)$.


    It remains to prove that Φ is open. Consider two open neighborhoods $W(K_G, U_G)$ of 0 in $\hat{G}$ and $W(K_H, U_H)$
of 0 in $\hat{H}$, where $K_G \subseteq G$ and $K_H \subseteq H$ are compact and $U_G, U_H$ are open neighborhoods of 0 in $\mathbb{T}$. Then
$K = (K_G \cup \{0\}) \times (K_H \cup \{0\})$ is a compact subset of $G \times H$ and $U = U_G \cap U_H$ is an open neighborhood of
0 in $\mathbb{T}$. Thus $W(K, U) \subseteq \Phi(W(K_G, U_G) \times W(K_H, U_H))$, because if $\chi \in W(K, U)$ then $\chi = \Phi(\chi_1, \chi_2)$, where
$χ1(x1) = \chi(x_1, 0) \in U \subseteq U_G$ for every $x_1 \in G$ and $\chi_2(x_2) = \chi(0, x_2) \in U \subseteq U_H$ for every $x_2 \in H$.


    It follows from Proposition 7.7 that the groups $\mathbb{T}, \mathbb{Z}, \mathbb{Z}(p^{\infty}), \mathbb{J}_p e \mathbb{R}$ satisfy $\hat{\hat{G}} \cong G$, namely the 
Pontryaginvan Kampen duality theorem. Using the next theorem this propertiy extends to all finite direct products of
these groups.


    Call a topological abelian group $G$ autodual, if $G$ satisfies $\hat{G} \cong G$. We have seen already that $\mathbb{R}$ and $\mathbb{Q}_p$ are
autodual. By Lemma 7.12 finite direct products of autodual groups are autodual. Now using this observation
and Lemma 7.12 we provide a large supply of groups for which the Pontryagin-van Kampen duality holds true.


\textbf{Proposition 7.13.} Let $P_1, P_2$ and $P_3$ be finite sets of primes, $m, n, k, k_p \in N (p \in P_3)$ and
$n_p, m_p \in \mathbb{N}_+ (p \in P_1 \cup P_2)$. Then every group of the form


    $G = \mathbb{T}^n \times \mathbb{Z}^m \times \mathbb{R}^k \times F \times \Pi_{p \in P_1} \mathbb{Z} (p^{\infty})^{n_p} \times \Pi_{p \in P_2} \mathbb{J}_{p}^{m_p} \times \Pi_{j \in P_3} \mathbb{Q}_{p}^{k_p}$


where $F$ is a finite abelian group, satisfies $\hat{\hat{G}} \cong G$


Moreover, such a group is autodual iff $n = m$, $P_1 = P_2$ and $n_p = m_p$ for all $p \in P_1 = P_2$. In particular,
$\hat{\hat{G}} \cong G$ holds true for all elementary locally compact abelian groups.


\emph{Proof.} Let us start by proving $\hat{F} = F^* \cong F$. Recall that $F$ has the form $F \cong \mathbb{Z}_{n_1} \times . . . \times \mathbb{Z}_{n_m}$. So applying
Theorem 7.14 we are left with the proof of the isomorphism $\mathbb{Z}^*_n \cong \mathbb{Z}_n$ for every $n \in \mathbb{N}_+$. The elements $x$ of
$T$ satisfying $nx = 0$ are precisely those of the unique cyclic subgroup of order $n$ of $\mathbb{T}$, we shall denote that
subgroup by $\mathbb{Z}_n$. Therefore, the group $Hom(\mathbb{Z}_n,\mathbb{Z}_n)$ of all homomorphisms $\mathbb{Z}_n \to \mathbb{Z}_n$ is isomorphic to $\mathbb{Z}_n$.


It follows easily from Lemma 7.12 that if $\hat{\hat{G}}_i \cong G_i$ (resp., $\hat{G}_i \cong G_i$) for a finite family $\{G_i\}^n_{i=1}$ of topological
abelian groups, then also $G = \Pi^n_{i=1} G_i$ satisfies $\hat{\hat{G}}_i \cong G$ (resp., $\hat{G}_i \cong G$). Therefore, it suffices to verify that the
groups $\mathbb{T}, \mathbb{Z}, \mathbb{Z}(p^{\infty})$, and $\mathbb{J}_p e$ satisfy $\hat{\hat{G}} \cong G$, while $\hat{\mathbb{R}} \cong \mathbb{R}$, $\hat{\mathbb{Q}_p} \cong \mathbb{Q}_p$ were already checked.


    It follows from Proposition 7.7 that $\hat{\mathbb{Z}} \cong \mathbb{T}$ and $\hat{\mathbb{T}} \cong \mathbb{Z}$, hence $\mathbb{Z} \cong \hat{\hat{\mathbb{Z}}}$ and $\mathbb{T} \cong \hat{\hat{\mathbb{T}}}$. Analogously, $\hat{\mathbb{Z} (p^{\infty})} \cong \mathbb{J}_p$
and $\hat{\mathbb{J}_p} \cong \mathbb{Z}(p^{\infty})$ yield $\mathbb{Z}(p^{\infty}) \cong \hat{\hat{\mathbb{Z}(p^{\infty})}}$ and $\mathbb{J}_p \cong \hat{\hat{\mathbb{J}_p}}$.


    The problem of characterizing all autodual locally compact abelian groups is still open [47, 48].


\textbf{Theorem t.14.} Let $\{D_i\}_{i \in I}$ be a family of discrete abelian groups and let $\{G_i\}_{i \in I}$ be a family of compact
abelian groups. Then


    $\hat{\bigoplus D_i}_{i \in \I} \cong \Pi_{i \in I} \hat{D_i}$ and $\hat{\Pi_{i \in I} G_i} \cong \bigoplus_{i \in I} \hat{G_i}$ (5)


\emph{Proof.} Let $\chi : \oplus_{i \in I} D_i \to \mathbb{T}$ be a character and let $\chi_i: D_i \to \mathbb{T}$ be its restriction to $D_i$. Then $\chi \mapsto (\chi_i) \in \Pi_{i \in I} \hat{D_i}$
is the first isomorphism in (5).


    Let $\chi : \Pi_{i \in I} G_i \to \mathbb{T}$ be a continuous character. Pick a neighborhood $U$ of 0 containing no non-trivial
subgroups of $\mathbb{T}$. Then there exists a neighborhood $V$ of 0 in $G = \Pi_{i \in I} G_i$ with $\chi(V) \subseteq U$. By the definition of
the Tychonov topology there exists a finite subset $F \subseteq I$ such that $V$ contains the subproduct $B = Q_{i \in I/F} G_i$.
Being $\chi(B)$ a subgroup of $\mathbb{T}$, we conclude that $\chi(B) = 0$ by the choice of $U$. Hence $\chi$ factorizes through the
projection $p : G \to \Pi_{i \in F} G_i = G/B$; so there exists a character $\chi' : \Pi_{i \in F} G_i \to \mathbb{T}$ such that $\chi = \chi' \circ p$.
Obviously, $\chi' \in \oplus_{i \in I} \hat{G_i}$. Then $\chi \mapsto \chi'$ is the second isomorphism in (5).


    In order to extend the isomorphism (5) to the general case of locally compact abelian groups one has to
consider a specific topology on the direct sum.


    Algebraic properties of the dual group $\hat{G}$ of a compact abelian group $G$ can be described in terms of
topological properties of the group $G$. We saw in Corollary 6.22 that $\hat{G}$ is torsion precisely when $G$ is totally
disconnected. Here is the counterpart of this property in the connected case:


\textbf{Proposition 7.15.} Let $G$ be a topological abelian group.


\begin{itemize}

    \item (a) If $G$ is connected, then the dual group $\hat{G}$ is torsion-free.

    \item (b) If $G$ is compact, then the dual group $\hat{G}$ is torsion-free iff $G$ is connected.

\end{itemize}


\emph{Proof.} (a) Since for every non-zero continuous character $\chi : G \to \mathbb{T}$ the image $\chi(G)$ is a non-trivial connected
subgroup of $\mathbb{T}$, we deduce that $\chi(G) = \mathbb{T}$ for every non-zero $\chi \in \hat{G}$. Hence $\hat{G}$ is torsion-free.


    (b) If the group $G$ is compact and disconnected, then by Theorem 4.19 there exists a proper open subgroup
$N$ of $G$. Take any non-zero character $\xi$ of the finite group $G/N$. Then $m \xi = 0$ for some positive integer $m$.
Now the composition $\chi$ of $\xi$ and the canonical homomorphism $G \to G/N$ satisfies $m\chi = 0$ as well. So $\hat{G}$ haas a
non-zero torsion character. This proves the implication left open by item (a).


    Let $G$ and $H$ be abelian topological groups. If $f : G \to H$ is a continuous homomorphism, define $\hat{f}: \hat{H} \to \hat{G}$
putting $\hat{f}(\chi) = \chi \circ f$ for every $\chi \in \hat{H}$.


\textbf{Lemma 7.16.} If $f:G \to H$ is a continuous homomorphism of topological abelian group, then $\hat{f} : \hat{H} \to \hat{G}$ is
a continuous homomorphism as well.


\begin{itemize}

    \item (a) If $f(G)$ is dense in $H$, then $\hat{f}$ is injective.

    \item (b) If $f$ is injective and $f(G)$ is either open or dense in $H$, then $\hat{f}$ is surjective.

    \item (c) if $f$ is a surjective homomorphism, such that every compact subset of $H$ is covered by some compact subset
    of $G$, then $\hat{f}$ is an embedding.

    \item (d) if $f$ is a quotient homomorphism and $G$ is locally compact, then $\hat{f}$ is an embedding.

    \item (e) If $f$ is a topological isomorphism, then $\hat{f}$ is a topological isomorphism too.

\end{itemize}


\emph{Proof.} Assume $K$ is a compact subset of $G$ and $U$ a neighborhood of 0 in $\mathbb{T}$. Then $f(K)$ is a compact set in $H$,
so $W = W_{\hat{G}}(f(K), U)$ is a neighborhood of 0 in $\hat{H}$ and $\hat{f}(W) \subseteq W(K, U)$. This proves the continuity of $\hat{f}$.


\begin{itemize}

    \item (a) If $\hat{f}(\chi) = 0$, then $\chi \circ f = 0$. By the density of $f(G)$ in $H$ this yields $\chi = 0$.

    \item (b) Let $\chi \in \hat{G}$. If $f(G)$ is open in $H$, then any extension $\xi : H \to \mathbb{T}$ of $\chi$ will be continuous on $f(G)$. There
    exists at least one such extension $\xi$ by Corollary 2.6. Hence $\xi \in Hb$ and $\chi = \hat{f}(\xi)$. Now consider the case
    when $f(G)$ is dense in $H$. Then $\tilde{H} = \tilde{G}$ and the characters of $H$ can be extended to characters of $G$ (see
    Theorem 3.79).

    \item (c) Assume $L$ is a compact subset of $G/H$ and $U$ a neighborhood of 0 in $\mathbb{T}$. Let $K$ be a compact set in $G$
    such that $f(K) = L$. Then $\hat{f}(W_{\hat{H}} (L, U)) = Im \hat{f} \cap W_{\hat{G}} (K, U)$, so $\hat{f}$ is an embedding.    

    \item (d) Follows from (c) and Lemma 4.6.

    \item (e) Obvious.

\end{itemize}


\textbf{Exercise 7.17.} Prove that $\hat{\mathbb{G}/\mathbb{Z}} \cong \Pi_p \mathbb{J}_p$


(Hint. Use the isomorphism $\mathbb{Q}/\mathbb{Z} \cong \oplus_p \mathbb{Z}(p^{\infty})$, Example 7.7 and Theorem 7.14.)
Now we shall see that the group $\mathbb{Q}$ satisfies the duality theorem (see item (b) below).


\textbf{Example 7.18.} Let $K$ denote the compact group $\hat{\mathbb{Q}}$. Then: 


    (a) $K$ contains a closed subgroup $H$ isomorphic to $\hat{\mathbb{Q}/\mathbb{Z}}$ such that $K/H \cong \mathbb{T}$;

    (ii) $\hat{K} \cong \mathbb{Q}$


    (a) Denote by $H$ the subgroup of all $\chi \in K$ such that $\chi{\mathbb{Z}} = 0$. To prove that $H$ is a closed subgroup of $K$ such
that $K/H$ is isomorphic to $\mathbb{T}$. To this end consider the continuous map $\rho : K \to \hat{Z}$ obtained by the restriction
to $\mathbb{Z}$ of every $\chi \in K$. Obviously, $ker \rho = H$, so $\mathbb{T} \cong \hat{\mathbb{Z}} \cong K/H$. To see that $H \cong \hat{\mathbb{Q}/\mathbb{Z}}$ note that the characters
of $\mathbb{Q}/\mathbb{Z}$ correspond precisely to those characters of $\mathbb{Q}$ that vanish on $\mathbb{Z}$, i.e., precisely $H$.

    
    (b) By Exercise 7.8 $K$ is a divisible torsion-free group, every non-zero $r \in \mathbb{Q}$ defines a continuous automorphism
$\lambda_r$ of $K$ by setting $\lambda_r(x) = rx$ for every $x \in K$. Then the composition $\rho \circ \lambda_r : K \to T$ defines a character
$\chi_r \in \hat{K}$ with $ker \chi_r = r^{-1} H$. For the sake of completeness let $\chi_0 = 0$. By Exercise 7.17 $\hat{\mathbb{Q}/\mathbb{Z}} \cong \Pi_p \mathbb{J}_p$ is totally
disconnected, so by Corollary 6.21 H has no surjective characters $\chi : H \to \mathbb{T}$. Now let $\chi \xi \hat[K]$ be non-zero.
Then $\chi(K)$ will be a non-zero closed divisible subgroup of $\mathbb{T}$, hence $\chi(K) = \mathbb{T}$. On the other hand, $N = ker \chi$ is
a proper closed subgroup of $K$ such that $N + H \neq \mathbb{T}$, as $\chi(H)$ is a proper closed subgroup of $\mathbb{T}$ by the previous
argument. Hence, $\chi(H)$ is finite, say of order $m$. Then $N + H$ contains $N$ is a finite-index subgroup, more
precisely $[H : (N \cap H)] = [(N + H) : N] = m$. Then $mH \leq \mathbb{N}$. Consider the character $\chi_{m-1}$ of $K$ having
ker $\chi_{m-1} = mH \leq \mathbb{N}$. By Corollary there exists $k \in \mathbb{Z}$ such that $\chi = k\chi_{m-1} = \chi_r$, where $r = km^{-1} \in \mathbb{Q}$. This
shows that $\mathbb{K} = {\chi_r : r \in \mathbb{Q}} \cong \mathbb{Q}$.


    The compact group $\hat{\mathbb{Q}}$ is closely related to the adele rings of the field $\mathbb{Q}$, more detail can be found in
[34, 38, 75, 97].
    

\textbf{Exercise 7.19.} Prove that a discrete abelian group $G$ satisfies $\hat{\hat{G}} \cong G$ whenever

\begin{itemize}

    \item (a) $G$ is divisible;

    \item (b) $G$ is free;

    \item (c) $G$ is of finite exponent;

    \item (d) $G$ is torsion and every primary component of $G$ is of finite exponent.

\end{itemize}

    (Hint. (a) Use Examples 7.7 and 7.18 (b) and the fact that every divisible group is a direct sum of copies of
Q and the groups $\mathbb{Z}(p^{\infty})$.


    (c) and (d) Use that fact that every abelian group of finite exponent is a direct sum of cyclic subgroups (i.e.,
Prufer's theorem, see (d) of Example 2.3)).


\subsection{The natural transformation $\omega$}


Let $G$ be a topological abelian group. Define $\omega_G : G \to \hat{\hat{G}}$ such that $\omega_G(x)(\chi) = \chi(x)$, for every $x \in G$ and for
every $\chi \in \hat{G}$. We show now that $\omega_G (x) \in \hat{\hat{G}}$.


\textbf{Proposition 7.21.} If $G$ is a topological abelian group. Then $\omega_G (x) \in \hat{\hat{G}}$ and $\omega_G : G \to \hat{\hat{G}}$ is a homomorphism.
If $G$ is locally compact, then the homomorphism $\omega_G$ is a continuous.


\emph{Proof.} In fact, 

    
    $\omega_G(x)(\chi + \psi) = (\chi + \psi)(x) = \chi (x) + \psi (x) = \omega_G(x)(\chi) + \omega_G(x)(\psi)$,


for every $\chi, \psi \in \hat{G}$. Moreover, if $U$ is an open neighborhood of 0 in $\mathbb{T}$, then $\omega_G(x)(W(\{x\}, U)) \subseteq U$. This
proves that $\omega_G(x)$ is a character of $\hat{G}$, i.e., $\omega_G(x) \in \hat{\hat{G}}$. For every $x, y \in G$ and for every $\chi \in \hat{G}$ we have
$\omega_G(x + y)(\chi) = (\chi)(x + y) = \chi(x) + \chi(y) = \omega_G(\chi)(x) + \omega_G(\chi)(y)$ and so $\omega_G$ is a homomorphism.


    Now assume $G$ is locally compact. To prove that $\omega_G$ is continuous, pick an open neighborhood $A$ of 0 in $\mathbb{T}$ and
a compact subset $K$ of $\hat{G}$. Then $W(K, A)$ is an open neighborhood of 0 in $\hat{\hat{G}}$. Let $U$ be an open neighborhood of
0 in $G$ with compact closure. Take an open symmetric neighborhood $B$ of 0 in $\mathbb{T}$ with $B+B \subseteq A$. Thus $W(\bar{U}, B)$
is an open neighborhood of 0 in $\hat{G}$. Since $K$ is compact, there exist finitely many characters $\chi_1, . . . , \chi_m$ of $G$
such that $K \subseteq (\chi_1 + W(\bar{U}, B)) \cup · · · \cup (\chi_m + W(\bar{U}, B))$. For every $i = 1, . . . , m$ there is an open neighborhood
$V_i$ of 0 in $G$ such that $\chi_i(V_i) \subseteq B$. Define $V = U \cap V_1 \cap · · · \cap V_m \subesteq U$ and note that $\chi_i(V ) \subseteq B$ for every
$i = 1, . . . , m$. Thus $\omega_G(V ) \subseteq W(K, A)$. Indeed, if $x \in V$ and $\chi \in K$, then $\chi_i(x) \subseteq B$ for every $i = 1, . . . , m$ and
there exists $i_0 \in {1, . . . , m}$ such that $\chi \in \chi_{i_0} + W(\bar{U}, B)$; so $\chi(x) = \chi_{i_0} (x) + \psi (x)$ with $\psi \in W(\bar{U}, B)$ and then
$\omega_G(x)(\chi) = \in(x) \in B + B \subseteq A$.


    In this chapter we shall have a precise approach, by saying that a group $G$ satisfies the Pontryagin-van
Kampen duality theorem when ωG is a topological isomorphism.


\textbf{Lemma 7.22.} If the topological abelian groups $G_i$ satisfy Pontryagin-van Kampen duality theorem for
$i = 1, 2, . . . , n$, then also $G =\Pi^{n}_{i=1} G_i$ satisfies Pontryagin-van Kampen duality theorem.


    Proof. Apply Lemma 7.12 twice to obtain an isomorphism $j : \Pi_n_{i=1} \hat{\hat{G}}_i \to \hat{\hat{G}}$. It remains to verify that the
product $\pi : G \to \Pi^n_{i=1} \hat{\hat{G}}_i$ of the isomorphisms $\omega_{G_i}: G_i \to \Pi^n_{i=1} \hat{\hat{G}}_i$ given by our hypothesis composed with the
isomorphism $j$ gives precisely $\omega_G$.


    Consider two categories $\mathcal{A}$ and $\mathcal{B}$. A covariant [contravariant] functor $F : \mathcal{A} \to \mathcal{B}$ assigns to each object
$A \in \mathcal{A}$ an object $F A \in \mathcal{B}$ and to each arrow $f : A \to A'$ in $\mathcal{A}$ an arrow $F f : F A \to F A'[F f : F A' \to F A]$ such
that $F id_A = id_{FA}$ and $F(g \circ f) = F g \circ F f [F(g \circ f) = F f \circ F g]$ for every arrow $f : A \to A'$ and $g : A' → A''$
in $\mathcal{A}$.


Let $F, F': A \to B$ be covariant functors. A natural transformation $\gamma$ from $F$ to $F'$ assigns to each $A \in \mathcal{A}$
an arrow $\gamma A : F A \to F'A$ such that for every arrow $f : A \to A'$ in $\mathcal{A}$ the following diagram is commutative


%The following diagram is in the shape of a square, but like the last one I will have to describe it%

$FA \to^{Ff} FA'$ %Is along the top%

$F'A \to^{F'f} F'A'$ %Is along the bottom%

%Down the sides are the following arrows%

$FA \to^{\gamma_{A}} F'A$

$FA' \to^{\gamma_{A'}} F'A'$


    A \emph{natural equivalence} is a natural transformation $\gamma$ such that each $\gamma_A$ is an isomorphism.


    If $\mathcal{H}$ denote the category of all Hausdorff abelian topological groups, the \emph{Pontryagin-van Kampen duality
functor}, defined by


    $G \mapsto \hat{G}$ and $f \mapsto \hat{f}$


for objects $G$ and morphisms $f$ of $\mathcal{H}$, is a contravariant functor $\hat{}: \mathcal{H} \to \mathcal{H}$. Let $\mathcal{L}$ be the full subcategory of $\mathcal{H}$
having as objects all locally compact abelian groups. According to Proposition 7.2, the functor $\hat{}$ sends $\mathcal{L}$ to
itself, i.e., defines a functor $\hat{}: \mathcal{L} \to \mathcal{L}$. The Pontryagin-van Kampen duality theorem states that $\omega$ is a natural
equivalence from id_$\mathcal{L}$ to $\hat{\hat{}}: \mathcal{L} \to \mathcal{L}$. We start by proving that $\omega$ is a natural transformation.


\textbf{Proposition 7.23.} $\omega$ is a natural transformation from id_${\mathcal{L}}$ to $\hat{\hat{}}: \mathcal{L} \to \mathcal{L}$


\emph{Proof.} By Proposition 7.21 $\omega_G$ is continuous for every $G \in \mathcal{L}$. Moreover for every continuous homomorphism
$f : G \to H$ the following diagram commutes:


%The following diagram is also in the shape of a square, so I will describe it%

$G \to^{f} H$ %Is along the top%

$\hat{\hat{G}} \to^{\hat{\hat{f}}} \hat{\hat{H}}$ %Is along the bottom%

%Down the sides are the following arrows%

$G \to^{\omega_G} \hat{\hat{G}}$ 

$H \to^{\omega_H} \hat{\hat{H}}$ 


In fact, if $x \in G$ and $\xi \in \hat{H}$, then $\omega_H(f(x))(\xi) = \xi(f(x))$. On the other hand,


    $(\hat{\hat{f}}(\omega_G(x)))(\xi) = (\omega_G(x) \circ \hat{f})(ξ) = ω_G(x)(\hat{f}(\xi)) = \omega_G(x)(\xi \circ f) = \xi(f(x))$.


Hence $\omega_H(f(x)) = \hat{\hat{f}}(\omega_G(x))$ for every $x \in G$.


\textbf{Remark 7.24}  Note that $\omega_G$ is a monomorphism if and only if $\hat{G}$ separates the points of $G$. Moreover, $\omega_G(G)$
is a subgroup of $\hat{\hat{G}}$ that separates the points of $\hat{G}$.


    Now we can prove the Pontryagin-van Kampen duality theorem in the case when $G$ is either compact or
discrete.


\textbf{Theorem 7.25.} If the abelian topological group $G$ is either compact or discrete, then $\omega_G$ is a topological
isomorphism.


\emph{Proof.} If $G$ is discrete, then $\hat{G}$ separates the points of $G$ by Corollary 2.7 and if $G$ is compact, then $\hat{G}$ separates
the points of $G$ by the Peter-Weyl Theorem 6.4. Therefore $\omega_G$ is injective by Remark 7.24. If $G$ is discrete,
then $\hat{G}$ is compact and $\omega_G(G) = \hat{\hat{G}}$ by Corollary 6.6. Since $\hat{\hat{G}}$ is discrete, $\omega_G$ is a topological isomorphism.


    Let now $G$ be compact. Then $\omega_G$ is open thanks to Theorem 4.9. Suppose that $\omega_G(G)$ is a proper subgroup
of $\hat{\hat{G}}?$. By the compactness of $G, \hat{\hat{G}}$ is compact, hence closed in $\hat{\hat{G}}$. By the Peter-Weyl Theorem 6.4 applied
to $\hat{\hat{G}}/\omega_G(G)$, there exists $\xi \in \hat{\hat{\hat{G}}} / \{0\}$ such that $\xi(\omega_G(G)) = \{0\}$. Since $\hat{G}$ is discrete, $\omega_{\hat{G}}$ is a topological
isomorphism and so there exists $\chi \in \hat{G}$ such that $\omega_{\hat{G}}(\chi) = \xi$. Thus for every $x \in G$ we have
$0 = \xi(\omega_G(x)) = \omega_{\hat{G}}(\chi)(\omega_G(x)) = \omega_G(x)(\chi) = \chi(x)$. It follows that $\chi ≡ 0$ and so that also $\xi ≡ 0$, a contradiction.


    Our next step is to prove the Pontryagin-van Kampen duality theorem when $G$ is elementary locally compact
abelian:


\textbf{Theorem 2.76.} If $G$ is an elementary locally compact abelian group, then $\omega_G$ is a topological isomorphism of
$G$ onto $\hat{\hat{G}}$.


\emph{Proof.} According to Lemma 7.22 and Theorem 7.25 it suffices to prove that $\omega_{\mathbb{R}}$ is a topologically isomorphism.
Of course, by the fact that $\hat{\mathbb{R}}$ is topologically isomorphic to $\mathbb{R}$, one concludes immediately that also $\mathbb{R}$ and $\hat{\hat{\mathbb{R}}}$
are topologically isomorphism. A more careful analysis of the dual $\hat{\mathbb{R}}$ shows the crucial role of the ($\mathbb{Z}$-)bilnear
map $\lambda : \mathbb{R} \times \mathbb{R} \to \mathbb{T}$ defined by $λ(x, y) = \chi__1(xy)$, where $\chi_1 : \mathbb{R} \to \mathbb{T}$ is the character determined by the canonical
quotient map $\mathbb{R} \to \mathbb{T} = \mathbb{R}/\mathbb{Z}$. Indeed, for every $y \in \mathbb{R}$ the map $\chi_y : \mathbb{R} \to \mathbb{T}$ defined by $x \mapsto \lambda(x, y)$ is an element
of $\hat{\mathbb{R}}$. Hence the second copy $\{0\} \times \mathbb{R}$ of $\mathbb{R}$ in $\mathbb{R} \times \mathbb{R}$ can be identified with $\hat{\mathbb{R}}$. On the other hand, every element
$x \in \mathbb{R}$ gives a continuous character $\mathbb{R} \to \mathbb{T}$ defined by $y \mapsto \lambda(x, y)$, so can be considered as the element $\omega_R(x)$ of
$\hat{\hat{R}}$. We have seen that every $\xi \in \hat{\hat{R}}$ has this form. This means that $\omega_{\mathbb{R}}$ is surjective. Since continuity of $\omega_{\mathbb{R}}$, as
well as local compactness of $\hat{\hat{\mathbb{R}}}$ are already established, $\omega_{\mathbb{R}}$ is a topological isomorphism by the open mapping
theorem.


    For a subset $X$ of $G$ the annihilator of $X$ in $\hat{G}$ is $A_{\hat{G}}(X) = \{\chi \in \hat{G} : \chi(A) = {0}\}$ and for a subset $Y$ of $\hat{G}$
the annihilator of $Y$ in $G$ is $A_G(Y ) = {x \in G : \chi(x) = 0 \text{ for every } x \in Y }$. When no confusion is possible we
shall omit the subscripts $_{\hat{G}}$ and $_G$.


    The next lemma will help us in computing the dual of a subgroup and a quotient group.


\textbf{Lemma 7.27} Let $G$ be a locally compact abelian group. If $M$ is a subset of $G$, then $A_{\hat{G}} (M)$ is a closed subgroup
of $\hat{G}$.


    \wmph{Proof.} It suffices to note that


        $A_{\hat{G}}(M) = \bigcap_{x \in M} \{ \chi \in \hat{G} : \chi(x) \} = \bigcap\{ker \omega_G(x) : x \in M \}$


where each $ker \omega(x)$ is a closed subgroup of $\haat{G}$


Call a continuous homomorphism $f : G \to H$ of topological groups proper if $f : G \to f(G)$ is open, whenever
$f(G)$ carries the topology inherited from $H$. In particular, a surjective continuous homomorphism is proper iff
it is open.


    A short sequence $0 \to G_1 \to^f G \to^h G_2 \to 0$ in $\mathcal{L}$, where $f$ and $h$ are continuous homomorphisms, is exact if
$f$ is injective, $h$ is surjective and $im f = ker h$. It is proper if $f$ and $h$ are proper.


\textbf{Lemma 7.28.} Let $G$ be a locally compact abelian group, $H$ a subgroup of $G$ and $i : H \to G$ the canonical
inclusion of $H$ in $G$. Then


\begin{itemize}

    \item (a) $\hat{i} : \hat{G} \to \hat{H}$ is surjective if $H$ is dense or open or compact;

    \item (b) $\hat{i}$ is injective if and only if $H$ is dense in $G$;

    \item (c) if $H$ is closed and $\pi : G \to G/H$ is the canonical projection, then the sequence
    
            $0 \to \hat{G/H} \to^{\hat{\pi}} \hat{G} \to^{\hat{i}} \hat{H}$

        is exact, $\hat{\pi}$ is proper and $im \hat{\pi} = A_{\hat{G}}(H)$. If $H$ is open or compact, then $\hat{i}$ is open and surjective.
\end{itemize}


\emph{Proof.} (a) Note that $\hat{i}$ is surjective if and only if for every $\chi \in \hat{H}$ there exists $\xi \in \hat{G}$ such that $\xi \upharpoonright_H = \chi$. If $H$ is
compact apply Corollary 6.20. Otherwise Lemma 7.16 applies.


    (b) If $H$ is dense, then $\hat{i}$ is injective by Lemma 7.16. Conversely, assume that $\bar{H}$ is a proper subgroup of $G$
and let $q : G \to G/\bar{H}$ be the canonical projection. By Theorem 6.19 there exists $\chi \in \hat{G/\bar{H}}$ not identically zero.
Then $\xi = \chi \circ q \in \hat{G}$ is non-zero and satisfies $\xi(H) = \{0\}$, i.e., $\hat{i}(\xi) = 0$. This implies that $\hat{i}$ is not injective.


    (c) According to Lemma 7.16 $\hat{i}$ is a monomorphism, since $\pi$ is surjective. We have that $\hat{i} \circ \hat{\pi} = \hat{\pi \circ i} = 0$.
If $\xi \in ker \hat{i} = {\chi \in \hat{G} : \chi(H) = {0}}$, then $\xi(H) = \{0\}$. So there exists $\xi_1 \in \hat{G/H}$ such that $\xi = \xi_1 \circ \pi$ (i.e.
$\xi = \hat{\pi}(\xi_1))$ and we can conclude that $ker \hat{i} = im \hat{\pi}$. So the sequence is exact and $im \hat{\pi} = ker \hat{i} = A_{\hat{G}}(H)$.


    To show that πb is proper it suffices to apply Lemma 7.16.


    If $H$ is open or compact, (a) implies that $\hat{i}$ is surjective. It remains to show that $\hat{i}$ is open. If $H$ is compact
then $\hat{H}$ is discrete by Example 7.1(2), so $\hat{i}$ is obviously open. If $H$ is open, let $K$ be a compact neighborhood
of 0 in $G$ such that $K \subseteq H$. Then $W = W_{\hat{G}}(K,\bar{\wedge_4})$ is a compact neighborhood of 0 in Gb. Since bi is surjective,
V = bi(W) = WHb (K,Λ4) is a neighborhood of 0 in $\hat{H}$. Now $M = \langle W \rangle$ and $M_1 = \langle V \rangle$ are open compactly
generated subgroups respectively of $\hat{G}$ and $\hat{H}$, and $\hat{i}(M) = M_1$. Since $M$ is $\sigma$-compact by Lemma 4.12, we can
apply Theorem 4.9 to the continuous surjective homomorphism $\hat{i} \upharpoonright_M: M \to M_1$ and so also bi is open.

 
\textbf{Corollary 7.29.} Let $G$ be a locally compact abelian group and let $H$ be a closed subgroup of $G$. Then
$\hat{G/H} \cong A_{\hat{G}}(H)$. Moreover, if $H$ is open or compact, then $\hat{H} \cong \hat{G}/A_{\hat{G}}(H)$.


    The next corollary says that the duality functor preserves proper exactness for some sequences.


\textbf{Corollary 7.30.} If the sequence $0 \to G_1 \to^f G \to^h G_2 → 0$ in $\mathcal{L}$ is proper exact, with $G_1$ compact or $G_2$ discrete,
then $0 \to \hat{G_2} \to^{\hat{h}} \hat{G} \to^{\hat{f}} \hat{G_1} \to 0$ is proper exact with the same property.


    Now we can prove prove the Pontryagin-van Kampen duality theorem, namely $\omega$ is a natural equivalence
from id_$\mathcal{L}$ to $\hat{\hat{}}: \mathcal{L} \to \mathcal{L}$.



\textbf{Theorem 7.31.} If $G$ is a locally compact abelian group, then $\omega_G$ is a topological isomorphism of $G$ onto $\hat{\hat{G}}$.


\emph{Proof.} We know by Proposition 7.23 that $\omega$ is a natural transformation from $id_\mathcal{L}$ to $\hat{\hat{}}: \mathcal{L} \to \mathcal{L}$. Our plan is to
chase the given locally compact abelian group G into an appropriately chosen proper exact sequence


    $0 \to G_1 \to^t G \to^h G_2 \to 0$


in $\mathcal{L}$, with $G_1$ compact or $G_2$ discrete, such that $G_1$ and $G_2$ satisfy the duality theorem. By Corollary 7.30 the
sequences


    $0 \to \hat{G_2} \to^{\hat{h}} \to \hat{G} \to^{\hat{f}} \to \hat{G_2} \to 0$ and $0 \to \hat{\hat{G_1}} \to^{\hat{\hat{f}}} \hat{\hat{G}} \to^{\hat{\hat{h}}} \hat{\hat{G_2}} \to 0$


are proper exact. According to Proposition 7.23 the following diagram commutes:


% The following diagram consists of the first sequence above the third sequences from Theorem 7.31, and the G's have arrows from the top down

% The following sequence is along the top

$0 \to G_1 \to^t G \to^h G_2 \to 0$

% The following sequence is along the bottom

$0 \to \hat{\hat{G_1}} \to^{\hat{\hat{f}}} \hat{\hat{G}} \to^{\hat{\hat{h}}} \hat{\hat{G_2}} \to 0$

% These are the following arrows from the top to bottom, from left to right

$G_1 \to^{\omega_{G_1}} \hat{\hat{G_1}}$

$G \to^{\omega_G} \hat{\hat{G}}$

$G_2 \to^{\omega_{G_2}} \hat{\hat{G_2}}$


According to Theorem 6.19, $\omega_{G_1}$, $\omega_G$, $\omega_{G_2}$ are injective. Moreover, $\omega_{G_1}$ and $\omega_{G_2}$ are surjective by our
choice of $G_1$ and $G_2$. Then $\omega_G$ must be surjective too. (Indeed, if $x \in ker \hat{\hat{h}}$, then there exists $y \in \omega_G(G)$ with
$\hat{\hat{h}}(x) = \hat{\hat{h}}(y)$, because $\hat{\hat{h}}(\omega_G(G)) = \hat{\hat{G_2}}$. Now $y - x \in ker \hat{\hat{h}} \subseteq \omega_G(G)$ and so $x \in y + \omega_G(G) = \omega_G(G)$.)
If $G$ is locally compact abelian and compactly generated, by Proposition 6.18 we can choose $G_1$ compact
and $G_2$ elementary locally compact abelian. Then $G_1$ and $G_2$ satisfy the duality theorem by Theorems 7.25
and 7.26, hence $\omega_G$ is surjective. Since $\omega_G$ is a continuous isomorphism and $G$ is $\sigma$-compact, we conclude with
Theorem 4.9 that $\omega_G$ is a topological isomorphism.


    In the general case of locally compact abelian group $G$, we can take an open compactly generated subgroup
$G_1$ of $G$. This will produce a proper exact sequence $0 \to G_1 \to^f G \to^h G_2 \to 0$ with $G_1$ compactly generated and
$G_2 \cong G/G_1$ discrete. By the previous case $\omega_{G_1}$ is a topological isomorphism and $\omega_{G_2}$ is an isomorphism thanks
to Theorem 7.25. Therefore $\omega_G$ is a continuous isomorphism.


    Moreover $\omega_G \upharpoonright_{f(G_1)}: f(G_1) \to \hat{\hat{f}}(\hat{\hat{G_1}})$ is a topological isomorphism (as $\omega_{G_1}, f : G_1 \to f(G_1)$
and $\hat{\hat{f}} : \hat{\hat{G_1}} \to \hat{\hat{f}}(\hat{\hat{G_1}})$ are topological isomorphisms) and $f(G_1)$ and $\hat{\hat{f}}(\hat{\hat{G_1}})$ are open subgroups respectively of $G$ and $\hat{\hat{G}}$. Thus
$\omega_G$ is a topological isomorphism.


    Our last aim is to prove that the annihilators define an inclusion-inverting bijection between the family of
all closed subgroups of a locally compact group $G$ and the family of all closed subgroups of $\hat{G}$. We use that fact
that one can identify $G$ and $\hat{\hat{G}}$ by the topological isomorphism $\omega_G$. In more precise terms:


\textbf{Exercise 7.32.} Let $G$ be a locally compact abelian group and $Y$ be a subset of $\hat{G}$. Then $A_{\hat{\hat{G}}}(Y ) = \omega_G(A_G(Y ))$.


\textbf{Lemma 7.33.} Let G be a locally compact abelian group and $H$ a closed subgroup of $G$. If $a \in G \ H$ then there
exists $\chi \in A(H)$ such that $\chi(x) \neq 0$.


\emph{Proof.} Let $\rho : \hat{G/H} \to A(H)$ be the topological isomorphism of Corollary 7.29. By Theorem 6.19 there exists
$\psi \in \hat{G/H}$ such that $\psi(a + H) \neq 0$. Therefore $\chi = \rho(\psi) \in A(H)$ and $\chi(a) = \rho(\psi)(a) = \psi(a + H) \neq 0$.


\textbf{Corollary 7.34.} If $G$ is a locally compact abelian group and $H$ a closed subgroup of $G$, then


    $H = A_G (A_{\hat{G}}(H)) = \omega_G^{-1}(A_{\hat{\hat{G}}}(A_{\hat{G}}(H)))$


\emph{Proof.} The first equality follows immediately from the above lemma.
The last equality follows from the equality $H = A_G (A_{\hat{G}}(H))$ and Exercise 7.32.


    By Lemma 7.29 the equality $H = A_G (A_{\hat{G}}(H))$ holds if and only if $H$ is a closed subgroup of $G$.


\textbf{Proposition 7.35.} Let $G$ be a locally compact abelian group and $H$ a closed subgroup of $G$. Then $\hat{H} \cong \hat{G} / A(H)$


\eemph{Proof.} Since $H = \omega_G^{-1}(A_{\hat{\hat{G}}}(A_{\hat{G}}(H)))$  by Lemma 7.34 we have a topological isomorphism $\phi$ from $H$ to $\hat{\hat{G} / A(H)}$
given by $\phi(h)(\alpha + A(H))=\alpha(h)$ for every $h \in H$ and $\alpha \in \hat{G}$. This gives rise to another topological isomorphism
$\hat{\phi} : \hat{\hat{\hat{G} / A(H)}} \to \hat{H}$.  By Pontryagin's duality theorem 7.31 $\omega_{\hat{G}/A(H)}$ is a topological isomorphism from $\hat{G}/A(H)$
to $\hat{\hat{\hat{G} / A(H)}}$. The composition gives the desired isomorphism.


Finally, let us resume for reader's benefit some of the most relevant points of Pontryagin-van Kampen duality
theorem established so far:


\textbf{Theorem 7.36.} Let $G$ be a locally compact abelian group. Then $\hat{G}$ is a locally compact abelian group and:


\begin{itemize}

    \item (a) the correspondence $H \mapsto A_{\hat{G}}(H)$, $N \mapsto A_G(N)$, where $H$ is a closed subgroup of $G$ and $N$ is a closed
    subgroup of $\hat{G}$, defines an order-inverting bijection between the family of all closed subgroups of $G$ and the
    family of all closed subgroups of $\hat{G}$;

    \item (b) for every closed subgroup $H$ of $G$ the dual group $\hat{H}$ is isomorphic to $\hat{G}/A (H)$, while $A(H)$ is isomorphic
    to the dual $\hat{G/H}$;

    \item (c) $\omega_G : G \to \hat{\hat{G}}$ is a topological isomorphism;

    \item (d) $G$ is compact (resp., discrete) if and only if $\hat{G}$ is discrete (resp., compact);

\end{itemize}


\emph{Proof} The first sentence is proved in Theorem 7.2. (a) is Corollary 7.34 while (b) is Proposition 7.35. (c) is
Theorem 7.31. To prove (d) apply Theorem 7.31 and Lemma 7.1.


    Using the full power of the duality theorem one can prove the following structure theorem on compactly
generated locally compact abelian groups.


\textbf{Theorem 7.37.} Let G be a locally compact compactly generated abelian group. Prove that $G \cong \mathbb{R}^n \times \mathbb{Z}^m \times K$,
where $n, m \in N$ and $K$ is a compact abelian group.


\emph{Proof.} According to Theorem 6.18 there exists a compact subgroup $K$ of $G$ such that $G/K$ is an elementary
locally compact abelian group. Taking a bigger compact subgroup one can get the quotient $G/K$ to be of the
form $\mathbb{R}^n \times \mathbb{Z}^m$ for some $n, m \in \mathbb{N}$. Now the dual group $\hat{G}$ has an open subgroup $A(K) \cong \hat{G/K} \cong \mathbb{R}^n \times \mathbb{T}^m$.
Since this subgroup is divisible, one has $\hat{G} \cong \mathbb{R}^n \times \mathbb{T}^m \times D$, where $D \cong \hat{G}/A (K)$ is discrete and $D \cong \hat{K}$. Taking
duals gives $G \cong \hat{\hat{G}} \cong \mathbb{R}^n \times \mathbb{Z}^m \times K$.


    Making sharp use of the annihilators one can prove the structure theorem on locally compact abelian groups
(see [67, 36] for a proof).


\textbf{Theorem 7.38.} Let $G$ be a locally compact abelian group. Then $G \cong \textbf{R}^n \times G_0$, where $G_0$ is a closed subgroup
of $G$ containing an open compact subgroup $K$.


    As a corollary one can prove:


\textbf{Corollary 7.39.} Every locally compact abelian group is isomorphic to a subgroup of a group of the form
$\mathbb{R}^n \times D \times K$, where $n \in \mathbb{N}, D$ is a discrete abelian group and $K$ is a compact abelian group.


\textbf{Exercise 7.40} Let $G$ be a locally compact abelian group. Prove that for $\chi_1, . . . , \chi_n \in \hat{G}$ and $\delta > 0$ one has


    $U_G (\chi_1, . . . , \chi_n; \delta) = \omega^{-1}_G (W_{\hat{\hat{G}}} ({\chi_1, . . . , \chi_n}, U)),$,


where $U$ is the neighborhood of 0 in $\mathbb{T} \cong \mathbb{S}$ determined by $|Argz| < \delta$.


\textbf{Exercise 7.41.} Let $G$ be a compact connected abelian group. Prove that $t(G)$ is dense in $G$ iff $\hat{G}$ is reduced.
Deduce that every compact connected abelian group $G$ has the form $G \cong G_1 \times \mathbb{Q}^\alpha$ for some cardinal $\alpha$, where
the compact subgroup $G_1$ coincides with the closure of the subgroup $t(G)$ of $G$.


(Hint. Note first that $\hat{G}$ is torsion-free. Deduce that $\hat{G}$ is reduced iff $\bigcap^{\infty}_{n=1} n \hat{G} = 0$. Show that this equality
is equivalent to density of $t(G) = \bigcup^{\infty}_{n=1} G[n]$ in $G$. To prove the second assertion consider the torsion-free dual
$\hat{G}$ and its decomposition $\hat{G} = d(\hat{G}) \times R$, where R is a reduced subgroup of $\hat{G}$. Now apply the first part and the
isomorphism $G \cong \hat{\hat{G}}$.)


\textbf{Exercise 7.42.} . Give example of a reduced abelian group $G$ such that $\bigcap^{\infty}_{n=1} n \hat{G} \neq 0$.
(Hint. Fix a prime number p and take an appropriate quotient of the group $\bigoplus^{\infty}_{n=1} \mathbb{Z}(p^n)$).

\end{document}
