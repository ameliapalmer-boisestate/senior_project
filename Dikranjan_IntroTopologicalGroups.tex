\documentclass[12pt]{article}
\usepackage{amsmath}
\usepackage{graphicx}
\usepackage{hyperref}
\usepackage[utf8]{inputenc}
\usepackage{mathrsfs}
\usepackage{amssymb}

\title{Introduction to Topological Groups}
\author{Dikran Dikranjan}

\begin{document}

\maketitle

\begin{abstract}
        These notes provide a brief introduction to topological groups with a special emphasis on Pontryagin-
    van Kampen's duality theorem for locally compact abelian groups. We give a completely self-contained 
    elementary proof of the theorem following the line from [36]. According to the classical tradition, the
    structure theory of the locally compact abelian groups is built parallelly.
\end{abstract}

\section{Introduction}


        Let $L$ denote the category of locally compact abelian groups and continuous homomorphisms and let $ \mathbb{T} = \mathbb{R} / \mathbb{Z} $
    be the unit circle group. For $G \in L$ denote by $ \widehat{G} $ the group of continuous homomorphisms (characters) $ G \to \mathbb{T} $
    equipped with the compact-open topology\footnote[1]{having as a base of the neighborhoods of 0 the family of sets $ W(K, U) = {\chi \in \widehat{G} : x(K) \subseteq U} $, where $ K \subseteq G $ is compact and
    $U$ is an open neighborhood of 0 in $T$.}. Then the assignment
    

    $G \mapsto \widehat{G}$


    is a contravariant endofunctor $ \widehat{}: L \to L $. The celebrated Pontryagin-van Kampen duality theorem ([82]) says
    that this functor is, up to natural equivalence, an involution i.e., $ \widehat{\widehat{G}} \cong G $ (see Theorem 7.36 for more detail).
    Moreover, this functor sends compact groups to discrete ones and viceversa, i.e., it defines a duality between
    the subcategory $ C $ of compact abelian groups and the subcategory $ D $ of discrete abelian groups. This allows for
    a very efficient and fruitful tool for the study of compact abelian groups, reducing all problems to the related
    problems in the category of discrete groups. The reader is advised to give a look at the Mackey's beautiful
    survey [75] for the connection of charactres and Pontryagin-van Kampen duality to number theory, physics and
    elsewhere. This duality inspired a huge amount of related research also in category theory, a brief comment on
    a specific categorical aspect (uniqueness and representability) can be found in $ \S $8.1 of the Appendix.
    

        The aim of these notes is to provide a self-contained proof of this remarkable duality theorem, providing all
    necessary steps, including basic background on topological groups and the structure theory of locally compact
    abelian groups. Peter-Weyl's theorem asserting that the continuous characters of the compact abelian groups
    separate the points of the groups (see Theorem 6.4) is certainly the most important tool in proving the duality
    theorem. The usual proof of Peter-Weyl's theorem involves Haar integration in order to produce sufficiently
    many finite-dimensional unitary representations. In the case of abelian groups the irreducible ones turn out the
    be one-dimensional. i.e., charactres. We prefer here a different approach. Namely, Peter-Weyl's theorem can be
    obtained as an immediate corollary of a theorem of Folner (Theorem 5.12) whose elementary proof uses nothing
    beyond elementary properties of the finite abelian groups, a local version of the Stone-Weierstraß approximation
    theorem proved in $ \S $2 and the Stone-Cech compactification of discrete spaces. As another application of Folner's ˇ
    theorem we describe the precompact groups (i.e., the subgroups of the compact groups) as having a topology
    generated by continuous characters. As a third application of Folner's theorem one can obtain the existence of
    the Haar integral on locally compact abelian groups for free (see [36, $ \S $2.4, Theorem 2.4.5]).
        

        The notes are organized as follows. In Section 2 we recall basic results and notions on abelian groups and
    general topology, which will be used in the rest of the paper. Section 3 contains background on topological
    groups, starting from scratch. Various ways of introducing a group topology are considered ($ \S $3.2), of which the
    prominent one is by means of characters ($ \S $3.2.3). In $ \S $3.6 we recall the construction of Protasov and Zelenyuk
    [88] of topologies arising from a given sequence that is required to be convergent to 0. Connectedness and
    related properties in topological groups are discussed in $ \S $3.5. In $ \S $3.7 the Markov's problems on the existence
    of non-discrete Hausdorff group topologies is discussed. In $ \S $3.7.1 we introduce two topologies, the Markov
    topology and the Zariski topology, that allow for an easier understanding of Markov's problems. In $ \S $3.7.2 we
    describe the Markov topology of the infinite permutation groups, while $ \S $7.3.3 contains the first two examples
    of non-topologizable groups, given by Shelah and Ol'shanskii, respectively. The problems arising in extension
    of group topologies are the topic of $ \S $7.3.4. Several cardinal invariants (weight, character and density character)
    are introduced in $ \S $3.8, whereas $ \S $3.9 discuses completeness and completions. Further general information on
    topological groups can be found in the monographs or surveys [3, 26, 27, 28, 36, 70, 79, 82].
    

        Section 4 is dedicated to specific properties of the (locally) compact groups used essentially in these notes.
    The most important property we recall in $ \S $4.1 is the open mapping theorem In $ \S $4.2 we give an internal
    description of the precompact groups using the notion of a big (large) set of the group. In $ \S $4.3 we recall
    (with complete proofs) the structure of the closed subgroups of $ \mathbb{R}^n $ as well as the description of the closure of an
    arbitrary subgroup of $ \mathbb{R}^n $. These groups play an important role in the whole theory of locally compact abelian
    groups. In $ \S $5 we expose the proof of Folner's theorem (see Theorem 5.12). The proof, follows the line of [36].
    An important ingredient of the proof is also the crucial idea, due to Prodanov, that consists in elimination of
    all discontinuous characters in the uniform approximation of continuous functions via linear combinations of
    characters obtained by means of Stone-Weierstra$\beta$ approximation theorem (Prodanov's lemma 5.10).


        In Section 6 we give various applications of Folner's theorem. The main one is an immediate proof of Peter-
    Weyl's theorem. In this chapter we give also several other applications of Folner's theorem and Peter-Weyl's
    theorem: a description of the precompact group topologies of the abelian groups ($ \S $6.1) and the structure of
    the compactly generated locally compact abelian groups ($ \S $6.3). Here we consider also a precompact version of
    Protasov and Zelenyuk's construction [88] of topologies making a fixed sequence converge to 0.


        Section 7 is dedicated to Pontryagin-van Kampen duality. In $ \S\S $7.1-7.3 we construct all tools for proving the
    duality theorem 7.36. More specifically, $ \S\S $7.1 and 7.2 contain various properties of the dual groups that allow
    for an easier computation of the dual in many cases. Using further the properties of the dual, we see in $ \S $7.3 that
    many specific groups G satisfy the duality theorem, i.e., $ G \cong \widehat{\widehat{G}} $. In $ \S $7.4 we stress the fact that the isomorphism
    $ G \cong \widehat{\widehat{G}} $ is natural by studying in detail the natural transformation $ \omega G : G \to \widehat{\widehat{G}} $ connecting the group with
    its bidual. It is shown in several steps that $ \omega G $ is an isomorphism, considering larger and larger classes of
    locally compact abelian groups $ G $ where the duality theorem holds (elementary locally compact abelian groups,
    compact abelian groups, discrete abelian groups, compactly generated locally compact abelian groups). The
    last step uses the fact that the duality functor is exact, this permits us to use all previous steps in the general
    case.
    

        In the Appendix we dedicate some time to several topics that are not discussed in the main body of the
    notes: uniqueness of the duality, dualities for non-abelian or non-locally compact-groups, some connection to
    the topological properties of compact group and dynamical systems.
    

        A large number of exercises is given in the text to ease the understanding of the basic properties of group
    topologies and the various aspects of the duality theorem.
       

        These notes are born out of two courses in the framework of the PhD programs at the Department of
    Mathematics at Milan University and the Department of Geometry and Topology at the Complutense University
    of Madrid held in April/May 2007. Among the participants there were various groups, interested in different
    fields. To partially satisfy the interest of the audience I included various parts that can be eventually skipped,
    at least during the first reading. For example, the reader who is not interested in non-abelian groups can skip
    $ \S\S $3.2.4, the entire $ \S $3.7 and take all groups abelian in $ \S\S $3 and 4 (conversely, the reader interested in non-abelian
    groups or rings may dedicate more time to $ \S\S $3.2.4, 3.7 and consider the non-abelian case also in the first half of
    $ \S $4.2, see the footnote at the beginning of $ \S $4.2). For the category theorists $ \S\S $4.3, 5.1-5.3, 6.2-6.3 may have less
    interest, compared to $ \S\S $3.1-3.9, 4.2, 6.1, 7.1-7.4 and 8.1-8.2. Finally, those interested to get as fast as possible
    to the proof of the duality theorem can skip $ \S\S $3.2.3, 3.2.4 and 3.6-3.9 (in particular, the route $ \S\S $5-7 is possible
    for the reader with sufficient knowledge of topological groups).
    

        Several favorable circumstances helped in creating these notes. My sincere thanks go to my colleagues V.
    Zambelli, E. Mart \'in-Peinador, S. Kazangian, M. J. Chasco, M. G. Bianchi, L. Au$ \beta $enhofer, X. Dom \'ingues,
    M. Bruguera, S. Trevijano, and E. Pacifici who made this course possible. The younger participants of the
    course motivated me with their constant activity and challenging questions. I thank them for their interst
    and patience. I thank also my PhD student at Udine University Anna Giordano Bruno who prepared a very
    preliminary version of these notes in 2005.
    

        This notes are dedicated to the memory of Ivan Prodanov whose original contributions to Pontryagin-van
    Kampen duality can hardly by overestimated. The line adopted here follows his approach from [84] and [36].


    \tableofcontents


    \subsection{Notation and Terminology}
            We denote by $\mathbb{P}$, $\mathbb{N}$ and $\mathbb{N}_+$ respectively the set of primes, the set of natural numbers and the set of positive
        integers. The symbol $c$ stands for the cardinality of the continuum. The symbols $\mathbb{Z}$, $\mathbb{Q}$, $\mathbb{R}$, $\mathbb{C}$ will denote the
        integers, the rationals, the reals and the complex numbers, respectively.
        

            The quotient $\mathbb{T} = \mathbb{R}/\mathbb{Z}$ is a compact divisible abelian group, topologically isomorphic to the unitary circle $\mathbb{S}$
        (i.e., the subgroup of all $z \in \mathbb{C}$ with $|z| = 1$). For $\mathbb{S}$ we use the multiplicative notation, while for $\mathbb{T}$ we use the
        additive notation.
        

            For an abelian group G we denote by Hom $(G,\mathbb{T})$ the group of all homomorphisms from $G$ to $\mathbb{T}$ written
        additively. The multiplicative form $G^* = \text{Hom} (G, \mathbb{S}) \cong \text{Hom} (G,\mathbb{T})$ will be used when necessary (e.g., concerning
        easier computation in $\mathbb{C}$, etc.). We call the elements of Hom $(G,\mathbb{T}) \cong \text{Hom} (G, \mathbb{S})$ \emph{characters}.
        

            For a topological group $G$ we denote by $c(G)$ the connected component of the identity 1 in $G$. If $c(G)$ is
        trivial, the group $G$ is said to be totally disconnected. If $M$ is a subset of $G$ then $\langle M \rangle$ is the smallest subgroup
        of $G$ containing $M$ and $\bar{M}$ is the closure of $M$ in $G$. The symbol $w(G)$ stands for the weight of $\tilde{G}$. Moreover G
        stands for the completion of a Hausdorff topological abelian group $G$ (see $\S$3.9).


\section{Background on topological spaces and abstract groups}


    \subsection{Background and abelian groups}


    Generally a group $G$ will be written multiplicatively and the neutral element will be denoted by $e_G$ or simply $e$
    or 1 when there is no danger of confusion. For a subset $A, A_1, A_2, \dotsc , A_n$ of a group $G$ we write


    $A^{-1} = {a^{-1} : a \in A}, and A_1 A_2 \dotsc A_n = {a_1 . . . a_n : a_i \in A_i, i = 1, 2, \dotsc , n}$    $ (*) $


    and we write $A_n$ for $A_1 A_2 \dotsc A_n$ if all $A_i = A$. Moreover, for$ A \subseteq G $we denote by $c_{G}(A)$ the centralizer of $A$,
    i.e., the subgroup ${x \in G : xa = ax for every a \in A}$.
    
    
        We use additive notation for abelian groups, consequently 0 will denote the neutral element in such a case.
    Clearly, the counterpart of $(*)$ will be $-A, A_1 + A_2 + \dotsc + A_n$ and $nA$.
    
    
        A standard reference for abelian groups is the monograph [46]. We give here only those facts or definitions
    that appear very frequently in the sequel.
    
    
        For $m \in N_+$, we use $\mathbb{Z}m$ or $\mathbb{Z}(m)$ for the finite cyclic group of order $m$. Let $G$ be an abelian group. The
    subgroup of torsion elements of $G$ is $t(G)$ and for $m \in N_+$
    
    
    $G[m] = {x \in G : mx = 0}$ and $mG = {mx : x \in G}$.
    
    
        For a family $ {G_i : i \in I} $ of groups we denote by $ \Pi_{i \in I} G_i $ the direct product $ G $ of the groups $ G_i $. The underlying
    set of $ G $ is the Cartesian product $ \Pi_{i \in I} G_i $ and the operation is defined coordinatewise. The direct sum $ \bigoplus_{i \in I} G_i $
    is the subgroup of $\Pi_{i \in I} G_i$ consisting of all elements of finite support. If all $ G_i $ are isomorphic to the same
    group $ G $ and $ |I| = \alpha $, we write $ \bigoplus_{\alpha} G $ (or $ G^{(\alpha)} $, or $ \bigoplus_I G $) for the direct sum $ \bigoplus_{i \in I} G_i $
    
    
        A subset $X$ of an abelian group $G$ is independent, if $ \sum_{i=1}^{n} k_i x_i = 0 $ with $ k_i \in \mathbb{Z} $ and distinct elements $ x_i $ of
    $ X, i = 1, 2, . . . , n $, imply $ k_1 = k_2 = . . . = k_n = 0 $. The maximum size of an independent subset of $ G $ is called
    \emph{free-rank} of $ G $ and denoted by $r_{0} (G) $. An abelian group $G$ is \emph{free} , if $ G $ has an independent set of generators $ X $.
    In such a case $ G \cong \bigoplus_{|X|} \mathbb{Z} $.


        For an abelian group $ G $ and a prime number p the subgroup $ G_[p] $ is a vector space over the finite field $ \mathbb{Z} /p \mathbb{Z} $.
    We denote by $r_{p} (G)$ its dimension over $ \mathbb{Z} /p \mathbb{Z} $ and call it \emph{p-rank} of $G$.
    
    
        Let us start with the structure theorem for finitely generated abelian groups.


    \textbf{Theorem 2.1.} \emph{If $ G $ is a finitely generated abelian group, then $ G $ is a finite direct product of cyclic groups.
    Moreover, if $ G $ has $ m $ generators, then every subgroup of $ G $ is finitely generated as well and has at most $ m $
    generators.}


    \textbf{Definition 2.2.} \emph{An abelian group $G$ is}

        \begin{itemize}

            \item (a) \emph{torsion if $t(G) = G$;}
        
            \item (b) \emph{torsion-free if $t(G) = 0$;}
        
            \item (c) \emph{bounded if $mG = 0$ for some $m > 0$;}
        
            \item (d) \emph{divisible if $G = mG$ for every $m > 0$;}
            
            \item (e) \emph{reduced if the only divisible subgroup of $G$ is the trivial one.}
        
        \end{itemize}


    \textbf{Example 2.3.}
    

        \begin{itemize}

            \item (a) \emph{The groups $ \mathbb{Z} $, $ \mathbb{Q} $, $ \mathbb{R} $, and $ \mathbb{C} $ are torsion-free. The class of torsion-free groups is stable
            under taking direct products and subgroups.}

            \item (b) \emph{The groups $Z_m \mathbb{Q} / \mathbb{Z}$ are torsion. The class of torsion groups is stable under taking direct sums, subgroups
            and quotients.}

            \item (c) \emph{ Let $m_1, m_2, \dotsb , m_k > 1$ be naturals and let $\alpha_1, \alpha_2, \dotsb , \alpha_k$ be cardinal numbers. Then the group $ \bigoplus^{k}_{i=1} \mathbb{Z}^{(\alpha_{i})}_{m_{i}} $
            is bounded. According to a theorem of Prufer every bounded abelian group has this form [46]. This generalizes the Frobenius-Stickelberger theorem about the structure of the finite abelian groups (see Theorem
            2.1).}

        \end{itemize}

    
    \textbf{Example 2.4.}
    

        \begin{itemize}

            \item \emph{The groups $ \mathbb{Q} $, $ \mathbb{R} $, $ \mathbb{C} $, and $ \mathbb{T} $ are divisible.}

            \item \emph{For $ p \in \mathbb{P} $ we denote be $ \mathbb{Z} (p^{\infty}) $ the Prufer group, namely the $p$-primary component of the torsion group
            $ \mathbb{Q} / \mathbb{Z} $ (so that $ \mathbb{Z} (p^{\infty})$ has generators $ c_n = 1/p^{n} + \mathbb{Z}, n \in \mathbb{N} $). The group $ \mathbb{Z}(p^{\infty}) $ is divisible.}

            \item \emph{The class of divisible groups is stable under taking direct products, direct sums and quotients.
            In particular, every abelian group has a maximal divisible subgroup $d(G)$.}

            \item \emph{[46] Every divisible group $ G $ has the form $(\bigoplus_{ r_{0} (G) } Q) \oplus (\bigoplus_{p \in P} \mathbb{Z} (p^{\infty})^ {(rp(G))} )$ .}

        \end{itemize}


        If $X$ is a set, a set $Y$ of functions of $X$ to a set $Z$ \emph{separates the points} of $X$ if for every $x, y \in X$ with $x \neq y$,
    there exists $f \in Y$ such that $f(x) \neq f(y)$. Now we see that the characters separate the points of a discrete
    abelian groups.


    \textbf{Theorem 2.5.} \emph{Let $ G $ be an abelian group, $ H $ a subgroup of $ G $ and $ D $ a divisible abelian group. Then for every
    homomorphism $ f : H \to D $ there exists a homomorphism $ \bar{f} : G \to D $ such that $ \bar{f} \upharpoonright_{H} = f $.}
    
    
        \emph{If $ a \in G \ H $ and $ D $ contains elements of arbitrary finite order, then $ \bar{f} $ can be chosen such that $ \bar{f} (a) \neq 0 $.}
    
    
        \emph{Proof}. Let $ H' $ be a subgroup of $ G $ such that $H' \supseteq H $ and suppose that $ g : H' \to D $ is such that $ g \upharpoonright_{H} = f $. We
    prove that for every $ x \in G $, defining $ N = H' + \langle x \rangle $, there exists $ \bar{g} : N \to D $ such that  $ \bar{g} \upharpoonright_{H'} = g $. There are two
    cases.
    

        If $ \langle x \rangle \cap H' = \{0\} $, then take any $ y \in D $ and define $ \bar{g}(h + kx) = g(h) + ky $ for every $ h \in H' $ and $ k \in \mathbb{Z} $. Then
    $ \bar{g} $ is a homomorphism. This definition is correct because every element of $ N $ can be represented in a unique way
    as $ h + kx $, where $ h \in H' $ and $ k \in \mathbb{Z} $.


        If $ C = \langle x \rangle \cap H' \neq {0} $, then $ C $ is cyclic, being a subgroup of a cyclic group. So $ C = \langle lx \rangle $ for some $l \in \mathbb{Z} $.
    In particular, $ lx \in H' $ and we can consider the element $ a = g(lx) \in D $. Since $ D $ is divisible, there exists $ y \in D $
    such that $ly = a$. Now define $ \bar{g} : N \to D $ putting $ \bar{g}(h + ky) = g(h) + ky $ for every $ h + kx \in N $, where $ h \in H' $
    and $ k \in \mathbb{Z} $. To see that this definition is correct, suppose that $h + kx = h' + k'x$ for $ h, h' \in H' $ and $ k, k' \in \mathbb{Z} $.
    Then $ h - h' = k'x - kx = (k' - k)x \in C $. So $k - k' = sl$ for some $ s \in \mathbb{Z} $. Since $ g : H' \to D $ is a homomorphism
    and $ lx \in H' $, we have


    $ g(h) - g(h') = g(h - h') = g(s(lx)) = sg(lx) = sa = sly = (k' - k)y = k'y - ky $.


        Thus, from $ g(h) - g(h') = k'y - ky $ we conclude that $ g(h) + ky = g(h') + k'y $. Therefore $ \bar{g} $ is correctly defined.
    Moreover $ \bar{g} $ is a homomorphism and extends $ g $.


        Let $ M $ be the family of all subgroups $ H_i $ of $ G $ such that $ H \leq H_i $ and of all homomorphisms $ f_i : H_i \to D $
    that extend $ f : H \to D $. For $ (H_i, f_i),(H_j , f_j ) \in M $ put $ (H_i, f_i) \leq (Hj , fj ) $ if $ H_i \leq H_j $ and $ f_j $ extends $ f_i $. In this
    way $ (M, \leq) $ is partially ordered. Let $ {(H_i, f_i)}_{i \in I} $ a totally ordered subset of $ (M, \leq) $. Then $ H_0 = \bigcup_{i \in I} H_i $ is a
    subgroup of $ G $ and $ f_0 : H_0 \to D $ defined by $ f_0 (x) = f_i (x) $ whenever $ x \in H_i $, is a homomorphism that extends
    $ f_i $ for every $ i \in I $. This proves that $ (M, \leq) $ is inductive and so we can apply Zorn's lemma to find a maximal
    element $ (H_{max}, f_{max}) $ of $ (M, \leq) $. It is easy to see that $ H_{max} = G $.


        Suppose now that $ D $ contains elements of arbitrary finite order. If $ a \in G \\ H $, we can extend $ f $ to $ H + \langle a \rangle $
    defining it as in the first part of the proof. If $ \langle a \rangle \cap H = \{0\}$ then $ \bar{f}(h + ka) = f(h) + ky $ for every $ k \in \mathbb{Z} $, where
    $ y \in D \ \{0\} $. If $ \langle a \rangle \cap H \neq \{0\} $, since $ D $ contains elements of arbitrary order, we can choose $ y \in D $ such that
    $ \bar{f}(h + ka) = f(h) + ky $ with $ y \neq 0 $. In both cases $ \bar{f}(a) = y \neq 0 $.


    \textbf{Corollary 2.6.} \emph{Let $ G $ be an abelian group and $ H $ a subgroup of $ G $. If $ \chi \in Hom (H,\mathbb{T}) $ and a $ \in G \\ H $, then $ \chi $
    can be extended to $ \bar{\chi} \in Hom (G,\mathbb{T}) $, with $ \bar{\chi}(a) \neq 0 $.}
    
    
    \textbf{Corollary 2.7.} \emph{If $ G $ is an abelian group, then $ Hom (G,\mathbb{T}) $ separates the points of $ G $}


    \textbf{Corollary 2.8.} \emph{If $ G $ is an abelian group and $ D $ a divisible subgroup of $ G $, then there exists a subgroup $ B $ of $ G $
    such that $ G = D \times B $.}


    \emph{Proof.} Consider the homomorphism $ f : D \to G $ defined by $ f(x) = x $ for every $ x \in D $. By Theorem 2.5 we can
    extend $ f $ to $ f : G \to G $. Then put $ B = ker \bar{f} $ and observe that $ G = D + B $ and $ D \cap B = \{0\} $; consequently
    $ G \cong D \times B $.


    \textbf{Corollary 2.9.} \emph{Every abelian group $ G $ can be written as $ G = d(G) \times R $, where $ RT $ is a reduced subgroup of $ G $.}


    \emph{Proof.} By Corollary 2.8 there exists a subgroup $ R $ of $ G $ such that $ G = d(G) \times R $. To conclude that $ R $ is reduced
    it suffices to apply the definition of $ d(G) $.


        The ring of endomorphisms of the group $ Z(p^{\infty}) $ will be denoted by $ \mathbb{J}_p $, it is isomorphic the inverse limit
    $ \underleftarrow{lim} \mathbb{Z} / p^{n} \mathbb{Z} $, known also as the ring of \emph{p-adic integers}. The field of quotients of $ \mathbb{J}_p $ (i.e., the field of \emph{p-adic
    numbers}) will be denoted by $ \mathbb{Q}_p $. Sometimes we shall consider only the underlying groups of these rings (and
    speak of ”the group \emph{p}-adic integers”, or ”the group \emph{p}-adic numbers).


    \subsection{Background on topological spaces}


        We assume the reader is familiar with the basic definitions and notions related to topological spaces. For the
    sake of completeness we recall here some frequently used properties related to compactness.

    \textbf{Definition 2.10.} \emph{A topological space $ X $ is }

    
        \begin{itemize}

            \item compact \emph{if for every open cover of $ X $ there exists a finite subcover};
            
            \item Lindeloff \emph{if for every open cover of $ X $ there exists a countable subcover};

            \item locally compact \emph{if every point of $ X $ has compact neighborhood in $ X $};
         
            \item $\sigma$-compact \emph{if $ X $ is the union of countably many compact subsets};
            
            \item of first category, \emph{if $ X = \bigcup^{\infty}_{n=1} A_n $ and every $ A_n $ is a closed subset of $ X $ with empty interior};
            
            \item of second category, \emph{if $ X $ is not of first category};
            
            \item connected \emph{if for every proper open subset of $ X $ with open complement is empty}.

        \end{itemize}


            Here we recall properties of maps:
    

    \textbf{Definition 2.11.} For a map $ f : (X, \tau) \to (Y, \tau') $ between topological spaces and a point $ x \in X $ we say:

    
            \begin{itemize}
    
                \item $ f $ is \emph{continuous} at $ x $ if for every neighborhood $ U $ of $ f(x) $ in $ Y $ there exists a neighborhood $ V $ of $ x $ in $ X $
                such that $ f(V) \subseteq U $,
                
                \item $ f $ is open in $ x \in X $ if for every neighborhood $ V $ of $ x $ in $ X $ there exists a neighborhood $ U $ of $ f(x) $ in $ Y $ such
                that $ f(V) \supseteq U $,
    
                \item $ f $ is continuous (resp., open) if $ f $ is continuous (resp., open) at every point $ x \in X $.
                
                \item $ f $ is closed if the subset $ f(A) $ of $ Y $ is closed for every closed subset $ A \subseteq X $.
                
            \end{itemize}


                Some basic properties relating spaces to continuous maps are collected in the next lemma:



    \textbf{Lemma 2.12.} 

    
            \begin{itemize}

                \item \emph{If $ f : X \to Y $ is a continuous surjective map, then $ Y $ is compact (resp., Lindeloff, $\sigma$-compact,
                connected) whenever $ X $ has the same property.}
                    
                \item \emph{If $ X $ is a closed subspace of a space $ Y $ , then $ X $ is compact (resp., Lindeloff, $\sigma$-compact, locally compact)
                whenever $ Y $ has the same property}

                \item If $ X = \Pi_{i \in I} X_i $, then $ X $ is compact (resp., connected) if every space $ X_i $ has the same property. If $ I $ is
                finite, the same holds for local compactness and $\sigma$-compactness.
                    
            \end{itemize}

    
            A partially ordered set $ (A, \leq) $ is \emph{directed} if for every $ \chi $, $ \beta \in A $ there exists $ \gamma \in A $ such that $ \gamma \leq \alpha $ and $ \gamma \leq \beta $.
        A subset $ B $ of $ A $ is \emph{cofinal}, if for every $ \alpha \in A $ there exists $ \beta \in B $ with $ \beta \geq \alpha $.
        

            A \emph{net} in a topological space $ X $ is a map from a directed set $ A $ to $ X $. We write $ x_\alpha $ for the image of $ \alpha \in A $ so
        that the net can be written in the form $ N = \{x_{\alpha}\}_{\alpha \in A} $. A \emph{subnet} of a net $ N $ is $ S = \{x_{\beta}\}_{\beta \in B} $ such that $ B $ is a
        cofinal subset of $ A $.
        

            A net ${x_{alpha}}_{\alpha \in A}$ in $ X $ converges to $ x \in X $ if for every neighborhood $ U $ of $ x $ in $ X $ there exists $ \beta \in A $ such that
        $ \alpha \in A $ and $ \alpha \geq \beta $ implies $ \alpha \in U $.    
    
    
    \textbf{Lemma 2.13.} Let $ X $ \emph{a topological space.}

    
        \begin{itemize}

            \item (a) \emph{If $ Z $ is a subset of $ X $, then $ x \in bar{Z} $ if and only if there exists a net in $ Z $ converging to $ x $.}
                
            \item (b) \emph{$ X $ is compact if and only if every net in $ X $ has a convergent subnet}.

            \item (c) \emph{A function $ f : X \to Y $ (where $ Y $ is a topological space) is continuous if and only if $ f(x_{\alpha}) \to f(x) $ in $ Y $ for 
            every net $ \{ x_{\alpha} \}_{\alpha \in A} $ in $ X $ with $ x_{\alpha} \to x $}

            \item (d) \emph{The space $ X $ is Hausdorff if and only if every net in $ X $ converges to at most one point in X}.
                
        \end{itemize}


        Let us recall that the \emph{connected component} of a point $ x $ in a topological space $ X $ is the largest connected
    subset of $ X $ containing $ x $. It is always a closed subset of $ X $. The space $ X $ is called \emph{totally disconnected} if all
    connected components are singletons.
        
        
        In a topological space $ X $ the \emph{quasi-component} of a point $ x \in X $ is the intersection of all clopen sets of $ X $
    containing $ x $. 


    \textbf{Lemma 2.14.} (Shura-Bura) \emph{In a compact space X the quasi-components and the connected components coincide.}
    
    
        A topological space $ X $ \emph{zero-dimensional} if X has a base of clopen sets. Zero-dimensional $ T_2 $ spaces are totally
    disconnected (as every point is an intersection of clopen sets).


    \textbf{Theorem 2.15.} (Vedenissov) \emph{Every totally disconnected locally compact space is zero-dimensional.}
    
    
        By $\beta X$ we denote the \emph{Cech-Stone compactification} of a topological Tychonov space $ X $, that is the compact
    space $\beta X$ together with the dense immersion $i : X \to \beta X$, such that for every function $f : X \to [0, 1]$ there
    exists $f^{\beta} : \beta X → [0, 1]$ which extends $ f $ (this is equivalent to ask that every function of $ X $ to a compact space
    $ Y $ can be extended to $\beta X$). Here $\beta X$ will be used only for a discrete space $ X $.


    \textbf{Theorem 2.16 (Baire category theorem).} \emph{A Hausdorff locally compact space $ X $ is of second category.}
    
    
        \emph{Proof.} Suppose that $ X = \bigcup^{\infty}_{n=1} A_n $ and assume that every $ A_n $ is closed with empty interior. Then the sets
    $ D_n = G \\ A_n $ are open and dense in $ X $. To get a contradiction, we show that $ \bigcap^{\infty}_{n=1} D_n $ is dense, in particular
    non-empty (so $ G \neq \bigcup^{\infty}_{n=1} A_n $, a contradiction).
    
    
        We use the fact that a Hausdorff locally compact space is regular. Pick an arbitrary open set $ V \neq \varnothing $. Then
    there exists an open set $ U_0 \neq \varnothing $ with $ \bar{U}_0 $ compact and $ \bar{U}_0 \subseteq V $ . Since $ D_1 $ is dense, $ U_0 \cap D_1 \neq \varnothing $. Pick $ x_1 \in U_0 \cap D_1 $
    and an open set $ U_1 \ni x_1 $ in $ X $ with $ \bar{U}_1 $ compact and $ U_1 \subseteq U_0 \cap D_1 $ . Proceeding in this way, for every $ n \in N_+ $
    we can find an open set $ U_n \neq \varnothing $ in $ G $ with $\bar{U}_n$ compact and $ \bar{U}_n \subseteq U_{n-1} \cap D_n $. By the compactness of every $ \bar{U}_n $
    there exists a point $ x \in \bigcap^{\infty}_{n=1} \bar{U}_n $. Obviously, $ x \in V \cap \bigcap^{\infty}_{n=1} D_n $.


    \textbf{Lemma 2.17.} \emph{If $ G $ is a locally compact $\sigma$-compact space, then $ G $ is a Lindeloff space.}
    
    
    \emph{Proof.} Let $ G = \bigcup_{\alpha \in I} U_{\alpha} $. Since $ G $ is $\sigma$-compact, $ G = \bigcup^{\infty}_{n=1} K_n $ where each $ K_n $ is a compact subset of $ G $.
    Thus for every $ n \in \mathbb{N}_+ $ there exists a finite subset $ F_n $ of $ I $ such that $ K_n \subseteq \bigcup_{n \in F_n} U_n $. Now $ I_0 = \bigcup^{\infty}_{n=1} F_n $ is a
    countable subset of $ I $ and $ K_n \subseteq \bigcup_{\alpha \in I_0} U_{\alpha} $ for every $ N \in \mathbb{N}_+ $ yields $ G = \bigcup_{\alpha \in I_0} U_{\alpha} $.
        
    
        Let X be a topological space. Let $ C(X, \mathbb{C}) $ be the $\mathbb{C}$-algebra of all continuous complex valued functions on
    $ X $. If $ f \in C(X, \mathbb{C}) $ let


    $\|f\|^{\infty} = sup\{|f(x)| : x \in X\}$.


    \textbf{Theorem 2.18 (Stone-WeiersreaB theorem).} \emph{Let $ X $ be a compact topological space. A $\mathbb{C}$-subalgebra $ \mathcal{A} $ of
    $ \mathcal{C}(X, \mathbb{C}) $ containing all constants and closed under conjugation is dense in $ \mathcal{C}(X, \mathbb{C}) $ for the norm $ \| \|_{\infty} $ if and
    only if $ \mathcal{A} $ separates the points of $ X $.}
    
    
        We shall need in the sequel the following local form of Stone-WeierstraB theorem.


    \textbf{Corollary 2.19.} \emph{Let $ X $ be a compact topological space and $ f \in \mathcal{C}(X, \mathbb{C}) $. Then $ f $ can be uniformly approximated
    by a $ \mathbb{C} $-subalgebra $ \mathcal{A} $ of $ \mathcal{C}(X, \mathbb{C}) $ containing all constants and closed under the complex conjugation if and only
    if $ \mathcal{A} $ separates the points of $ X $ separated by $ f \in \mathcal{C}(X, \mathbb{C}) $.}
    
    
    \emph{Proof.} Denote by $ G : X \to \mathbb{C}^A $ the diagonal map of the family $\{g : g \in A\}$. Then $ Y = G(X) $ is a compact
    subspace of $ \mathbb{C}^A $ and by the compactness of $ X $, its subspace topology coincides with the quotient topology of
    the map $ G : X \to Y $ . The equivalence relation ~ in $ X $ determined by this quotient is as follows: $ x ~ y $ for
    $ x, y \in X $ by if and only if $ G(x) = G(y) $ (if and only if $ g(x) = g(y) $ for every $ g \in A $). Clearly, every continuous
    function $ h : X \to \mathbb{C} $, such that $ h(x) = h(y) $ for every pair $ x, y $ with $ x ~ y $, can be factorized as $ h = \bar{h} \circ q $, where
    $\bar{h} \in C(Y, \mathbb{C})$. In particular, this holds true for all $ g \in A $ and for $ f $ (for the latter case this follows from our
    hypothesis). The $\mathbb{C}$-subalgebra $ A \subseteq C(Y, \mathbb{C}) $ is closed under the complex conjugation and contains all constants.
    It is easy to see that it separates the points of $ Y $. Hence we can apply Stone - WeierstraB theorem 2.18 to $ Y $
    and $ \bar{A} $ to deduce that we can uniformly approximate the function $ \bar{f} $ by functions of $ \bar{A} $. This produces uniform
    approximation of the function $ f $ by functions of $ A $.
    


\section{General properties of topological groups}


    \subsection{Definition of a topological group}


    Let us start with the following fundamental concept:


    \textbf{Let $ G $ be a geoup}
        
    
        \begin{itemize}

            \item A topology $\tau$ on $ G $ is said to be a \emph{group topology} if the map $ f : G \times G \to G $ defined by $f(x, y) = xy^{-1}$
            is continuous.

            \item A \emph{topological group} is a pair $ (G, \tau) $ of a group $ G $ and a group topology $\tau$ on $ G $.
        \end{itemize}
        
    
        If $ \tau $ is Hausdorff (resp., compact, locally compact, connected, etc.), then the topological group $ (G, \tau) $ is
    called Hausdorff (resp., compact, locally compact, connected, etc.). Analogously, if $ G $ is cyclic (resp., abelian,
    nilpotent, etc.) the topological group $ (G, \tau) $ is called cyclic (resp. abelian, nilpotent, etc.). Obviously, a topology
    $ \tau $ on a group $ G $ is a group topology if the maps 
    

    $ \mu : G\times G \to G $ and $\iota : G \to G$


    defined by $ \mu(x, y) = xy $ and $ \iota(x) = x^{-1} $ are continuous when $ G \times G $ carries the product topology.
    Here are some examples, starting with two trivial ones: for every group $ G $ the discrete topology and the
    indiscrete topology on $ G $ are group topologies. Non-trivial examples of a topological group are provided by the
    additive group $ \mathbb{R} $ of the reals and by the multiplicative group $ \mathbb{S} $ of the complex numbers $ z $ with $ |z| = 1 $, equipped
    both with their usual topology. This extends to all powers $ \mathbb{R}^n $ and $ \mathbb{S}^n$. These are abelian topological groups.
    For every $ n $ the linear group $GL_n (\mathbb{R})$ equipped with the topology induced by $R^{n^2}$ is a non-abelian topological
    group. The groups $\mathbb{R}^n$ and $GL_n (R)$ are locally compact, while $\mathbb{S}$ is compact.


    \textbf{Example 3.2.} For every prime $ p $ the group $\mathbb{J}_p$ of $p$-adic integers carries the topology induced by $ \Pi^{\infty}_{n=1} \mathbb{Z}(p^n) $,
    when we consider it as the inverse limit $\underleftarrow{lim} \mathbb{Z}/p^n \mathbb{Z}$. The same topology can be obtained also when we consider $\mathbb{J}_p$
    as the ring of all endomorphims of the group $\mathbb{Z}(p^{\infty})$. Now $\mathbb{J}_p$ embeds into the product $\mathbb{Z}(p^{\infty})^{\mathbb{Z}(p^{\infty})}$ carrying the
    product topology, while $\mathbb{Z}(p^{\infty})$ is discrete. We leave to the reader the verification that this is a compact group
    topology on $\mathbb{J}_p$. Basic open neighborhoods of 0 in this topology are the subgroups $ p^n \mathbb{J}_p of (\mathbb{J}_p, +) $ (actually,
    these are ideals of the ring $\mathbb{J}_p$) for $n \in N$. The field $\mathbb{Q}_p$ becomes a locally compact group by declaring $\mathbb{J}_p$ open
    in $\mathbb{Q}_p$ (i.e., an element $x \in \mathbb{Q}_p$ has as typical neighborhoods the cosets $x + p^n \mathbb{J}_p, n \in \mathbb{N}$.
    
    
        Other examples of group topologies will be given in $\S$3.2.
    
    
        If $G$ is a topological group written multiplicatively and $a \in G$, then the translations $x \mapsto ax$ and $x \mapsto xa$
    as well as the \emph{internal automorphism} $x \mapsto axa^{-1}$ are homeomorphisms. Consequently, the group $G$ is discrete
    if the point 1 is isolated, i.e., the singleton \{1\} is open. In the sequel $aM$ will denote the image of a subset
    $M \subseteq G$ under the (left) translation $x \mapsto ax$, i.e., $aM := {am : m \in M}$. This notation will be extended also to
    families of subsets of $G$, in particular, for every filter $\mathcal{F}$ we denote by $a \mathcal{F}$ the filter ${aF : F \in \mathcal{F}}$.
    
    
        Making use of the homeomorphisms $x \mapsto ax$ one can prove:


    \textbf{Exercise 3.3.} \emph{Let $f : G \to H$ be a homomorphism between topological groups. Prove that $f$ is continuous
    (resp., open) if $f$ is continuous (resp., open) at $1 \in G$.}



        PAGE TEN BELOW



        For a topological group $G$ and $g \in G$ we denote by $\mathcal{V}_G,\tau (g)$ the filter of all neighborhoods of the element $g$
    of $G$. When no confusion is possible, we shall write briefly also $\mathcal{V}_G(g), \mathcal{V}_{\tau} (g)$ or even $\mathcal{V}(g)$. Among these filters
    the filter $V_{G,\tau} (1)$, obtained for the neutral element $g = 1$, plays a central role. It is useful to note that for every
    $a \in G$ the filter $\mathcal{V}_G (a)$ coincides with $a \mathcal{V}_G (1) = \mathcal{V}_G (1) a$. More precisely, we have the following:
    
    
    \textbf{Theorem 3.4.} \emph{Let $G$ be a group and let $\mathcal{V}(1)$ be the filter of all neighborhoods of 1 in some group topology $\tau$
    on G. Then:}


    \begin{itemize}

        \item (a) for every $U \in \mathcal{V}(1)$ there exists$ V \in V(1)$ with $V \times V \subseteq U$;
    
        \item (b) for every $U \in \mathcal{V}(1)$ there exists $V \in V(1)$ with $V^{-1} \subseteq U$;
    
        \item (c) for every $U \in \mathcal{V}(1)$ and for every $a \in G$ there exists $V \in \mathcal{V}(1)$ with $aV a^{-1} \subseteq U$.
    
    \end{itemize}


        Conversely, if $\mathcal{V}$ is a filter on $G$ satisfying (a), (b) and (c), then there exists a unique group topology $\tau$ on
    $G$ such that $\mathcal{V}$ coincides with the filter of all $\tau$-neighborhoods of 1 in $G$.

    
    \emph{Proof.} To prove (a) it suffices to apply the definition of the continuity of the multiplication $\mu : G \times G \to G$ at
    $(1, 1) \in G \times G$. Analogously, for (b) use the continuity of the map $\iota : G \to G$ at $1 \in G$. For item (c) use the
    continuity of the internal automorphism $x \mapsto axa^{-1}$ at $1 \in G$.
    
    
    Let $\mathcal{V}$ be a filter on $G$ satisfying all conditions (a), (b) and (c). Let us see first that every $U \in \mathcal{V}$ contains 1.
    In fact, take $W \in \mathcal{V}$ with $W \times W \subseteq U$ and choose $V \in V(1)$ with $V \subseteq W$ and $V^{-1} \subseteq W$. Then $1 \in V · V^{-1} \subseteq U$.


    Now define a topology $\tau$ on $G$ whose open sets $O$ are defined by the following property:
    
    
    $\tau := \{O \subseteq G : (\forall a \in \mathcal{O})(\exists U \in \mathcal{V})$ such that $aU \subseteq O\}$.
    
    
        It is easy to see that $\tau$ is a topology on $G$. Let us see now that for every $g \in G$ the filter $g \mathcal{V}$ coincides with the filter
    $V_{(G,\tau)}(g)$ of all $\tau$-neighborhoods of $g$ in $(G, \tau)$. The inclusion $g\mathcal{V} \supseteq \mathcal{V}(G,\tau)(g)$ is obvious. Assume $U \in \mathcal{V}$. To see
    that $gU \in V_{(G,\tau)}(g)$ we have to find a $\tau$-open $O \subseteq gU$ that contains $g$. Let $O := {h \in gU : (\exists W \in V) hW \subseteq gU}$.
    Obviously $g \in O$. To see that $O \in \tau$ pick $x \in O$. Then there exists $W \in \mathcal{V}$ with $xW \subseteq gU$. Let $V \in \mathcal{V}$ with
    $V \times V \subseteq W$, then $xV \subseteq O$ since $xvV \subseteq gU$ for every $v \in V$.
    
    
        We have seen that $\tau$ is a topology on $G$ such that the $\tau$-neighborhoods of any $x \in G$ are given by the filter
    $x \mathcal{V}$. It remains to see that $\tau$ is a group topology. To this end we have to prove that the map $(x, y) \mapsto xy^{-1}$ is
    continuous. Fix $x, y$ and pick a $U \in \mathcal{V}$. By (c) there exists a $W \in \mathcal{V}$ with $W y^{-1} \subseteq y^{-1}U$. Now choose $V \in \mathcal{V}$
    with $V · V^{-1} \subseteq W$. Then $O = xV \times yV$ is a neighborhood of $(x, y)$ in $G \times G$ and $f(O) \subseteq xV \times V^{-1}y^{-1} \subseteq xW y^{-1} \subseteq xy^{-1}U$.
    
    
    In the above theorem one can take instead of a filter $\mathcal{V}$ also a \emph{filter base}, i.e., a family $\mathcal{V}$ with the property
    
    
    $(\forall U \in \mathcal{V})(\forall V \in \mathcal{V})(\exists W \in \mathcal{V})W \subseteq U \cap V$
    
    
    beyond the proprieties $(a)-(c)$.


        A neighborhood $U \in \mathcal{V}(1)$ is \emph{symmetric}, if $U = U^{-1}$. Obviously, for every $U \in \mathcal{V}(1)$ the intersection
    $U \cap U^{-1} \in \mathcal{V}(1)$ is a symmetric neighborhood, hence every neighborhood of 1 contains a symmetric one.
    
    
    Let $\{\tau_i : i \in I\}$ be a family of group topologies on a group $G$. Then their supremum $\tau = sup_{i \in I} \tau_i$ is a group
    topology on $G$ with a base of neighborhoods of 1 formed by the family of all finite intersection $U_1 \cap U_2 \cap \dots \cap U_n$,
    where $U_k \in V_{\tau_{i_k}}(1)$ for $k = 1, 2, \dots, n$ and the $n$-tuple $i_1, i_2, \dots , i_n$ runs over all finite subsets if $I$.
    
    
    \textbf{Exercise 3.5.} \emph{If ($a_n$) is a sequence in $G$ such that $a_n \to 1$ for every member $\tau_i$ of a family $\{\tau_i: i \in I\}$ of group
    topologies on a group $G$, then $a_n \to 1$ also for the supremum $sup_{i \in I} \tau_i$.}
    
    
    \subsection{3.2 Examples of group topologies}
    
    
        Now we give several series of examples of group topologies, introducing them by means of the filter $\mathcal{V}(1)$ of
    neighborhoods of 1 as explained above. However, in all cases we avoid the treat the whole filter $\mathcal{V}(1)$ and
    we prefer to deal with an essential part of it, namely a base. Let us recall the precise definition of a base of
    neighborhoods.


    \textbf{Definition 3.6. } Let $G$ be a topological group. A family $\mathcal{B} \subseteq \mathcal{V}(1)$ is said to be a \emph{base of neighborhoods} of 1 (or
    briefly, \emph{a base at} 1) if for every $U \in \mathcal(1)$ there exists a $V \in \mathcal{B}$ contained in $U$ (such a family will necessarily be
    a filterbase).

\end{document}