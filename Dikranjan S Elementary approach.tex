\documentclass[12pt]{article}
\usepackage{amsmath}
\usepackage{graphicx}
\usepackage{hyperref}
\usepackage[utf8]{inputenc}
\usepackage{mathrsfs}
\usepackage{amssymb}

\title{An elementary approach to Haar integration and Pontryagin duality in
locally compact abelian groups}
\author{Dikran Dikranjan,  Luchezar Stoyanov }

\begin{document}

\maketitle

\begin{abstract}
        We offer an elementary proof of Pontryagin duality theorem for compact and discrete
    abelian groups. To this end we make use of an elementary proof of Peter-Weyl theorem
    due to Prodanov that makes no recourse to Haar integral. As a long series of applications
    of this approach we obtain proofs of Bohr-von Neumann's theorem on almost periodic
    functions, Comfort-Ross' theorem on the description of the precompact topologies on
    abelian groups, and, last but not least, the existence of Haar integral in LCA groups.
\end{abstract}

\section{Introduction}


        Pontryagin duality is a formidable tool in the theory of topological groups and harmonic analysis, as well as many other
    fields of mathematics [20]. This is why it is desirable that experts in all these fields become familiar with this extraordinary
    technique. Unfortunately, the doors to this magnificent castle are “guarded” by another masterpiece in mathematics, Peter-
    Weyl’s theorem in the sense that its proof is the hardest inevitable part in the proof of the duality theorem. Usually, the
    standard expositions in topological groups provide a proof of Peter-Weyl's theorem in full generality which requires a heavy
    machinery coming from functional analysis, whereas the proof of Pontryagin duality theorem needs only the Peter-Weyl's
    theorem in the abelian case. This motivated the first tentative of Prodanov to provide such a proof in [27]. A second proof
    appeared in the monograph [8], that is out of print since many years. Meanwhile, the proof was further improved as a result
    of its exposition at graduate courses or elsewhere. Since this approach seems to remain relatively unknown to a part of the
    wide audience, we decided to present here this improved version of the elementary proof from [8] as well as several
    applications with the hope that Prodanov’s ideas will become widely known and used. Other ideas of Prodanov related to
    dualities are discussed in [7].
    
    
        Let $\mathcal{L}$ denote the category of locally compact abelian (LCA) groups and continuous homomorphisms. Prominent examples
    of compact abelian groups are the powers of the circle group $\mathbb{T}$, as well as their closed subgroups. This is the most general
    instance of a compact abelian group, i.e. every compact abelian group is isomorphic to a closed subgroup of a power of $\mathbb{T}$ (see
    Corollary 5.4). Furthermore, every abelian group of the form    $\mathbb{R}^n \times G$, where $G$ has an open compact subgroup $K$, is locally
    compact. Actually, every locally compact abelian group has this form (see e.g. [15,25]).
    
    
        For $G \in \mathcal{L}$ denote by $\widehat{G}$ the group of continuous homomorphisms $G \to \mathcal{T}$ equipped with the compact-open topology
    having as a base of the neighborhoods of 0 the family of sets $W (K, U) = {\chi \in \widehat{G}: \chi(K) \subseteq U}$, where $K \subseteq G$ is compact and
    U is an open neighborhood of the neutral element in $\matbb{T}$. If $G \in \mathcal{L}$, then also $\widehat{G} \in \mathcal{L}$. So


    $G \mapsto \widehat{G}$, $f \mapsto \widehat{f}$


    is a contravariant endofunctor $\widehat{}:\mathcal{L} \to \mathcal{L}$, where $\widehat{f}$ is defined for $f : H \to G$ as $\widehat{f} : \widehat{G} \to \widehat{H}$, with $\mathcal{f}(\chi) = \chi \circ f$ for $\chi \in \widehat{H}$.
    
    
        Pontryagin-van Kampen duality theorem says that this functor is an involution i.e., $\tilde{\bar{\bar{G}}} \cong G$ for every $G \in \mathcal{L}$. Moreover,
    this functor sends compact groups to discrete ones and vice versa, i.e., it defines a duality between the subcategory $\mathcal{C}$ of
    compact abelian groups and the subcategory $\mathcal{D}$ of discrete abelian groups.
        Let $G$ be a topological abelian group. Define $\omega_G : G \to \widehat{\widehat{G}}$ such that $\omega_G (x)(\chi) = \chi(x)$, for every $\chi \in G$ and for every $\chi \in \widehat{G}$.
    In a more precise form, the Pontryagin-van Kampen duality theorem says that $\omega_G$ is a topological isomorphism for every
    $G \in \mathcal{L}$.


        Injectivity of $\omega_G(x)$ means that $\widehat{G}$ separates the points of $G$ (i.e., for every $x \in G \\ \{0\}$, there exists $\chi \in \widehat{G}$ with $\chi(x) \neq 1$).
    So the main step in the proof of the Pontryagin–van Kampen duality is the following theorem, proved in Section 5 (Corollary 5.3).


    \textbf{Peter-Weyl's theorem} \emph{(Alebian Case). If G is a compact alebian group, then $\widehat{G}$ separates the points of $G$}


        The main tool in the proof of this theorem is the so-called Følne'’s Theorem 5.1 which is a rather deep and general
    fact on its own account. An elementary proof of it was found by Prodanov [27] and we present it in Section 5. It is based
    on a topology-free local form of the Stone-Weierstraß theorem for subsets of an abelian group (Proposition 4.6) and the
    so-called Prodanov's lemma (see Section 4 below) which in certain situations derives existence of continuous characters on
    topological abelian groups from that of discontinuous ones. This remarkable lemma is very efficient in applications, yet its
    proof is completely elementary.


    Using Peter-Weyl's theorem we then derive


    \textbf{Pontryagin-van Kampen duality theorem} \emph{(Discrete-compact case)). If G is a discrete or compact alebian group, then $\omega_G : G \to \widehat{G}$ is a topological isomorphism)}


        The techniques developed in the first 5 sections of this paper have a wide range of possible applications, however here
    we limit ourselves with just two more substantial consequences. The first one is the description of the algebra $A(G)$ of
    almost periodic functions on an abelian group G and Bohr-von Neumann theorem (see Section 6). The second one is the
    following celebrated theorem:


    \textbf{Existance of Haar Integral}. \emph{Every locally compact abelian group $G$ admits a Haar integral, i.e. a non-zero linear functional
    $f : C_0 (G) → \mathbb{C}$ which is positive (i.e. if $f \in C_0 (G)$ is real-valued and $f \geqslant 0$, then also  $\int f \geqslant 0$) and invariant (i.e.,  $\int f_a =  f$ for
    every $f \in C_0(G)$ and $a \in G$, where $f_a (x) = f (x + a)$, and $C_0 (G)$ denotes the space of all continuous complex-valued functions with
    compact support}


        The proof proceeds roughly speaking like this: in Section 6 we show that for every abelian group $G$ there is a unique
    invariant mean $\mathcal{I}$ on $A(G)$ (Theorem 6.13). Applying Prodanov’s lemma we show that the one-dimensional subspace of $A(G)$
    formed by the constant functions splits as a direct summand. The resulting projection $A(G) \to \mathbb{C}$ gives a linear positive
    functional that is nothing else but the invariant mean $\mathcal{I}$. Using the fact that every continuous function on a compact group
    is almost periodic, this invariant mean gives also a (unique) Haar integral for compact abelian groups. Finally, the Haar
    integral is extended to arbitrary LCA groups (Theorem 7.5). We should mention that elementary proofs of the existence and
    uniqueness of the Haar integral on general locally compact groups were given by $H$. Cartan [4] (see also Chapter 4 in [15]
    or Section II.9 in [22]) and $A$. Weil (see e.g. Section II.8 in [22]). Our proof, fully exploiting the advantage of the abelian
    structure, is shorter and more transparent.


    The paper is organized as follows. In Section 2 we give the proof of Bogoliouboff-Følner's lemmas (the articulation of
    the proof does not faithfully follow [8, $\S\S$1.2-1.3], we give a new Lemma 2.2, while [8, Lemma 1.3.1] is absorbed in the
    proof of Lemma 2.5), Section 3 is dedicated to the subgroups of the compact group, namely the precompact groups. Here
    we use Folner's lemma to give a description of the zero-neighborhoods in the Bohr topology. Section 3 can be omitted at
    first reading by the reader interested to get to the proof of Pontryagin–van Kampen duality theorem as soon as possible.
    In Section 4 comes Prodanov's lemma and its first applications to independence of characters. Here we give a local form
    of Stone-WeierstraB theorem used further in the proof of Folner's theorem. In Section 5 come the proofs of Peter-Weyl's
    theorem, Folner's theorem, Comfort-Ross' theorem and Pontryagin-van Kampen duality theorem in the discrete-compact
    case. In Section 5.3 we give some facts on the structure of the locally compact abelian groups used here and in the last
    section. Section 6 is dedicated to the algebra $A(G)$ of almost periodic functions on an abelian group G and the proof of
    Bohr-von Neumann's Theorem 6.6. In Section 6.3 we build the invariant mean on A(G).


        We denote by $\mathbb{P}, \mathbb{N}$ and $\mathbb{N}_+$ respectively the set of primes, the set of natural numbers and the set of positive integers.
    The symbols $\mathbb{Z}, \mathbb{Q}, \mathbb{R}, \mathbb{C}$ will denote the integers, the rationals, the reals and the complex numbers, respectively. In the circle
    group $\mathbb{T}$ we use the multiplicative notation. Concerning abelian groups our notation and terminology follow [13], while for
    basic topological background we refer the reader to [10].


        For an abelian group $G$ we denote by $Hom(G,\mathbb{T})$ or $G^*$ the group of all homomorphisms from $G$ to $\mathbb{T}$ and we call the
    elements of $Hom(G,\mathbb{T})$ characters. So $\widehat{G}$ is a subgroup of $G^*$, when $G$ is a topological group. In particular, $G^* = G$ when $G$
    is discrete.


    When $G$ carries no specific topology, we shall always assume that $G$ is discrete, so that $\widehat{G} = G^*$. If $G$ is a topological
    abelian group, we shall denote by $G_d$ the group $G$ equipped with the discrete topology, so that $G^* = \widehat{G_d}$.
    
    
    If $M$ is a subset of $G$ then $<M>$ is the smallest subgroup of $G$ containing $M$ and $\bar{M}$ is the closure of $M$ in $G$.
    For a set $X$ we denote by $B(X)$, the $\mathbb{C}$-algebra of all bounded complex-valued functions on $X$. For $f \in B(X)$ let


    $\|f\| = sup\{|f(x)|: x \in X\}$\dots


        The $\mathbb{C}$-algebra $B(X)$ will always carry the uniform convergence topology, i.e., the topology induced by this norm. If $X$ is a
    topological space, $C(X)$ will denote the $\mathbb{C}$-algebra of all continuous complex-valued functions on $X$.
    In the sequel various subspaces of the $\mathbb{C}$-algebra $B(G)$ for an abelian group $G$ will be used. The group $G$ acts on $B(G)$
    by translations $f \mapsto f_a$, where $f_a (x) = f (x + a)$ for all $x \in G$ and $a \in G$. These translations are isometries, i.e., $\| f_a \|=\| f \|$ for
    all $a \in G$.


        If $G$ is a topological abelian group, we denote by $\mathfrak{X} (G)$ the $\mathbb{C}$-subspace of $B(G)$ spanned by $\widehat{G}$ (i.e., consisting of all linear
    combinations of continuous characters of the group $G$ with coefficients from $\mathbb{C}$) and by $\mathfrak{X}_0(G)$ its $\mathbb{C}$-subspace spanned by
    $\widehat{G} \ \{1\}$. Note that $\mathfrak{X}(G)$ is a subalgebra of $B(X)$ and $\mathfrak{X}(G) = \mathbb{C} · 1 + \mathfrak{X}_0(G)$, where 1 denotes the constant function 1. Both
    $\mathcal{X} (G)$ and $\mathfrak{X}_0(G)$ are invariant under the action $f \mapsto f_a$ of the group $G$. We shall see that $\mathfrak{X}_0(G)$ is a hyperspace of $\mathfrak{X}(G)$,
    i.e., $\mathfrak{X}(G) = \mathbb{C} · 1 \oplus \mathfrak{X}_0(G)$.
        
    
        Furthermore, let $\mathfrak{A}(G)$ denote the closure of $\mathfrak{X}(G)$ in $B(G)$ (i.e., the set of all functions $f \in B(G)$ such that for every $\epsilon > 0$
    there exists a $g \in \mathfrak{X}(G)$ with $\| f - g \| \leqslant \epsilon$). Hence $\mathfrak{A}(G)$ is again a $\mathbb{C}$-subalgebra of $B(G)$ containing all constants and closed
    under complex conjugation.
    
    
        Clearly, $\mathfrak{X}(G), \mathfrak{X}_0(G)$ and $\mathfrak{A}(G)$ make perfectly sense also for a \emph{non-topologized} group $G$, which will always be considered
    with the \emph{discrete topology}. Hence, in this case $\mathfrak{X}(G)$ is the $\mathbb{C}$-subalgebra of $B(G)$ spanned by $G^*, \mathfrak{A}(G)$ is the closure of $\mathfrak{X}(G)$
    in $B(G)$, etc. So, for a topological abelian group $G$ one can consider also the underlying discrete group $G_d$, so that $\mathfrak{X}(G)$ will
    be a $\mathbb{C}$-subalgebra of the larger subalgebra $\mathfrak{X}(G_d)$ (spanned by $G^*$). Analogously, $\mathfrak{A}(G)$ is a $\mathbb{C}$-subalgebra of $\mathfrak{A}(G_d)$ - the
    closure of $\mathfrak{X}(G_d)$ in $B(G)$.


        It is good to point out (see Corollary 4.3) that


        $\mathfrak{X}(G_d) \cap C(G) = \mathfrak{X}(G)$ and $C(G) \cap \mathfrak{A}(G_d) = \mathfrak{A}(G)$


        i.e., if a linear combination of distinct characters is continuous, then we can assume that all characters are continuous (and
    the same paradigm applies to uniform approximations by means of linear combinations of characters).


    \section{Bogoliouboff-Folner's lemma}


        A subset $X$ of an abelian group $G$ is \emph{big} if there exists a finite subset $F$ of $G$ such that $G = F + X$. If $|F | \leqslant k$, we say that
    $X$ is $k$-big.


    \textbf{Example 2.1.} Let $G$ be an abelian group. For characters $\chi_i, i=1,\dots,\mathbb{n}$ of $G$ and $\delta > 0$ let


    $U_G(\chi_1,...,\chi_n; \delta) := x \in G:= \{Arg(\chi_i(x)) < \delta, i = 1,...,n\}$.


        For a subset $E$ of an abelian group $G$ let $E_{(2)} = E - E, E_{(4)} = E - E + E - E, E_{(6)} = E - E + E - E + E - E$ and so on.
    Clearly, for every homomorphism $q : G \to H$ and $E \subseteq G$ one has $q(E_{(4)}) = q(E)_{(4)}$. The next lemma shows a more subtle
    aspect of this relation in the case of some special quotients.
    
    
    \textbf{Lemma 2.2.} \emph{. Let $F$ be an abelian group and let $r,t \in \mathbb{N}$. For $A = \mathbb{Z}^r \times F$ and $n \in N_+$ let $A_n = (-n,n]^r \times F, N = 2tn\mathbb{Z}^r$ (considered as
    a subgroup of $A$) and let $q : A \to A/N$ be the canonical homomorphism. If $B \subseteq A_n$, then every $a \in A_n$ with $q(a) \in q(B)_{(4)}$ satisfies}


    $a \in \begin{cases}
        B_(4), \text{if t>2, or}
        B_(4)+{0, \pm 2n \pm 4n}^r \times {0} \text{if t=1 or 2.}
    \end{cases}$


    \textbf{Proof.} It follows from our hypothesis $q(a) \in q(B)_{(4)}$ that there exist $b_1, b_2, b_3, b_4 \in B$ and $c = (c_i) \in N$ such that $a = b_1 -
    b_2 + b_3 - b_4 + c$. Now $c = a - b_1 + b_2 - b_3 + b_4 \in (A_n)_{(4)} + A_n$ implies $|c_i| \leqslant 5_n$ for each $i$. So $c = 0$, in case $t \geqslant 3$, as $2tn|c_i$
    for each $i$. Thus $a \in B_{(4)}$ when $t \geqslant 3$. Otherwise, from $2tn|c_i$ for every $i$ and $|c_i| \leqslant 5n$, we conclude that $c_i \in {0,\pm 2n \pm 4n}$
    for $i = 1, 2,...,r$. 
\end{document}
