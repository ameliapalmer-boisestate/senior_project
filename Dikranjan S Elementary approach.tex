\documentclass[12pt]{article}
\usepackage{amsmath}
\usepackage{graphicx}
\usepackage{hyperref}
\usepackage[utf8]{inputenc}
\usepackage{mathrsfs}
\usepackage{amssymb}

\title{An elementary approach to Haar integration and Pontryagin duality in
locally compact abelian groups}
\author{Dikran Dikranjan,  Luchezar Stoyanov }

\begin{document}

\maketitle

\begin{abstract}
        We offer an elementary proof of Pontryagin duality theorem for compact and discrete
    abelian groups. To this end we make use of an elementary proof of Peter-Weyl theorem
    due to Prodanov that makes no recourse to Haar integral. As a long series of applications
    of this approach we obtain proofs of Bohr-von Neumann's theorem on almost periodic
    functions, Comfort-Ross' theorem on the description of the precompact topologies on
    abelian groups, and, last but not least, the existence of Haar integral in LCA groups.
\end{abstract}

\section{Introduction}


        Pontryagin duality is a formidable tool in the theory of topological groups and harmonic analysis, as well as many other
    fields of mathematics [20]. This is why it is desirable that experts in all these fields become familiar with this extraordinary
    technique. Unfortunately, the doors to this magnificent castle are “guarded” by another masterpiece in mathematics, Peter-
    Weyl’s theorem in the sense that its proof is the hardest inevitable part in the proof of the duality theorem. Usually, the
    standard expositions in topological groups provide a proof of Peter-Weyl's theorem in full generality which requires a heavy
    machinery coming from functional analysis, whereas the proof of Pontryagin duality theorem needs only the Peter-Weyl's
    theorem in the abelian case. This motivated the first tentative of Prodanov to provide such a proof in [27]. A second proof
    appeared in the monograph [8], that is out of print since many years. Meanwhile, the proof was further improved as a result
    of its exposition at graduate courses or elsewhere. Since this approach seems to remain relatively unknown to a part of the
    wide audience, we decided to present here this improved version of the elementary proof from [8] as well as several
    applications with the hope that Prodanov’s ideas will become widely known and used. Other ideas of Prodanov related to
    dualities are discussed in [7].
    
    
        Let $\mathcal{L}$ denote the category of locally compact abelian (LCA) groups and continuous homomorphisms. Prominent examples
    of compact abelian groups are the powers of the circle group $\mathbb{T}$, as well as their closed subgroups. This is the most general
    instance of a compact abelian group, i.e. every compact abelian group is isomorphic to a closed subgroup of a power of $\mathbb{T}$ (see
    Corollary 5.4). Furthermore, every abelian group of the form    $\mathbb{R}^n \times G$, where $G$ has an open compact subgroup $K$, is locally
    compact. Actually, every locally compact abelian group has this form (see e.g. [15,25]).
    
    
        For $G \in \mathcal{L}$ denote by $\widehat{G}$ the group of continuous homomorphisms $G \to \mathcal{T}$ equipped with the compact-open topology
    having as a base of the neighborhoods of 0 the family of sets $W (K, U) = {\chi \in \widehat{G}: \chi(K) \subseteq U}$, where $K \subseteq G$ is compact and
    U is an open neighborhood of the neutral element in $\matbb{T}$. If $G \in \mathcal{L}$, then also $\widehat{G} \in \mathcal{L}$. So


    $G \mapsto \widehat{G}$, $f \mapsto \widehat{f}$


    is a contravariant endofunctor $\widehat{}:\mathcal{L} \to \mathcal{L}$, where $\widehat{f}$ is defined for $f : H \to G$ as $\widehat{f} : \widehat{G} \to \widehat{H}$, with $\mathcal{f}(\chi) = \chi \circ f$ for $\chi \in \widehat{H}$.
    
    
        Pontryagin-van Kampen duality theorem says that this functor is an involution i.e., $\tilde{\bar{\bar{G}}} \cong G$ for every $G \in \mathcal{L}$. Moreover,
    this functor sends compact groups to discrete ones and vice versa, i.e., it defines a duality between the subcategory $\mathcal{C}$ of
    compact abelian groups and the subcategory $\mathcal{D}$ of discrete abelian groups.
        Let $G$ be a topological abelian group. Define $\omega_G : G \to \widehat{\widehat{G}}$ such that $\omega_G (x)(\chi) = \chi(x)$, for every $\chi \in G$ and for every $\chi \in \widehat{G}$.
    In a more precise form, the Pontryagin-van Kampen duality theorem says that $\omega_G$ is a topological isomorphism for every
    $G \in \mathcal{L}$.


        Injectivity of $\omega_G(x)$ means that $\widehat{G}$ separates the points of $G$ (i.e., for every $x \in G \\ \{0\}$, there exists $\chi \in \widehat{G}$ with $\chi(x) \neq 1$).
    So the main step in the proof of the Pontryagin–van Kampen duality is the following theorem, proved in Section 5 (Corollary 5.3).


\end{document}