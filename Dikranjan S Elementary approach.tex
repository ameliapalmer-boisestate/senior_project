\documentclass[12pt]{article}
\usepackage{amsmath}
\usepackage{graphicx}
\usepackage{hyperref}
\usepackage[utf8]{inputenc}
\usepackage{mathrsfs}
\usepackage{amssymb}

\title{An elementary approach to Haar integration and Pontryagin duality in
locally compact abelian groups}
\author{Dikran Dikranjan,  Luchezar Stoyanov }

\begin{document}

\maketitle

\begin{abstract}
        We offer an elementary proof of Pontryagin duality theorem for compact and discrete
    abelian groups. To this end we make use of an elementary proof of Peter-Weyl theorem
    due to Prodanov that makes no recourse to Haar integral. As a long series of applications
    of this approach we obtain proofs of Bohr-von Neumann's theorem on almost periodic
    functions, Comfort-Ross' theorem on the description of the precompact topologies on
    abelian groups, and, last but not least, the existence of Haar integral in LCA groups.
\end{abstract}

\section{Introduction}


        Pontryagin duality is a formidable tool in the theory of topological groups and harmonic analysis, as well as many other
    fields of mathematics [20]. This is why it is desirable that experts in all these fields become familiar with this extraordinary
    technique. Unfortunately, the doors to this magnificent castle are “guarded” by another masterpiece in mathematics, Peter-
    Weyl’s theorem in the sense that its proof is the hardest inevitable part in the proof of the duality theorem. Usually, the
    standard expositions in topological groups provide a proof of Peter-Weyl's theorem in full generality which requires a heavy
    machinery coming from functional analysis, whereas the proof of Pontryagin duality theorem needs only the Peter-Weyl's
    theorem in the abelian case. This motivated the first tentative of Prodanov to provide such a proof in [27]. A second proof
    appeared in the monograph [8], that is out of print since many years. Meanwhile, the proof was further improved as a result
    of its exposition at graduate courses or elsewhere. Since this approach seems to remain relatively unknown to a part of the
    wide audience, we decided to present here this improved version of the elementary proof from [8] as well as several
    applications with the hope that Prodanov’s ideas will become widely known and used. Other ideas of Prodanov related to
    dualities are discussed in [7].
    
    
        Let $\mathcal{L}$ denote the category of locally compact abelian (LCA) groups and continuous homomorphisms. Prominent examples
    of compact abelian groups are the powers of the circle group $\mathbb{T}$, as well as their closed subgroups. This is the most general
    instance of a compact abelian group, i.e. every compact abelian group is isomorphic to a closed subgroup of a power of $\mathbb{T}$ (see
    Corollary 5.4). Furthermore, every abelian group of the form    $\mathbb{R}^n \times G$, where $G$ has an open compact subgroup $K$, is locally
    compact. Actually, every locally compact abelian group has this form (see e.g. [15,25]).
    
    
        For $G \in \mathcal{L}$ denote by $\widehat{G}$ the group of continuous homomorphisms $G \to \mathcal{T}$ equipped with the compact-open topology
    having as a base of the neighborhoods of 0 the family of sets $W (K, U) = {\chi \in \widehat{G}: \chi(K) \subseteq U}$, where $K \subseteq G$ is compact and
    U is an open neighborhood of the neutral element in $\matbb{T}$. If $G \in \mathcal{L}$, then also $\widehat{G} \in \mathcal{L}$. So


    $G \mapsto \widehat{G}$, $f \mapsto \widehat{f}$


    is a contravariant endofunctor $\widehat{}:\mathcal{L} \to \mathcal{L}$, where $\widehat{f}$ is defined for $f : H \to G$ as $\widehat{f} : \widehat{G} \to \widehat{H}$, with $\mathcal{f}(\chi) = \chi \circ f$ for $\chi \in \widehat{H}$.
    
    
        Pontryagin-van Kampen duality theorem says that this functor is an involution i.e., $\tilde{\bar{\bar{G}}} \cong G$ for every $G \in \mathcal{L}$. Moreover,
    this functor sends compact groups to discrete ones and vice versa, i.e., it defines a duality between the subcategory $\mathcal{C}$ of
    compact abelian groups and the subcategory $\mathcal{D}$ of discrete abelian groups.
        Let $G$ be a topological abelian group. Define $\omega_G : G \to \widehat{\widehat{G}}$ such that $\omega_G (x)(\chi) = \chi(x)$, for every $\chi \in G$ and for every $\chi \in \widehat{G}$.
    In a more precise form, the Pontryagin-van Kampen duality theorem says that $\omega_G$ is a topological isomorphism for every
    $G \in \mathcal{L}$.


        Injectivity of $\omega_G(x)$ means that $\widehat{G}$ separates the points of $G$ (i.e., for every $x \in G \\ \{0\}$, there exists $\chi \in \widehat{G}$ with $\chi(x) \neq 1$).
    So the main step in the proof of the Pontryagin–van Kampen duality is the following theorem, proved in Section 5 (Corollary 5.3).


    \textbf{Peter-Weyl's theorem} \emph{(Alebian Case). If G is a compact alebian group, then $\widehat{G}$ separates the points of $G$}


        The main tool in the proof of this theorem is the so-called Følne'’s Theorem 5.1 which is a rather deep and general
    fact on its own account. An elementary proof of it was found by Prodanov [27] and we present it in Section 5. It is based
    on a topology-free local form of the Stone-Weierstraß theorem for subsets of an abelian group (Proposition 4.6) and the
    so-called Prodanov's lemma (see Section 4 below) which in certain situations derives existence of continuous characters on
    topological abelian groups from that of discontinuous ones. This remarkable lemma is very efficient in applications, yet its
    proof is completely elementary.


    Using Peter-Weyl's theorem we then derive


    \textbf{Pontryagin-van Kampen duality theorem} \emph{(Discrete-compact case)). If G is a discrete or compact alebian group, then $\omega_G : G \to \widehat{G}$ is a topological isomorphism)}


        The techniques developed in the first 5 sections of this paper have a wide range of possible applications, however here
    we limit ourselves with just two more substantial consequences. The first one is the description of the algebra $A(G)$ of
    almost periodic functions on an abelian group G and Bohr-von Neumann theorem (see Section 6). The second one is the
    following celebrated theorem:


    \textbf{Existance of Haar Integral}. \emph{Every locally compact abelian group $G$ admits a Haar integral, i.e. a non-zero linear functional
    $f : C_0 (G) → \mathbb{C}$ which is positive (i.e. if $f \in C_0 (G)$ is real-valued and $f \geqslant 0$, then also  $\int f \geqslant 0$) and invariant (i.e.,  $\int f_a =  f$ for
    every $f \in C_0(G)$ and $a \in G$, where $f_a (x) = f (x + a)$, and $C_0 (G)$ denotes the space of all continuous complex-valued functions with
    compact support}


        The proof proceeds roughly speaking like this: in Section 6 we show that for every abelian group $G$ there is a unique
    invariant mean $\mathcal{I}$ on $A(G)$ (Theorem 6.13). Applying Prodanov’s lemma we show that the one-dimensional subspace of $A(G)$
    formed by the constant functions splits as a direct summand. The resulting projection $A(G) \to \mathbb{C}$ gives a linear positive
    functional that is nothing else but the invariant mean $\mathcal{I}$. Using the fact that every continuous function on a compact group
    is almost periodic, this invariant mean gives also a (unique) Haar integral for compact abelian groups. Finally, the Haar
    integral is extended to arbitrary LCA groups (Theorem 7.5). We should mention that elementary proofs of the existence and
    uniqueness of the Haar integral on general locally compact groups were given by $H$. Cartan [4] (see also Chapter 4 in [15]
    or Section II.9 in [22]) and $A$. Weil (see e.g. Section II.8 in [22]). Our proof, fully exploiting the advantage of the abelian
    structure, is shorter and more transparent.


    The paper is organized as follows. In Section 2 we give the proof of Bogoliouboff-Følner's lemmas (the articulation of
    the proof does not faithfully follow [8, $\S\S$1.2-1.3], we give a new Lemma 2.2, while [8, Lemma 1.3.1] is absorbed in the
    proof of Lemma 2.5), Section 3 is dedicated to the subgroups of the compact group, namely the precompact groups. Here
    we use Folner's lemma to give a description of the zero-neighborhoods in the Bohr topology. Section 3 can be omitted at
    first reading by the reader interested to get to the proof of Pontryagin–van Kampen duality theorem as soon as possible.
    In Section 4 comes Prodanov's lemma and its first applications to independence of characters. Here we give a local form
    of Stone-WeierstraB theorem used further in the proof of Folner's theorem. In Section 5 come the proofs of Peter-Weyl's
    theorem, Folner's theorem, Comfort-Ross' theorem and Pontryagin-van Kampen duality theorem in the discrete-compact
    case. In Section 5.3 we give some facts on the structure of the locally compact abelian groups used here and in the last
    section. Section 6 is dedicated to the algebra $A(G)$ of almost periodic functions on an abelian group G and the proof of
    Bohr-von Neumann's Theorem 6.6. In Section 6.3 we build the invariant mean on A(G).


        We denote by $\mathbb{P}, \mathbb{N}$ and $\mathbb{N}_+$ respectively the set of primes, the set of natural numbers and the set of positive integers.
    The symbols $\mathbb{Z}, \mathbb{Q}, \mathbb{R}, \mathbb{C}$ will denote the integers, the rationals, the reals and the complex numbers, respectively. In the circle
    group $\mathbb{T}$ we use the multiplicative notation. Concerning abelian groups our notation and terminology follow [13], while for
    basic topological background we refer the reader to [10].


        For an abelian group $G$ we denote by $Hom(G,\mathbb{T})$ or $G^*$ the group of all homomorphisms from $G$ to $\mathbb{T}$ and we call the
    elements of $Hom(G,\mathbb{T})$ characters. So $\widehat{G}$ is a subgroup of $G^*$, when $G$ is a topological group. In particular, $G^* = G$ when $G$
    is discrete.


    When $G$ carries no specific topology, we shall always assume that $G$ is discrete, so that $\widehat{G} = G^*$. If $G$ is a topological
    abelian group, we shall denote by $G_d$ the group $G$ equipped with the discrete topology, so that $G^* = \widehat{G_d}$.
    
    
    If $M$ is a subset of $G$ then $<M>$ is the smallest subgroup of $G$ containing $M$ and $\bar{M}$ is the closure of $M$ in $G$.
    For a set $X$ we denote by $B(X)$, the $\mathbb{C}$-algebra of all bounded complex-valued functions on $X$. For $f \in B(X)$ let


    $\|f\| = sup\{|f(x)|: x \in X\}$\dots


        The $\mathbb{C}$-algebra $B(X)$ will always carry the uniform convergence topology, i.e., the topology induced by this norm. If $X$ is a
    topological space, $C(X)$ will denote the $\mathbb{C}$-algebra of all continuous complex-valued functions on $X$.
    In the sequel various subspaces of the $\mathbb{C}$-algebra $B(G)$ for an abelian group $G$ will be used. The group $G$ acts on $B(G)$
    by translations $f \mapsto f_a$, where $f_a (x) = f (x + a)$ for all $x \in G$ and $a \in G$. These translations are isometries, i.e., $\| f_a \|=\| f \|$ for
    all $a \in G$.


        If $G$ is a topological abelian group, we denote by $\mathfrak{X} (G)$ the $\mathbb{C}$-subspace of $B(G)$ spanned by $\widehat{G}$ (i.e., consisting of all linear
    combinations of continuous characters of the group $G$ with coefficients from $\mathbb{C}$) and by $\mathfrak{X}_0(G)$ its $\mathbb{C}$-subspace spanned by
    $\widehat{G} \ \{1\}$. Note that $\mathfrak{X}(G)$ is a subalgebra of $B(X)$ and $\mathfrak{X}(G) = \mathbb{C} \cdot 1 + \mathfrak{X}_0(G)$, where 1 denotes the constant function 1. Both
    $\mathcal{X} (G)$ and $\mathfrak{X}_0(G)$ are invariant under the action $f \mapsto f_a$ of the group $G$. We shall see that $\mathfrak{X}_0(G)$ is a hyperspace of $\mathfrak{X}(G)$,
    i.e., $\mathfrak{X}(G) = \mathbb{C} \cdot 1 \oplus \mathfrak{X}_0(G)$.
        
    
        Furthermore, let $\mathfrak{A}(G)$ denote the closure of $\mathfrak{X}(G)$ in $B(G)$ (i.e., the set of all functions $f \in B(G)$ such that for every $\epsilon > 0$
    there exists a $g \in \mathfrak{X}(G)$ with $\| f - g \| \leqslant \epsilon$). Hence $\mathfrak{A}(G)$ is again a $\mathbb{C}$-subalgebra of $B(G)$ containing all constants and closed
    under complex conjugation.
    
    
        Clearly, $\mathfrak{X}(G), \mathfrak{X}_0(G)$ and $\mathfrak{A}(G)$ make perfectly sense also for a \emph{non-topologized} group $G$, which will always be considered
    with the \emph{discrete topology}. Hence, in this case $\mathfrak{X}(G)$ is the $\mathbb{C}$-subalgebra of $B(G)$ spanned by $G^*, \mathfrak{A}(G)$ is the closure of $\mathfrak{X}(G)$
    in $B(G)$, etc. So, for a topological abelian group $G$ one can consider also the underlying discrete group $G_d$, so that $\mathfrak{X}(G)$ will
    be a $\mathbb{C}$-subalgebra of the larger subalgebra $\mathfrak{X}(G_d)$ (spanned by $G^*$). Analogously, $\mathfrak{A}(G)$ is a $\mathbb{C}$-subalgebra of $\mathfrak{A}(G_d)$ - the
    closure of $\mathfrak{X}(G_d)$ in $B(G)$.


        It is good to point out (see Corollary 4.3) that


        $\mathfrak{X}(G_d) \cap C(G) = \mathfrak{X}(G)$ and $C(G) \cap \mathfrak{A}(G_d) = \mathfrak{A}(G)$


        i.e., if a linear combination of distinct characters is continuous, then we can assume that all characters are continuous (and
    the same paradigm applies to uniform approximations by means of linear combinations of characters).


    \section{Bogoliouboff-Folner's lemma}


        A subset $X$ of an abelian group $G$ is \emph{big} if there exists a finite subset $F$ of $G$ such that $G = F + X$. If $|F | \leqslant k$, we say that
    $X$ is $k$-big.


    \textbf{Example 2.1.} Let $G$ be an abelian group. For characters $\chi_i, i=1,\dots,\mathbb{n}$ of $G$ and $\delta > 0$ let


    $U_G(\chi_1,...,\chi_n; \delta) := x \in G:= \{Arg(\chi_i(x)) < \delta, i = 1,...,n\}$.


        For a subset $E$ of an abelian group $G$ let $E_{(2)} = E - E, E_{(4)} = E - E + E - E, E_{(6)} = E - E + E - E + E - E$ and so on.
    Clearly, for every homomorphism $q : G \to H$ and $E \subseteq G$ one has $q(E_{(4)}) = q(E)_{(4)}$. The next lemma shows a more subtle
    aspect of this relation in the case of some special quotients.
    
    
    \textbf{Lemma 2.2.} \emph{. Let $F$ be an abelian group and let $r,t \in \mathbb{N}$. For $A = \mathbb{Z}^r \times F$ and $n \in N_+$ let $A_n = (-n,n]^r \times F, N = 2tn\mathbb{Z}^r$ (considered as
    a subgroup of $A$) and let $q : A \to A/N$ be the canonical homomorphism. If $B \subseteq A_n$, then every $a \in A_n$ with $q(a) \in q(B)_{(4)}$ satisfies}


    $a \in \begin{cases}
        B_(4), \text{if t>2, or}
        B_(4)+{0, \pm 2n \pm 4n}^r \times {0} \text{if t=1 or 2.}
    \end{cases}$


    \textbf{Proof.} It follows from our hypothesis $q(a) \in q(B)_{(4)}$ that there exist $b_1, b_2, b_3, b_4 \in B$ and $c = (c_i) \in N$ such that $a = b_1 -
    b_2 + b_3 - b_4 + c$. Now $c = a - b_1 + b_2 - b_3 + b_4 \in (A_n)_{(4)} + A_n$ implies $|c_i| \leqslant 5_n$ for each $i$. So $c = 0$, in case $t \geqslant 3$, as $2tn|c_i$
    for each $i$. Thus $a \in B_{(4)}$ when $t \geqslant 3$. Otherwise, from $2tn|c_i$ for every $i$ and $|c_i| \leqslant 5n$, we conclude that $c_i \in {0,\pm 2n \pm 4n}$
    for $i = 1, 2,...,r$. 


        For every big set $E$ containing a big set of the form $U_G (\chi_1,...,\chi_n; \delta)$ in a group $G$, one can find a big set $V$ (namely
    $V = U_G (\chi_1,...,\chi_n; \delta/8))$, such that $V_{(8)} \subseteq E$. The main result of this section shows that this property can be inverted in
    the following sense (see also Corollary 3.5):


    \textbf{Lemma 2.3} (Folner's lemma). Let $A$ be an abelian group. If $k$ is a positive integer and $V$ is a $k$-big subset of $A$, then there exist
    $\chi_1,...,\chi_m \in A^*$, where $m = k^2$, such that $U_A (\chi_1,...,\chi_m; \frac{\pi}{2}) \subseteq V_{(8)}$.
    
    
        For $A = \mathbb{Z}$ Lemma 2.3 is due to Bogoliouboff [2]. He proved that in this case $V_{(8)}$ can be replaced by the smaller set $V_{(4)}$.
    In the general case the lemma was proved by Folner [11]. In [12] he (as well as Ellis and Keynes [9]) proved that also in
    this case $V_{(8)}$ can be replaced by the smaller set $V_{(4)}$, but this proof uses integrals (mean values). We are not aware if this
    fact can be proved without any recourse to such tools.
    
    
        The first step is the following elementary lemma due to Bogoliouboff [2] for cyclic groups, the general case is due to
    Folner [11]. A proof of this lemma can be found also in Maak [19] or [8, Lemma 1.2.3].


    \textbf{Lemma 2.4} (Bogoliouboff lemma). If $G$ is a finite abelian group and $E$ is a non-empty subset of $G$, then there exist $\chi_1,...,\chi_m \in G^*$,
    where $m = [{(\frac{|G|}{|E|})}^2]$, such that $U(\chi_1,...,\chi_m; \frac{\pi}{2} ) \subseteq E_{(4)}$.
    
    
        Let us note that the estimate for the number $m$ of characters is certainly non-optimal when $E$ is too small. For example,
    when $E$ is just the singleton $\{0\}$, the upper bound given by the lemma is just $|G|^2$, while one can certainly find at most
    $m = |G| - 1$ characters $\chi_1,...,\chi_m$ (namely, all non-trivial $\chi_i \in G^*$) such that $U(\chi_1,...,\chi_m; \frac{\pi}{2} ) = \{0\}$. For certain groups
    (e.g., $G = \mathbb{Z}^k_2$) one can find even a much smaller number (say $m = \log_2 |G|)$. Nevertheless, in the cases relevant for the proof
    of Følner's theorem, namely when the subset $E$ is relatively large with respect to $G$, this estimate seems more reasonable.
    
    
        The next step is to extend Bogoliouboff's lemma to finitely generated abelian groups:


    \textbf{Lemma 2.5} (Bogoliouboff-Folner's lemma). Let $A$ be a finitely generated abelian group and let $r = r_0 (A)$ be the rank of $A$. If $k$ is
    a positive integer and $V$ is a $k$-big subset of $A$, then there exist $\varrho_1,...,\varrho_s \subseteq A^*$, where $s = 3^{2r} k^2$, such that $U_A (\varrho_1,...,\varrho_s; \frac{\pi}{2}) \subseteq V_{(4)}$.


    \textbf{Proof.}. Fix $\varepsilon > 0$ sufficiently small to have $[\frac{3^{2r} k^2}{(1-k \epsilon)^2}] = 3^{2r} k^2$. Let $A = \mathbb{Z}^r \times F$, where $F$ is a finite abelian group and put
    $A_n = (-n,n]^r \times F$ . It is easy to prove that for $a = (a_1,...,a_r; f ) \in \mathbb{Z}^r \times F$ there exists n0 ∈ N such that

    
        $|(An - a) \cap A_n| \geqslant  |F| \cdot \Pi^r_{i=1} (2n - |a_i|)$ 


    for all $n > n_0$. From $|A_n|=|F|(2n)^r$, one easily gets


        $lim_{n \to \infty} \frac{|(A_n - a) \cap A_n|}{|A_n|} = 1 $  (1)


    for every $a ∈ A$. Let us see that this yields


        $|V \cap A_n| > (\frac{1}{k} - \varepsilon)|An|$  (2)


    for all sufficiently large $n$. Let $\bigcup^{k}_{i=1} (a_i + V) = A$ for $a_1,...,a_k \in A$. From (1) there exists $N > n_0$ such that
    $|(A_n - a_i) \bigcap A_n| > (1 - \varepsilon)|A_n|$ for every $n \geqslant N$ and consequently,


        $|(A_n - a_i) \ A_n| < \varepsilon|A_n|$  (3)


    for every $i = 1,...,k$. Since $A_n = \bigcup^k_{i=1}(a_i + V ) \cap A_n$, for every $n$ there exists $i_n \in {1,...,k}$ such that


        $\frac{1}{k}|A_n| \leqslant |(a_{i_n} + V ) \cap A_n| = |V \cap (A_n - a_{i_n})|$.


    Since $V \cap (A_n - a_{i_n} ) \subseteq (V \cap A_n) \cup ((A_n - a_{i_n} ) \ A_n)$, (3) yields


        $\frac{1}{k}|A_n| \leqslant | V \cap (A_n - a_{i_n} ) \leqslant |V \cap A_n| + |(A_n - a_{i_n} ) \ A_n | < |V \cap A_n| + \varepsilon|A_n|.$


    This proves (2).


        For $n$ with (2) let $G = A/(6n \mathbb{Z}^r)$ and $E = q(V \cap A_n)$ where $q$ is the canonical projection of $A$ onto $G$. Observe that $q  \upharpoonright {A_n}$
    is injective, as $(A_n - A_n) \cap ker q = {0}$. Then (2) gives


        $|E|=|V \cap A_n| > (\frac{1}{k} - \varepsilon) |An| = (\frac{1}{k} - \varepsilon)(2n)^r|F|$


    and so


        $\frac{|G|}{|E|} \leqslant \frac{(6n)^r |F|}{(\frac{1}{k} - \varepsilon) (2n)^r |F|} = \frac{3^r k}{1 - k \varepsilon}$


    By the choice of $\varepsilon$, this yields $[(\frac{|G|}{|E|})^2] \leqslant  s$. Pick $n > $N to have (2). Apply Lemma 2.4 to find $s$ characters $\xi_{1,n},...,\xi_{s,n} ∈ G^*$
    such that


        $U_G (\xi_{1,n},...,\xi_{s,n}; \frac{\pi}{2}) \subseteq E_{(4)}$.


    For $j = 1,..., s$ let $\varrho_{j,n} = \xi_{j,n} \circ q \in A^*$. If $a \in A_n \cap U A(\varrho_1,n,...,\varrho_s,n; \frac{\pi}{2} )$ then


        $q(a) \in U_G (\xi_{1,n},...,\xi_{s,n}; \frac{\pi}{2}) \subseteq E_{(4)} = q(V \cap A_n)_{(4)}$.


    So for $B = V \cap A_n$ and $t = 3$ we can apply Lemma 2.2 to get $a \in V_{(4)}$. Therefore, $A_n \cap U_A(\varrho_{1,n},...,\varrho_{s,n}; \frac{\pi}{2} ) \subseteq V_{(4)}$ for all $n$
    satisfying (2).


        By the compactness of the metrizable group $A^*$, there exist $\varrho_1,...,\varrho_s \in A^*$ and a subsequence $\{n_l\}_l$ of $\{n\}_{n \in \mathbb{N}_+}$ such that
    $\varrho_i(a) = lim_l \varrho_{i,nl} (a)$ for every $i = 1,..., s$ and $a \in A$. Then


    $U_A (\varrho_1,...,\varrho_s ;\frac{\pi}{2}) \subseteq V_{(4)}$. (4)


        Indeed, take $a \in U_A(\varrho_1,...,\varrho_s; \frac{\pi}{2})$. Since $A = \bigcup^{\infty}_{l=m} A_{n_l}$ for every $m \in \mathbb{N}_+$, there exists $n_0 > N$ and $a \in A_{n_0}$. As
    $\varrho_i(a) = lim_l \varrho_{i,n_l}(a)$ for every i$ = 1,..., s$, we can pick $l$ to have $n_l \leqslant n_0$ and $|Arg(\varrho_{i,n_l} (a))| < \pi/2$ for every $i = 1,..., s$, i.e.,
    $a \in A_{n_l} \cap U_A(\varrho_1,n_l,...,\varrho_s,n_l; \frac{\pi}{2} ) \subseteq V_{(4)}$. This proves (4). 


        Now we see that by a more careful application of the same argument one can eliminate the dependence of the number $s$
    of characters on the free rank $r$ of the group $A$ in Bogoliouboff-Folner's lemma. The price to pay for this is taking the larger
    set $V_{(8)}$ instead of $V_{(4)}$. However, this is irrelevant since we obtain the special case of Lemma 2.3 for \emph{finitely generated} groups.


    \textbf{Corollary 2.6.} Lemma 2.3 holds true for finitely generated groups.


    \textbf{Proof.} Let $A$ be a finitely generated abelian group, let $k$ be a positive integer and $m = k^2$. We have to prove that if $V$ is a
    $k$-big subset of $A$, then there exist $\chi_1,...,\chi_m \in A^*$, such that $U_A(\chi_1,...,\chi_m; \frac{\pi}{2} ) \subseteq V_{(8)}$.


    Let $r = r_0 (A)$ and $s = 3^{2r}k^2$. By Lemma 2.5 there exist $\varrho_1,...,\varrho_s \in A^*$ such that


        $U_A (\varrho_1,...,\varrho_s; \frac{\pi}{2}) \subseteq V_{(4)}$.


    Since $A$ is finitely generated, we can identify $A$ with $\mathbb{Z}^r \times F$, where $F$ is a finite abelian group. For $t \in {1,...,r}$ define a
    monomorphism $i_t : \mathbb{Z} \to A$ by letting


    $i_t (n) = (\underbrace{0,..., 0,n}_{t}, 0,..., 0; 0) \in A$.
    

    Then each $κ_{j,t} = \varrho_j \circ i_t$, where $j \in {1,..., s}$, $t \in {1,...,r}$, is a character of $\mathbb{Z}$. By Example 2.1 the subset


        $L = U_{\mathbb{Z}} ({κ_{j,t}: j = 1,..., s, t = 1,...,r}; \frac{\pi}{8r})$

    
    of $\mathbb{Z}$ is big, so infinite. Let $L^0 = \bigcup^{r}_{t=1} i_t (L)$, i.e., this is the set of all elements of $A$ of the form $i_t (n)$ with $n \in \mathbf{L}$ and
    $t \in {1,...,r}$. Obviously, $L^0 \subseteq U_A(\varrho_1,...,\varrho_s; \frac{\pi}{8r} )$, so


        $L^0_{(4r)} \subseteq U_A(\varrho_1,...,\varrho_s; \frac{\pi}{2}) \subseteq V_{(4)}$. (5)


    Define $A_n = (-n,n]^r \times F$ and pick $\varepsilon > 0$ such that $\varepsilon < \frac{1}{6k^4}$ . Then $[( \frac{k}{1-k\epsilon} )^2] = k^2$, and as in the proof of (2), we get
    $|V \cap A_n| > ( \frac{1}{k} - \varepsilon)|A_n|$ for sufficiently large $n$. Moreover, we choose this sufficiently large $n$ from $L$. Let
    $G_n = A/(2n \mathbb{Z}^r) \cong Z^r_{2n} \times F$ and $E = q(A_n \cap V)$ where $q$ is the canonical projection $A \to G_n$. Then $q\upharpoonright_{A_n}$ is injective as $(A_n - A_n) \cap ker q = 0$.
    So $q$ induces a bijection between $A_n$ and $G_n$ on one hand, and between $V \cap A_n$ and $E$. Thus $|A_n|=|G_n| = (2n)^r|F|$, $|E| =|V \cap A_n| > ( \frac{1}{k} - \varepsilon)|A_n|$ and so


        $(\frac{|G_n|}{|E|})^2 \leqslant (\frac{k}{1-k \varepsilon})^2$, hence $[(\frac{|G_n|}{|E|})^2]\leqslant[(\frac{k}{1-k \varepsilon})^2]=k^2$.
    

    To the finite group $G_n$ apply Lemma 2.4 to get $\xi_{1,n},...,\xi_{m,n} \in G^*_n$, with $m = k^2$, such that


        $U_{G_n}(\xi_{1,n}, . . . , \xi_{m,n}; \frac{\pi}{2}) \subseteq E_{(4)}$;


    Let $\chi_{j,n} = \xi_{j,n} \circ q \in A^*$. If $a \in A_n \cap U_A(\chi_{1,n},...,\chi_{m,n}; \frac{\pi}{2} )$, then $q(a) \in U_{G_n} (\xi_{1,n},...,\xi_{m,n}; \frac{\pi}{2} ) \subseteq E_{(4)} = q(V \cap A_n)_{(4)}$. From
    Lemma 2.2 we deduce $a \in (V \cap A_n)_{(4)} + \{0,\pm 2n \pm 4n\}^r$. As $\{0,\pm 2n \pm 4n\}^r \subseteq L^0_{(4r)}$ (by our choice $n \in L$), and $L^0_{(4r)} \subseteq V_{(4)}$
    by (5), we get $a \in V_{(8)}$. Therefore


        $A_n \cap U_A (\chi_{1,n},\ldots ,\chi_{m,n}; \frac{\pi}{2}) \subseteq V_{(8)}$

    
    for sufficiently large $n \in L$. By the compactness of $A^*$, there exist $\chi_1,...,\chi_m \in A^*$ and a subsequence $\{n_l\}_l$ of ${n}_{n \in \mathbb{N}_+}$ such
    that $\chi_j (a) = lim_l \chi_{j,n_l}(a)$ for every $j = 1,...,m$ and for every $a \in A$. Using $A = \bigcup{A_{n_l}: l > k, n_l ∈ L}$ for every $k \in {N}_+$ we
    can conclude as in the above proof that $U_A (\chi_1,...,\chi_m; \frac{\pi}{2}) \subseteq V_{(8)}$.


    \textbf{Proof of Lemma 2.3}. Let $g_1,..., g_k \in A$ be such that $A = \bigcup^k_{i=1} (g_i + V )$. Suppose that $G$ is a finitely generated subgroup
    of $A$ containing $g_1,..., g_k$. Then $G = \bigcup^k_{1=1}(g_i + V \cap G)$ and so $V \cap G$ is a $k$-big subset of $G$. By the above corollary there
    exist $\varphi_1,G,...,\varphi_m,G \in G^*$, where $m = k^2$, such that


        $U_G(\varphi_{1,G}, ... , \varphi_{m,G}; \frac{\pi}{2}) \subseteq (V \cap G)_{(8)} \subseteq V_{(8)}$


    One can extend each $\varphi_{iG}$ to a character of $A$, so that we assume from now on $\varphi_{1,G},...,\varphi_{m,G} \in A^*$ and


        $G \cap U_A (\varphi_{1,G}, ... , \varphi_{m,G}; \frac{\pi}{2}) = U_G(\varphi_{1,G}, ... , \varphi_{m,G}; \frac{\pi}{2}) \subseteq V_{(8)}$ (6)


        Let $\mathcal{G}$ be the family of all finitely generated subgroups $G$ of $A$ containing $g_1,..., g_k$. It is a directed set under inclusion. So
    we get $m$ nets ${\varphi_{j,G}}_{G \in \mathcal{G}}$ in $A^*$ for $j = 1,...,m$. By the compactness of $A^*$, there exist subnets $\{\varphi_{j,G_\beta}\}_\beta$ and $\chi_1,...,\chi_m \in A^*$
    such that

    
        $\chi_j(x) = lim_\beta \varphi_{j,G_\beta} (x)$ for every $x \in A and j = 1,...,m.$  (7)


    From (6) and (7) we conclude as before that $U_A(\chi_1,...,\chi_m; \frac{\pi}{2} ) \subseteq V_{(8)}$.


\section{Precompact Groups}


        This section is dedicated to the subgroups of the compact groups. Although all groups in this paper are abelian, let us
    note that Definition 3.1, Example 3.2, Proposition 3.3, as well as Proposition 3.8, Theorem 3.9, work for arbitrary groups.


    \textbf{Definition 3.1} A topological group $G$ is totally bounded if every open non-empty subset $U$ of $G$ is big. A Hausdorff totally
    bounded group is called \emph{precompact}.


    \textbf{Example 3.2} Clearly, compact groups are precompact. Subgroups of precompact groups are precompact. In particular, all
    subgroups of compact groups are precompact.


        Actually, one can prove that a group having a dense precompact subgroup is necessarily precompact. Hence the compact
    groups are precisely the complete precompact groups. Therefore, this gives an external description of the precompact groups
    as subgroups of the compact groups. Here we collect some other properties of this class.
    
    
    \textbf{Proposition 3.3}
        \begin{itemize}

            \item (a) The class of totally bounded (precompact) groups is stable under taking subgroups, direct products and (Hausdorff) quotients.
            
            \item (b)Every topological group $(G, \tau)$ admits a finest totally bounded group topology $\tau^+$ with $\tau^+ \leqslant \tau$. In particular, every group $G$
            admits a finest totally bounded group topology $\mathcal{P}_G$.

        \end{itemize}


\textbf{Proof.} The proof of (a) is straightforward and (b) follows from (a).


    A topological group $(G, \tau)$ is said to be maximally almost periodic, if the topology $\tau^+$ is Hausdorff. The assignment
$(G, \tau) \mapsto (G, \tau^+)$ is a functor from the category of all topological groups to the subcategory of all totally bounded groups. It
sends maximally almost periodic groups to precompact groups.


    If $H$ is a family of characters of an abelian group $G$, then the family $\{U_G(\chi_1,...,\chi_n; \delta): \delta > 0, \chi_i \in H, i = 1,...,n\}$,
taken as a pre-base of the filter of neighborhoods of 0, generates of a group topology $\mathcal{T}_H$ on $G$. This is the coarsest
group topology on $G$ that makes every character from $H$ continuous. One calls $T_{G^*}$ Bohr topology of $G$. It follows from
Example 2.1 that all group topologies of the form $\mathcal{T}_H$ on $G$ are totally bounded. So in particular, $\mathcal{P}_G \geqslant \mathcal{T}_{G^*}$ (see also Corollary 3.6).


    We say that the characters of $H$ separate the points of $G$ iff for every $x \in G$, $x \neq 0$, there exists a character $\chi \in H$ with
$\chi(x) \neq 1$. Clearly, the characters of $H$ separate the points of $G$ iff $\mathcal{T}_H$ is Hausdorff. In such a case, $\mathcal{T}_H$ coincides with
the topology induced by the natural embedding $G \hookrightarrow T^H$. The next proposition summarizes some of the above comments:


\textbf{Proposition 3.4} Let $G$ be an abelian group


    \begin{itemize}
        
        \item (a) All topologies of the form $\mathcal{T}_H$, where $H \leqslant G^*$, are totally bounded    
        
        \item (b) $\mathcal{T}_H$ is precompact iff $H$ separates the points of $G$. In particular, $\mathcal{T}_{G^*}$ is precompact.
        
    \end{itemize}


\textbf{Proof.} (a) follows from Example 2.1. For the second assertion in (b) observe that $G^*$ separates the points of $G$


    It requires a considerable effort to prove that, conversely, every totally bounded group topology has the form $\mathcal{T}_H$ for
some $H$ (see Theorem 5.2). At this stage we can prove it only for $\mathcal{P}_G$. In order to do this we need the following corollary of
Folner's lemma providing also an internal description of the neighborhoods of 0 in the Bohr topology of $A$.


\textbf{Corollary 3.5} For a subset E of an abelian group $A$ the following are equivalent:


    \begin{itemize}
            
        \item (a) $E$ contains $V_{(8)}$ for some big subset $V$ of $A$;
        
        \item (b) for every $n \in \mathbb{N}_+$, $E$ contains $V_{(2n)}$ for some big subset $V$ of $A$;

        \item (c) $E$ is a neighborhood of 0 in the Bohr topology of $A$.

    \end{itemize}


\textbf{Proof.} The implication $(a) \Rightarrow (c)$ follows from Folner's lemma. The implication $(c) \Rightarrow (b)$ follows from Proposition 3.4.


    So far we have only the inequality $\mathcal{P}_G \geqslant \mathcal{T}_{G^*}$ between the finest totally bounded group topology PG and the Bohr
topology $\mathcal{T}_{G^*}$ of an abelian group (the latter is precompact, by item (b) of Proposition 3.4). The equivalence of items (a)
and (c) of Corollary 3.5 proves that these two topologies coincide.


\textbf{Corollary 3.6.} For an abelian group $G$ the Bohr topology $\mathcal{T}_{G^*}$ coincides with the finest precompact group topology $\mathcal{P}_G$.


    The following old problems concerning the group $\mathbb{Z}$ is still open (see Cotlar and Ricabarra [6], Ellis and Keynes [9],
Følner [12], Glasner [14] or Pestov [24, Question 1025]):


\textbf{Question 3.7} Does there exist a big set $V \subseteq \mathbb{Z}$ such that $V - V$ is not a neighborhood of 0 in the Bohr topology of $\mathbb{Z}$?


    It is known that every infinite abelian group $G$ admits a big set with empty interior with respect to the Bohr topology [1]
(more precisely, these authors prove that every totally bounded group has a big subset with empty interior).


    It is easy to see that $\widehat{(G, \tau)} = \widehat{(G, \tau^+)}$ for every topological abelian group $\widehat{(G, \tau)}$. Actually, the group $G$ admits a
“universal” precompact continuous surjective homomorphic image $q : G \to G^+$:


\textbf{Proposition 3.8} For a topological group $(G, \tau)$ denote by $G^+$ the quotient group $G/\bar{\{0\}}^{\tau^+}$of $(G, \tau^+)$. Then $G^+$ is precompact and
every continuous homomorphism $f : G \to P$, where $P$ is a precompact group, factors through the canonical homomorphism $q : G \to G^+$.


\textbf{Proof.} The precompactness of $G^+$ is obvious. If $f : G \to P$ is a continuous homomorphism, where $P$ is a totally bounded
group, then $f$ factors through the identity homomorphism $id_G :(G, \tau) \to (G, \tau^+)$. In case $P$ is precompact, then the continuous
homomorphism $f :(G, \tau^+) \to P$ factors uniquely through the canonical homomorphism $q : G \to G^+ as f (\bar{\{0\}}^{\tau^+}) = 0$.


    It can be easily deduced from of the above proposition, that the assignment $G \mapsto G^+$ is a functor from the category of
all topological groups to the subcategory of all precompact groups, namely the composition of the functor $(G, \tau) \mapsto (G, \tau^+)$
and the canonical $T_0$-reflector $G \to G/\bar{\{0\}}$.


    Letting $bG$ denote the compact completion of $G^+$ we obtain:


\textbf{Theorem 3.9.} Every topological group $G$ admits a compact group $bG$ and a continuous homomorphism $b_G : G \to bG$ of $G$ with dense
image, such that for every continuous homomorphism $f : G \to K$ into a compact group K there exists a (unique) continuous
homomorphism $f' : bG \to K$ with $f' \circ b_G = f $.


    The compact group $bG$ and the homomorphism $b_G : G \to bG$ from the above theorem are called Bohr compactification
of the topological group $G$. Clearly, the assignment $G \mapsto bG$ is a functor from the category of all topological groups to
the subcategory of all compact groups. In some sense the Bohr compactification of a topological group $G$ is the compact
group $bG$ that best approximates $G$ in the sense of Theorem 3.9.
    

    The terms Bohr topology and Bohr compactification have been chosen as a reward to Harald Bohr for his work [3]
on almost periodic functions closely related to the Bohr compactification (see Theorems 6.6 and 6.9). Otherwise, Bohr
compactification is due to $A$. Weil. More general results were obtained later by Holm [17] and Prodanov [26].


\section{Prodanov's lemma and a first applications to independence of characters}


\subsection{Prodanov's lemma}


    The next lemma is due to Prodanov [27]. In [8, Lemma 1.4.1] it is proved for abelian groups $G$ that carry a topology $\tau$
such that for every $g \in G$ and $n \in \mathbb{Z}$ the functions $x \mapsto x + g$ and $x \mapsto nx$ are continuous in $(G, \tau)$. The fact that this topology
is not assumed to be Hausdorff will be crucial in the applications of the lemma.


\textbf{Lemma 4.1} (Prodanov's lemma). Let $G$ be a topological abelian group, let $U$ be an open subset of $G$, $f$ a complex-valued continuous
function on $U$ and $M$ a convex closed subset of $\mathbb{C}$. Let $k \in \mathbb{N}_+$ and $\chi_1,...,\chi_k \in G^*$. Suppose that $c_1,..., c_k \in C$ are such that
$\sum^k_{j=1} c_j \chi_j(x) - f(x) \in M$ for every $x \in U$. If $\{\chi_1,...,\chi_n\}=\{\chi_1,...,\chi_k\} \cap \widehat{G}$, with $n \leqslant k$, then $\sum^n_{i=1} c_i \chi_i(x) - f (x) \in M$ for every
$x \in U$.


\textbf{Proof.} If all characters $\chi_1,...,\chi_n$ are continuous there is nothing to prove. Assume that $\chi_k \in G^*$ is discontinuous. Then
there exists a net ${x_\gamma}_\gamma$ in $G$ such that $lim_\gamma x_\gamma = 0$ and there exist $y_j = lim_\gamma \chi_j (x_\gamma)$ (so $|y_j| = 1$) for all $j = 1,...,k$, but
$y_k \neq 1$. Moreover, $y_j = lim_\gamma \chi_j (x_\gamma )= 1$ when $\chi_j$ is continuous.


    Consider $\sum^k_{j=1} c_j \chi_j (x + tx_\gamma) - f (x + tx_\gamma)$, where $t \on \mathbb{Z}$. Since $lim_\gamma x_\gamma = 0$, we have $x + tx_\gamma \in U$ for every $x \in U$ and
for every sufficiently large \gamma. Thus $\sum^k_{j=1} c_j \chi_j (x) \chi_j(x\gamma)^t - f(x + tx_\gamma) \in M$ and so passing to the limit $\sum^k_{j=1} c_j \chi_j (x) \chi_j(x)y^t_j - f(x) \in M$
because $f$ is continuous and $M$ is closed.


    Take an arbitrary $n \in \mathbb{N}$. By the convexity of $M$, for $t = 0,...,n$, we obtain


    $\frac{1}{n + 1} \sum^n_{t=0} (\sum^k_{j=1} c_j \chi_j (x)y^t_j - f (x)) \in M$.


    Then $\sum^n_{t=0} y^t_k = \frac{y^{n+1}_k -1}{y_k-1}$ because $y_k \neq 1$. Hence $\sum_{j=1}^{k-1} c_{jn} \chi_j(x) + \frac{c_k}{1+n} \frac{1-y^{n+1}_k}{1-y_k} \chi_k(x) - f (x) \in M$ for all $x \in U$, where 
$c_{jn} = \frac{\sum^{n}_{t=0} c_j y^t_j}{n+1}$. Now for every $j = 1, 2,...,k - 1$


\begin{itemize}

    \item $|c_{jn}| \leqslant |c_j| \frac{\sum^{n}_{t=0} |y_k|^t}{n+1} = |c_j|$ (because $|y_j| = 1$), and

    \item if $y_j = 1$ then $c_{jn} = c_j$.

\end{itemize}


By the boundedness of the sequences $\{c_{jn}\}^{\infty}_{n=1}$ for $j = 1,...,k - 1$, there exists a subsequence $\{n_m\}^{\infty}_{m=1}$ such that all limits
$c'_j = lim_m c_{jn_m}$ exist for $j = 1,...,k - 1$. On the other hand, $|yk| = 1$, so

    $\lim_{n} \frac{c_k}{n+1}, \frac{1-y^{n+1}_k}{1-y_k} = 0$.


Taking the limit for $m \to \infty$ in


    $\sum_{j=1}^{k-1} c_{jn_m} \chi_j(x) + \frac{c_k}{1+n_m} \times \frac{1-y^{n_m + 1}_{k}}{1 - y_k} \chi_k (x) - f(x) \in M$


gives


    $\sum_{j=1}^{k-1} c'_j \chi_j (x) - f(x) \in M$ for $x \in U$;


moreover $c'_j = c_j$ for every $j = 1,...,k - 1$, such that $\chi_j$ is continuous


    The condition (1) is obtained by the hypothesis, removing the discontinuous character $\chi_k$ in such a way that the coefficients
of the continuous characters remain the same. Iterating this procedure, we can remove all discontinuous characters
among $\chi_1,...,\chi_k$ leaving unchanged the coefficients of the continuous ones.


    This lemma allows to “produce continuity out of nothing” in the process of approximation.


\textbf{Corollary 4.2} Let $G$ be a topological abelian group, $f \in C(G)$ and $\varepsilon > 0$. If $||\sum^{k}_{j=1} c_j \chi_j - f || \leqslant \varepsilon$ for some $k \in \mathbb{N}_+, \chi_1,...,\chi_k \in G^*$
and $c_1,..., c_k \in \mathbb{C}$, then also $|| \sum^{s}_{i=1} c_{m_i} \chi_{m_i} - f || \leqslant \varepsilon$, where $\{\chi_{m_1} ,...,\chi_{m_s} \} = \{\chi_1,...,\chi_n\} \cap \widehat{G}$, with $m_1 < ... < m_s$.


    In particular, if $f = \sum^{k}_{j=1} c_j \chi_j$ for some $k \in \mathbb{N}_+, \chi_1,...,\chi_k \in G^*$ and $c_1,..., c_k \in \mathbb{C}$, then also $f = \sum^{s}_{i=1} c_{m_i} \chi_{m_i}$ where
$\{\chi_{m_1} ,...,\chi_{m_s}\}$ are the continuous characters in the linear combination. In other words, C(G) ∩ X(Gd) coincides with the
$\mathbb{C}$-subalgebra $\mathfrak{X}(G$) of $B(G)$ generated by $\widehat{G}$. For further use in the sequel we isolate also the following equality
$C(G) \cap \mathfrak{X}(G_d) = \mathfrak{X}(G)$, i.e.,


\textbf{Corollary 4.3}. $C(G) \cap \mathfrak{A}(G_d) = \mathfrak{A}(G)$ for every topological abelian group $G$.


In other words, as far as continuous functions are concerned, in the definition of $\mathfrak{A}(G)$ it is irrelevant whether one
approximates via (linear combinations of) continuous or discontinuous characters.


\subsection{4.2. Local forms of Stone-Weierstrab theorem}


\textbf{Theorem 4.4.} (Stone–Weierstraß theorem). Let $X$ be a compact topological space. A $\mathbb{C}$-subalgebra $\mathcal{A}$ of $C(X)$ containing all constants
and closed under conjugation is dense in $C(X)$ for the norm $|| ||$ if and only if $\mathcal{A}$ separates the points of $X$.


    We shall need in the sequel the following local form of Stone-Weierstrab theorem.


\textbf{Corollary 4.5.} Let $X$ be a compact topological space and $f \in C(X)$. Then $f$ can be uniformly approximated by a $\mathbb{C}$-subalgebra $\mathcal{A}$
of $C(X)$ containing all constants and closed under the complex conjugation if and only if $\mathcal{A}$ separates the points of $X$ separated by
$f \in C(X)$.


\textbf{Proof.} Denote by $G : X \to \mathbb{C}^{\mathcal{A}}$ the diagonal map of the family $\{g: g \in \mathcal{A}\}$. Then $Y = G(X)$ is a compact subspace of $\mathbb{C}^{\mathcal{A}}$ and
by the compactness of $X$, its subspace topology coincides with the quotient topology of the map $G : X \to Y$. The equivalence
relation $\sim$ in $X$ determined by this quotient is as follows: $x ~ y$ for $x, y \in X$ if and only if $G(x) = G(y)$ (if and only if
$g(x) = g(y)$ for every $g \in \mathcal{A}$). Clearly, every continuous function $h : X \to \mathbb{C}$, such that $h(x) = h(y)$ for every pair $x, y$ with
$x \sim y$, can be factorized as $h = \bar{h} \circ G$, where $\bar{h} \in C(Y)$. In particular, this holds true for all $g \in A$ and for $f$ (for the latter case
this follows from our hypothesis). Let $\bar{A}$ be the $\mathbb{C}$-subalgebra $\{h: h \in \mathcal{A}\}$ of $\mathcal{C}(Y, \mathbb{C})$. It is closed under complex conjugation
and contains all constants. Moreover, it separates the points of $Y$. (If $y \neq y'$ in $Y$ with $y = G(x)$, $y' = G(x')$, $x, x' \in X$, then
$x \nsim x'$. So there exists $h \in \mathcal{A}$ with $\bar{h}(y) = h(x) \neq h(x') = \bar{h}(y')$. Hence $\bar{\mathcal{A}}$ separates the points of $Y$ .) Hence we can apply
Stone-Weierstrab Theorem 4.4 to $Y$ and $\bar{\mathcal{A}}$ to deduce that we can uniformly approximate the function $\bar{f}$ by functions of $\bar{\mathcal{A}}$.
This produces uniform approximation of the function $f$ by functions of $\bar{\mathcal{A}}$. 


    Now we give an apparently topology-free local form of the Stone-Weierstrab theorem for subsets of an abelian group.


\textbf{Proposition 4.6.}Let $G$ be an abelian group and $H$ be a group of characters of $G$. If $X$ is a subset of $G$ and $f$ is a complex-valued
bounded function on $X$ then the following conditions are equivalent:

    \begin{itemize}

        \item (a) $f$ can be uniformly approximated on $X$ by a linear combination of elements of $H$ with complex coefficients;

        \item (b) for every $\varepsilon > 0$ there exist $\delta > 0$ and $\chi_1,...,\chi_m \in H$ such that $x - y \in U_G (\chi_1,...,\chi_m; \delta)$ yields $|f (x) - f (y)| < \varepsilon$ for every
            $x, y \in X$.

    \end{itemize}

\textbf{Proof} (a) $\Rangle$ (b) Let $\varepsilon > 0$. By (a) there exist $c_1,...,c_m \in \mathbb{C}$ and $\chi_1,...,\chi_m \in H$ such that $|| \sum^{m}_{i=1} c_i \chi_i - f || < \frac{\epsilon}{4}$, that is
$|\sum^{m}_{i=1} c_i \chi_i (x) -f(x)| < \frac{\epsilon}{4}$ for every $x \in X$.


    On the other hand note that $|\sum^{m}_{i=1} c_i \chi_i (x) - \sum^{m}_{i=1} c_1 \chi_1 (y)| \leqslant \sum^{m}_{i=1} |c_i| \times |\chi_i (x) - \chi_i (y)|$ and that
    $|\chi_i (x-y)-1| = |\chi_i (x) \chi_i (y)^{-1}-1|=|\chi_1 (x) - \chi_i(y)|$. If we take


        $\delta = \frac{\varepsilon}{2m \max_{i=1,...,m} |c_i|}$


then $x - y \in U(\chi_1, \dots, \chi_m; \delta)$ implies $\sum^m_{i=1} |c_i| * |\chi_i (x) - \chi_i (y)| , \frac{\varepsilon}{2}$ and so also $|\sum^m_{i=1} c_i \chi_i (x) - \sum^m_{i=1} c_i \chi_i (x)| < \frac{\varepsilon}{2}$. 
Consequently,


    $|f(x) - f(y)| \leqslant |f(x) - \sum^m_{i=1} c_i \chi_i (x)| + |\sum^m_{i=1} c_i \chi_i (x) - \sum^m_{i=1} c_i \chi_i (y)| + |\sum^m_{i=1} c_i \chi_i (y) - f(y)| < \varepsilon$


$(b) \Longrightarrow (a)$ Let $\beta (X)$ be the Cech-Stone compactification of $X$ endowed with the discrete topology. If $F : X \to \mathbb{C}$ is bounded,
there exists a unique continuous extension $F^{\beta}$ of $F$ to $\beta X$. Let $\mathcal{S}$ be the collection of all continuous functions $g$ on $\beta X$
such that $g = \sum^n_{j=1} c_j \chi^\beta_j$ with $\chi_j \in H$, $c_j \in \mathbb{C}$ and $n \in \mathbb{N}_+$. Then $\mathbb{S}$ is a subalgebra of $\mathcal{C} (\beta X, \mathbb{C})$ closed under conjugation
and contains all constants. In fact in $\mathcal{S}$ we have $\chi_k^\beta \chi_j^\beta = (\chi_k \chi_j)^\beta$ by definition and $\bar{\chi^\beta} = (\bar{\chi})^\beta$ because $\chi \bar{\chi} = 1$ and so
$(\chi \bar{\chi})^\beta = \chi^\beta(\bar{\chi})^\beta$, that is $(\bar{\chi})^\beta = (\chi^{-1})^\beta = \bar{\chi^\beta} $.


    Now we will see that $\mathbb{S}$ separates the points of $\beta X$ separated by $f^\beta$, in order to apply the local Stone-Weierstrab
Corollary 4.5. Let $x, y \in \beta X$ and $f^\beta (x) \neq f \beta (y)$. Consider two nets ${x_i}_i$ and ${y_i}_i$ in $X$ such that $x_i \to x$ and $y_i \to y$. Since
$f^\beta$ is continuous, we have $f^\beta (x) = lim f (x_i)$ and $f^\beta (y) = lim f (y_i)$. Along with $f^\beta (x) \neq f^\beta (y)$ this implies that there exists
$\varepsilon > 0$ such that $|f (x_i) - f (y_i)| \geqslant \varepsilon$ for every sufficiently large $i$. By the hypothesis there exist $\delta > 0$ and $\chi_1,...,\chi_k \in H$
such that for every $u, v \in X$ if $u - v \in U_G (\chi_1, \dots ,\chi_k; \delta)$ then $| f (u) - f (v)| < \varepsilon$. Assume $\chi^\beta_j (x) = \chi^\beta_j (y)$ holds true for every
$j = 1, \dots ,k$. Then $x_i - y_i \in U_G(\chi_1,...,\chi_k; \delta)$ for every sufficiently large $i$, and this contradicts (a). So each pair of points
of $\beta X$ separated by $f \beta$ is also separated by $\mathcal{S}$. Since $\beta X$ is compact, one can apply the local version of the Stone-WeierstraB
Corollary 4.5 to $\mathbb{S}$ and $f^\beta$ and so $f^\beta$ can be uniformly approximated by $\mathcal{S}$. To conclude note that if $g = \sum c_j\chi^\beta_j$ on $\beta X$
then $g \upharpoonright_X = \sum c_j \chi_j$.


    The reader familiar with uniform spaces will note that item (b) is nothing else but $T_H$ -uniform continuity of $f$.


    The use of the Cech-Stone compactification in the above proof is inspired by Nobeling and Beyer [23] who proved that
if $S$ is a subalgebra of $B(X)$, for some set $X$, containing the constants and stable under conjugation, then $g \in B(X)$ belongs
to the closure of $S$ with respect to the norm topology if and only if for every net $(x_\alpha)$ in $X$ the net $g(x_\alpha)$ is convergent
whenever the nets $f(x_\alpha)$ are convergent for all $f \in S$.


\subsection{4.3. Independence of characters}


    We apply now Prodanov's lemma for indiscrete $G$. Note that for $U$ and $f$ as in that lemma, this yields $U = G$ and $f$ is
necessarily a constant function.


\textbf{Corollary 4.7}. Let $G$ be an abelian group, $g \in \mathfrak{X}_0 (G)$ and M be a closed convex subset of $\mathbb{C}$. If $g(x) - c \in M$ for some $c \in \mathbb{C}$ and for
every $x \in G$, then $c \in M$.


\emph{Proof.} Assume that $g(x) = \sum^k_{j=1} c_j \chi_j(x)$ for some $c_1,..., c_k \in \mathbb{C}$ and non-constant $\chi_1,...,\chi_k \in G^*$. Apply Lemma 4.1 with $G$
indiscrete, $U = G$ and $f$ the constant function $c$. Since all characters $\chi_1,...,\chi_k$ are discontinuous, we conclude $c \in M$ with
Lemma 4.1. 


    Taking as $M$ the closed disk with center 0 and radius $\varepsilon > 0$ we obtain the following corollary:


\textbf{Corollary 4.8.} Let $G$ be an abelian group, $g \in \mathfrak{X}_0 (G)$ and $\varepsilon > 0$. If $\|g(x) + c\| \leqslant  \varepsilon$ for some $c \in \mathbb{C}$, then $|c| \leqslant \varepsilon$.


    Using this corollary we shall see now that for an abelian group $G$ the characters $G^*$ not only span $\mathfrak{X}(G)$ as a base, but
they have a much stronger independence property (see Corollary 4.11 and the preceding text).


\textbf{Corollary 4.9.} If $G$ is an abelian group and $\chi_0,\chi_1,...,\chi_k \in G^*$ are distinct characters, then $\|\chi_0 - \sum^k_{j=1} c_j \chi_j\| \leqslant 1$ for every
$c_1, \dots, c_k \in \mathbb{C}$.


\textbf{Proof.} Let $\varepsilon = \| \sum^k_{j=1} c_j \chi_j - \chi_0 \|$. By our assumption $\xi_j = \chi_j \chi^{-1}_0$ is non-constant for every $j = 1, 2, \dots, n$. So our assumption
and the choice of $\varepsilon$ yield $g = \sum^m_{j=1} c_j \xi_j \in \mathfrak{X}_0$ and $\| g(x) - 1 \| = \| \sum^m_{j=1} c_j \chi_j \chi^{-1}_0 - 1\| \leqslant \varepsilon$. Then $|1| \leqslant \epsilon$ by the previous
corollary. 


    Clearly, $\frac{1}{2}$ can be replaced by any constant $< 1$ in the next corollary.


\textbf{Corollary 4.10.} Let $G$ be an abelian group and $\chi \in G^*$ such that there exist $k \in \mathbb{N}_+, \chi_1,...,\chi_k \in H$ and $c_1,..., c_k \in \mathbb{C}$ such that
$\| \sum^k_{j=1} c_j \chi_j (x) - \chi(x)\|  \frac{1}{2}$. Then $\chi = \chi_i$ for some $i$.


    This corollary implies, in particular, that $G^*$ is a base of the $\mathbb{C}$-linear space $\maathfrak{X}(G)$. Actually, a much stronger property
holds: for every $\chi \in G^*$ the $\mathbb{C}$-hyperspace of $\mathfrak{X}(G)$ spanned by $G^* \ {χ}$ is closed. In particular:


\textbf{Corollary 4.11.}


\textbf{Corollary 4.12.}


\textbf{Proof.}


\section{Peter-Weyl's theorem and Pontryagin-van Kampen duality theorem}


    Here we give further applications of Prodanov's lemma towards obtaining an elementary proof of Folner's theorem and
consequently Peter-Weyl's theorem, Comfort and Ross's theorem.


\subsection{5.1 Folner's theorem}


    Again, this theorem can be proved for abelian groups $G$ endowed with a topology such that for every $g \in G$ and every
$n \in Z$, the functions $x \mapsto g + x$ and $x \mapsto nx$ on $G$ are continuous [8, Theorem 1.4.3].


\textbf{Theorem 5.1} (Folner's Theorem). Let $G$ be a topological abelian group. If $k$ is a positive integer and $E$ is a $k$-big subset of $G$, then for
every neighborhood $U$ of 0 in $G$ there exist $\chi_1,...,\chi_m \in G$, where $m = k^2$, and $\delta > 0$ such that


    $U_G(\chi_1,...,\chi_m; \delta) \subseteq U - U + E_{(8)}$.  (1)


\textbf{Proof.} By Lemma 2.3, there exist $\varphi_1,...,\varphi_m \in G^*$ such that $U_G(\varphi_1,...,\varphi_m; \frac{\pi}{2} ) \subseteq E_{(8)}$, where the characters $\varphi_j$ can be
discontinuous. Using Prodanov's lemma, we intend to replace these characters by continuous ones “enlarging” $E_{(8)}$ to
$U - U + E_{(8)}$.


    Clearly, $C := \bar{E_{(8)} + U }\subseteq E_{(8)} + U - U$. Consider the open set $X = U \cup (G \ C)$ and the continuous function $f : X \to C$
defined by


    $f(x) = \{^{0 \text{ if } x \in U}_{1 \text{ if } x \in G / C} $


    Let $H$ be the subgroup of $G^*$ generated by $\varphi_1,...,\varphi_m$. Take $x, y \in X$ with $x - y \in U_G(\varphi_1,...,\varphi_m; \frac{\pi}{2} ) \subseteq E_{(8)}$. So if $y \in U$ then
$x \in E_{(8)} + U$ and consequently $x \notin G / E_{(8)} + U$, that is $x \in U$. Analogously, $x \in U$ yields $y \in U$. Thus $f(x) = f(y)$ whenever
$x - y \in U_G(\varphi_1,...,\varphi_m; \frac{\pi}{2} ) \subseteq E_{(8)}$. So by Proposition 4.6 one can uniformly approximate $f$ on $X$ by characters from $H$. Hence
one can find products $\xi_J = \varphi^{j1}_1 \cdot \dots \cdot \varphi^{jm}_m$, with $\tilde{j} = (j_1,..., j_m) \in \mathbb{Z}^m$ and $c_{\tilde{j}} \in \mathbb{C}$ such that $|\sum_{\tilde{J}} c_{\tilde{J}} \xi_{\tilde{J}}(x) - f(x)| \leqslant \frac{1}{3}$ for every
$x \in X$. Letting $x = 0$ gives $|\sum_{\tilde{J}} c_{\tilde{J}}|  \frac{1}{3}$, and consequently,


    $\frac{2}{3} \leqslant |\sum_{\tilde{J}} c_{\tilde{J}} - 1|$


    Since $X$ is open and $f$ is continuous, applying Lemma 4.1 to the convex closed set $M = \{z \in \mathbb{C}: |z|  \frac{1}{3}\}$, we may assume
that all characters $\xi_{\tilde{J}}$ are continuous.


Since the group $H$ is $m$-generated, its subgroup $\Phi = H \cap \hat{G}$ of $H$ has at most $m$ generators $\chi_1,...,\chi_m \in \Phi$. Thus,
$\xi_{\tilde{J}} = \chi^{s_1(\tilde{J})}_1 \cdot ... \cdot \chi^{s_m(\tilde{J})}_m$ with appropriate $s_1(\tilde{J}), . . . , s_m(\tilde{J}) \in \mathbb{Z}$. Choose $\varepsilon > 0$ with $\varepsilon \sum_\tilde{J} |c_\tilde{J}| < \frac{1}{3}$. There exists $\delta > 0$ such that
$|\xi_{\tilde{J}}(x) - 1| \leqslant \varepsilon$ for all summands $\xi_{\tilde{J}}$ whenever $x \in U_G (\chi_1,...,\chi_m; \delta)$. We claim that (1) holds for this $\delta$. Indeed, assume that
there exists $x \in U_G (\chi_1,...,\chi_m; \delta) \ (U - U + E_{(8)})$. Then $x \in G \ C$, and the definition of $f$ gives $f(x) = 1$, so


$|\sum_{\tilde{J}} c_{\tilde{J}} - 1| \leqslant |\sum_{\tilde{J}} c_{\tilde{J}}(1 - \xi_{\tilde{J}}(x))| + |\sum_{\tilde{J}} c_{\tilde{J}} \xi_{\tilde{J}}(x) - f(x)| \leqslant \epsilon \sum_{\tilde{J}} |c_{\tilde{J}}| + \frac{1}{3} < \frac{2}{3}$


which contradicts (2). This proves (1). 


    In Folner's proof [12], $\pi /2$ stays in place of $\delta > 0$. He makes recourse to mean values and a theorem of Godeman on
positively defined functions. Cotlar and Ricabarra [6] use mean values and Krein-Mil'man's theorem to obtain a similar
result: for an abelian topological group $G$ and $x \in G / \{0\}$ there exists $\chi \in G$ with $\chi(x) \neq 1$ if and only if there exists a big
neighborhood $U$ of 0 such that $x \notin  U_{(6)}$.


\subsection{5.2. Peter-Weyl's theorem and precompact group topologies on abelian groups}


    In this subsection we prove Peter-Weyl's theorem using Folner's theorem. Moreover, we use Prodanov's lemma (via
Corollary 4.12) to prove Comfort-Ross' theorem [5] describing the precompact topologies of the abelian groups.
The next theorem characterizes the precompact topologies on abelian groups.


\textbf{Theorem 5.2}. Let $(G, \tau)$ be an abelian group. The following conditions are equivalent:

\begin{itemize}

    (a) $\tau$ is precompact;

    (b) there exists a group $H$ of continuous characters of $G$ that separates the points of $G$ and such that $\tau = \mathcal{T}_H$.

\end{itemize}

\textbf{Proof.} $(b) \Rightarrow (a)$ was already proved in Proposition 3.4.


$(a) \Rightarrow (b)$ If $H = \hat{(G, \tau)}$ then $\mathcal{T}_H \subseteq τ$. Let $U$ and $V$ be open neighborhoods of 0 in $(G, \tau)$ such that $V_{(10)} \subseteq U$. Then $V$ is
big and by Folner's Theorem 5.1 there exist continuous characters $\chi_1,...,\chi_m$ of $G$ such that $U_G(\chi_1,...,\chi_m; \delta) \subseteq V_{(10)} \subseteq U$
for some $\delta > 0$. Thus $U \in \mathcal{T}_H$ and $τ \subseteq \mathcal{T}_H$.


\textbf{Corollary 5.3} (Peter-Weyl's theorem). If $G$ is a compact abelian group, then $G$ separates the points of $G$.


\textbf{Proof.} Let $\tau$ be the topology of $G$. By Theorem 5.2 there exists a group $H$ of continuous characters of $G$ (i.e., $H \subseteq \hat{G}$) such
that $\tau = \mathcal{T}_H$. Since $H \subseteq \hat{G}$ we conclude that $G$ separates the points of $G$. 


\textbf{Corollary 5.4.} If $G$ is a compact abelian group, then $G$ is isomorphic to a (closed) subgroup of the power $\mathbb{T}^{\hat{G}}$.


\textbf{Proof.} Since the characters $\chi \in G$ separate the points of $G$, the diagonal map determined by all characters defines a continuous
injective homomorphism $G \hookrightarrow \mathbb{T}^{\hat{G}}$. By the compactness of G and the open mapping theorem, this is the required
embedding. 


\textbf{Corollary 5.5.} If $G$ is compact, then $C(G) \subseteq \mathfrak{A}(G)$.


\textbf{Proof.} Stone-WeierstraB theorem and Peter-Weyl's theorem imply that every $f ∈ C(G)$ can be uniformly approximated by
linear combinations, with complex coefficients, of characters of $G$ when $G$ is compact, i.e. $C(G) \subseteq \mathfrak{A}(G)$. 


    In the sequel $w(G)$ and $\chi(G)$ will denote the weight and the character, respectively, of a topological group $G$.


\textbf{Theorem 5.6} (Comfort and Ross). Let $G$ be an abelian group. Then the map $H \mapsto^{T} \mathcal{T}_H$ is an order preserving bijection between subgroups
$H \leqslant G^*$ and totally bounded group topologies on $G$. Moreover, $w(G,\mathcal{T}_H) = \chi(G,\mathcal{T}_H) = |H|$. In particular, $\mathcal{T}_H$ is metrizable iff $H$ is
countable.


\textbf{Proof.} Theorem 5.2 yields that $\mathcal{T}_H$ is totally bounded and $T$ is surjective. By Corollary 4.12, $\mathcal{T}_{H_1} = \mathcal{T}_{H_2}$ for $H_1, H_2 \in \mathcal{H}$ yields
$H_1 = H_2$. Therefore, $T$ is a bijection.


    Since $\mathcal{T}_H$ is the initial topology of $G$ with respect to the diagonal map $G \to \mathbb{T}^H$ , $\chi(G,\mathcal{T}_H ) \leqslant w(G,\mathcal{T}_H) \leqslant w(\mathbb{T}^H ) = |H|$. To
check $\kappa := \chi(G,\mathcal{T}_H ) \geqslant |H|$, pick a base $\mathcal{B}$ of the neighborhoods at 0 of $\mathcal{T}_H$ of size $\kappa$. By the definition of $\mathcal{T}_H$, every element
$B \in \mathcal{B}$ can be written as $B = U_G(\chi_{1,B} ,...,\chi_{n_B,B} ; 1/m_B )$, where $n_B ,m_B \in \mathbb{N}$ and $\chi_{i,B} \in H$ for $i = 1,...,n_B$ . Then the subset
$H' = \{\chi_{i,B} : B \in B, i = 1,...,n_B \}$ of $H$ has $|H'| \leqslant \kappa$ and $\mathcal{T}_{(H')} = \mathcal{T}_H$. So, $\langle H' \rangle = H$. Hence $|H|=|H'| \leqslant \kappa$. 


\subsection{5.3. Pontryagin-van Kampen duality theorem and structure of the locally compact abelian groups}


    Here we give a complete proof of Pontryagin-van Kampen duality theorem [25,18] in the case of discrete and compact
groups (see Remarks 5.10 and 5.14 for the general case). To this end we need the following immediate corollary of
Theorem 5.6.


\textbf{Corollary 5.7.} If $(G, \tau)$ is a compact abelian group and $H \leqslant \hat{G}$ separates the points of $G$, then $H = \hat{G}$.


\textbf{Proof.} $\tau = \mathcal{T}_{\hat{G}}$ by Theorem 5.2. Since $\mathcal{T}_H \subseteq \mathcal{T}_{\hat{G}}}$ and $\mathcal{T}_H$ is Hausdorff by Theorem 5.6, then $\mathcal{T}_H = \mathcal{T}_{\hat{G}}$ by the compactness
of $\mathcal{T}_{\hat{G}}$ . Now again Theorem 5.6 yields $H = \hat{G}$. 


\textbf{Theorem 5.8}. If $G$ is a discrete or compact abelian group then $\omega_G : G \to \hat{\hat{G}}$ is an isomorphism.


\textbf{Proof}. First, notice that for both discrete and compact $G$ the group $\hat{G}$ separates the points of $G$ - in the compact case this
follows from Peter-Weyl's theorem. Thus, for any $x \neq 0$ in $G$ there exists $\chi \in \hat{G}$ with $\chi(x) \neq 1$, so $\omega_G(x)(\chi) = \chi(x) \neq 1$, so
$\omega_G(x) \neq 1$. Thus, $\omega_G$ is injective for both discrete and compact $G$.


    If $G$ is discrete, then for the compact group $K = \hat{G}$ the subgroup $\omega_G(G)$ of $K = \hat{\hat{G}}$ separates the points of $K$, so by the
above corollary $\omega_G(G) = \hat{\hat{G}}$. Thus, $\omega_G$ is an isomorphism in this case.


    Now assume that $G$ is compact. Since in this case the compact-open topology coincides with the pointwise convergence
one, it suffices to note that for a fixed element $\chi \in G$ and an open neighborhood $U$ of 1 in $\mathbb{T}$ the set $W = \chi^{-1} (U)$ is an
open neighborhood of 0 in $G$ and $\omega_G(W ) \subseteq W_{\hat{G}}(\chi, U)$. This proves that $\omega_G$ is continuous. By the compactness of $G$ the
subgroup $K = \omega_G(G)$ of $\hat{\hat{G}}$ is closed and $\omega_G : G \to K$ is a topological isomorphism. So, it suffices to see that $K = \hat{\hat{G}}$. Suppose
for a contradiction that $K \neq \hat{\hat{G}}$. By Peter-Weyl's theorem applied to the compact group $\hat{\hat{G}}/K$, there exists a non-zero $\xi \in \hat{\hat{\hat{G}}}$
such that $\xi(K) = 0$. Since $\hat{G}$ is discrete, by the case considered above $\omega_G : \hat{G} \to \hat{\hat{\hat{G}}}$ is an isomorphism. Thus there exists $\chi \in \hat{G}$
such that $\xi = \omega_{\hat{G}}(\chi)$. Now for every $x \in G$ one has $\omega_G(x) \in K$, thus


    $0 = \xi(\omega_G(x)) = \omega_{\hat{G}}(\chi)(\omega_G(x)) = \omega_G(x)(\chi) = \chi(x)$.


    Hence $\chi = 0$ and consequently, $\xi = 0$, a contradiction. 


    The following corollary complements the properties of the order preserving bijection $H \mapsto^T \mathcal{T}_H$ described in Theorem 5.6.
By Theorem 5.2, $\mathcal{T}_H$ is Hausdorff iff $H$ separates the points of $G$.


    \textbf{Corollary 5.9.} Let $G$ be an abelian group. Then a subgroup $H \leqslant G^*$ separates the points of $G$ iff $H$ is dense in $G^*$.


    \textbf{Proof.} Note that $\bar{H} \neq G^*$ if and only if the quotient group $G^* / \bar{H}$ is non-trivial. By Corollary 5.3, this is equivalent to the
existence of $\xi \in \hat{G^*} / \{0\}$ with $\xi \upharpoonright H = 0$. As $\hat{G^*} = \omega_G (G)$ by Theorem 5.8, this is equivalent to the existence of $x \in G / \{ 0 \}$
such that $\chi(x) = 0$ for all $\chi \in H$, i.e., $H$ does not separate the points of $G$. 


    The non-symmetry in Theorem 5.8, due to the fact that (the restriction of) the duality functor does not have the same
domain and co-domain has an easy remedy. One can take the larger class $\mathcal{L}_0$ of locally compact abelian groups that have an
open compact subgroup. It contains both discrete and compact groups and for $G \in \mathcal{L}_0$ also $\hat{G} \in \mathcal{L}_0$, i.e., the duality functor
sends $\mathcal{L}_0$ to itself.


    \textbf{Remark 5.10.} One can extend Theorem 5.8 to $\mathcal{L}_0$. To this end one needs the following properties.


    (a) It can be deduced from Corollary 5.7 that if $G \in \mathcal{L}$ and $K$ is a compact subgroup of $G$, then for the inclusion
map $j : K \hookrightarrow G$ the dual homomorphism $\hat{j} : \hat{G} \to \hat{K}$ is surjective, i.e., every continuous character $\chi : K \to \mathbb{T}$ extends to a
continuous character of $G$. Obviously, the same conclusion remains true when $K$ is an open subgroup of $G$.


    (b) The homomorphisms $\omega_G$ provide a natural transformation between the identity of $\mathcal{L}$ and $\hat{\hat{ }}$.
Using (a), (b) and Theorem 5.8, one can prove that $\omega_G : G \to \hat{\hat{G}}$ is an isomorphism for every $G \in \mathcal{L}_0$.


    The next lemma is a necessary step in the proof of Theorem 5.12 that describes, to a certain extent, the structure of the
locally compact abelian groups.


    \textbf{Lemma 5.11} Let $G$ be a locally compact abelian group such that the subgroup $\langle x \rangle$ generated by $x$ is dense in $G$ for some $x \in G$. Then $G$
is either discrete or compact.


    \textbf{Proof.} Let $x \in G$ be such that $\langle x \rangle$ is dense in $G$. If the topology induced on $\langle x \rangle$ is discrete, then $G = \langle x \rangle$ is discrete.
    
    
    Assume that the topology induced on $\langle x \rangle$ by $G$ is not discrete. Then $\langle x \rangle$ is infinite, so it is algebraically isomorphic to $\mathbb{Z}$.
Identifying $\mathbb{Z}$ with $\langle x \rangle$, we may assume that $\mathbb{Z} \subset G$ and $\mathbb{Z}$ is a dense subgroup of $G$.
    

    We will now show that $\mathbb{Z}$ is precompact in the induced topology. First, notice that every non-empty open subset of $\mathbb{Z}$
contains arbitrarily large integers. Indeed, otherwise, there would exist a neighborhood $U$ of 0 in $\mathbb{Z}$ having a maximal
element, and then $U \cap (-U)$ would be finite; a contradiction.
    

    Let $U$ be a compact neighborhood of 0 in $G$. We will show that $U \cap \mathbb{Z}$ is big in $\mathbb{Z}$, i.e. $U \cap \mathbb{Z} + F = \mathbb{Z}$ for some finite
$F \subset \mathbb{Z}$.


    Let $V$ be a neighborhood of 0 in $G$ with $V - V \subset U$. Since $U$ is compact, there exist $g_1, g_2,..., g_m \in G$ with


    $U \subset \bigcup^m_{j=1} (g_j + V)$


By the above remark, for every $j$ the set $g_j + V$ contains arbitrarily large elements of $\mathbb{Z}$; in particular there exists an integer
$n_j > 0$ with $n_j \in g_j + V$. Then $g_j \in n_j - V$ for each $j$, so


    $U \subset \bigcup^m_{j=1}(g_j + V ) \subset \bigcup^m_{j=1} (n_j + V - V ) \subset \bigcup^m_{j=1}(n_j + U)$,


and therefore


    $U \cap \mathbb{Z} \subset \bigcup^m_{j=1}(n_j + U \cap \mathbb{Z}). $    (3)


Setting $N = \max \{n_1,n_2,...,n_m\}$ and $F = \{1, 2,..., N\}$, we will now show that $\mathbb{Z} = U \cap \mathbb{Z} + F$, which will complete the proof
of the lemma. Given $t \in \mathbb{Z}$, since $U \cap \mathbb{Z}$ contains arbitrary large integers, $s_0 = \min \{s \in U \cap \mathbb{Z}: s \geqslant t \}$ is well defined and
$s_0 \geqslant t$. Then (3) implies $s_0 = n_j + u_0$ for some $j = 1,...,m$ and some $u_0 \in U \cap \mathbb{Z}$. Since $u_0 = s_0 - n_j < s_0$ and $u_0 \in U \cap \mathbb{Z}$,
we must have $u_0 < t \leqslant s_0$. Thus, $t = k + u_0$ for some integer $k = 1, 2,...,n_j$, so $t \in U \cap \mathbb{Z} + \{k\} \subset U \cap \mathbb{Z} + F$ . This proves that
$\mathbb{Z} = U \cap \mathbb{Z} + F$ . 


    We say that a topological group $G$ is compactly generated if there exists a compact subset $K$ of $G$ such that $G$ coincides
with the subgroup generated by $K$. If $G \in \mathcal{L}_0$ is compactly generated, then $G$ has a compact-open subgroup $K$. The quotient
$G/K$ is discrete and compactly generated, so finitely generated. Hence, $G/K \cong \mathbb{Z}^n \times F$ for some $n \in \mathbb{N}$ and some finite
abelian group $F$. Hence $G$ has an open compact subgroup $N$ containing $K$ such that $N/K \cong F$ and $G/N \cong \mathbb{Z}^n$, consequently
$G \cong N \times \mathbb{Z}^n$. Now we see that an appropriate counterpart of this property remains true for arbitrary compactly generated
locally compact abelian groups.


\textbf{Theorem 5.12.} Every compactly generated locally compact abelian group has a discrete subgroup $H \cong \mathbb{Z}^n$ for some $n \in \mathbb{N}$ and such
that $G/H$ is compact.


    \textbf{Proof.} Suppose first that there exist $g_1,..., g_m \in G$ such that $G = \bar{\langle g_1,..., g_m \rangle}$. We proceed by induction. For $m = 1$ apply
Lemma 5.11: if $G$ is infinite and discrete take $H = G$ and if $G$ is compact $H = \{0\}$. Suppose now that the property holds
for some $m \geqslant 1$ and $G = \bar{\langle g_1,..., g_{m+1} \rangle}$. If every $\bar{\langle g_i \rangle}$ is compact, then so is $G$ and we can take $H = \{0\}$. If $\langle g_{m+1} \rangle$ is discrete,
consider the canonical projection $\pi : G \to G_1 = G/g_{\langle m+1 \rangle}$. Since $G_1$ has a dense subgroup generated by $m$ elements,
by the inductive hypothesis there exists a discrete subgroup $H_1$ of $G_1$ such that $H_1 \cong \mathbb{Z}^n$ and $G_1/H_1$ is compact. Therefore
$H = \pi^{-1}(H_1)$ is a closed countable subgroup of $G$. Thus $H$ is locally compact and countable, hence discrete. Since
$H/\langle g_{m+1} \rangle \cong \mathbb{Z}^n$, we conclude that $H \cong \mathbbZ^{n+1}$. Now $G/H \cong G_1/H_1$, so $G/H$ is compact.


    Now consider the general case. There exists a compact subset $K$ of $G$ that generates $G$. Replacing $K$ by $(K + U) - (K + U)$
for some compact neighborhood $U$ of 0, we may assume that $K$ is a compact symmetric neighborhood of 0 in $G$. Then $K + K$
is compact too, so there exists a finite $F \subset G$ with $K + K \subset K + F$. The latter inclusion easily implies $G = \langle F \rangle + K$, so $G/\bar{\langle F \rangle}$ 
is compact. By the above argument, there exists a discrete subgroup $H$ of $\bar{\langle F \rangle}$ isomorphic to $\mathbb{Z}^n$ for some $n \in \mathbb{N}$ and such
that $\bar{\langle F \rangle}/H$ is compact. Therefore, $G/H$ is compact as well. 


    One can prove that a general compactly generated locally compact abelian group has the form $N \times \mathbb{Z}^n \times \mathbb{R}^m$, with $n,m \in \mathbb{N}$
and $N$ compact (see [8,16,15,21]).


    Using Peter-Weyl's theorem and Theorem 5.12, one obtains the following counterpart of Peter-Weyl theorem for locally
compact abelian groups:


\textbf{Theorem 5.13.} If $G$ is a locally compact abelian group, then $\hat{G}$ separates the points of $G$.


\textbf{Proof.} Let $x \in G / \{0\}$ and let $V$ be a compact neighborhood of 0 in $G$. Then $G_1 = (V \cup \{x\})$ is an open compactly generated
subgroup of $G$. In particular $G_1$ is locally compact. By Theorem 5.12 there exists a discrete subgroup $H$ of $G_1$ such that
$H \cong \mathbb{Z}^m$ for some $m \in \mathbb{N}$ and $G_1/H$ is compact. In particular, $\bigcap_{n \in \mathbb{N}_+} nH = \{0\}$ and so there exists $n \in \mathbb{N}_+$ such that $x \notin nH$.
Since $H/nH$ is finite, the quotient $G_2 = G_1/nH$ is compact. Consider the canonical projection $\pi : G_1 \to G_2$ and note that
$\pi(x) = y \neq 0$ in $G_2$. By Corollary 5.3 there exists $\xi \in \hat{G}$ such that $\xi(y) \neq 0$. Consequently $\chi = \xi \circ \pi \in \hat{G_1}$ and $\chi(x) \neq 0$.
Since $\mathbb{T}$ is divisible, there exists $\bar{\chi} \in \hat{G}$ such that $\bar{\chi}\upharpoonright_{G_1} = \chi$. Since $G_1$ is an open subgroup of $G$, this extension will be
continuous.


    This theorem ensures one of the main steps in the proof of Pontryagin-van Kampen duality theorem in the general case
of a locally compact abelian group $G$, namely the injectivity of the homomorphism $\omega_G : G \to \hat{\hat{G}}$.


\textbf{Remark 5.14.} The above mentioned injectivity of ωG is necessary to extend Theorem 5.8 to the whole class $\mathcal{L}$. To this end,
beyond (a) and (b) from Remark 5.10, one needs also the following facts [8,16,21]:


    (i) $\omega_G$ is an isomorphism for all elementary locally compact abelian groups $G$ (i.e., $G \cong \mathbb{R}^m \times \mathbb{Z}^n \times \mathbb{T}^d \times F$ , where $n,m,d \in \mathbb{N}$
and $F$ is a finite abelian group).


    (ii) Every compactly generated $G \in \mathcal{L}$ has a compact subgroup $N$ such that $G/N$ is an elementary locally compact abelian
group.


\section{6. Almost periodic functions of the abelian groups}


\subsection{6.1. Two approaches to almost periodicity}


    For a group $G$ a complex-valued function $f \in B(G)$ is almost periodic\footnote[1]{This definition, in the case of $G = R$, is due to Bochner} if the orbit $\{f_a: a \in G\}$ is relatively compact in
the uniform topology of $B(G)$ (i.e., if every sequence (f_{a_n} ) of translations of $f$ has a subsequence that converges uniformly
in $B(G)$). Denote by $A(G)$ the set of all almost periodic functions of the group $G$.


    \textbf{Fact 6.1.} $A(G)$ is a closed $\mathbb{C}$-subalgebra of $B(G)$ closed under the complex conjugation.


    \textbf{Proof.} To check that $A(G)$ is a $\mathbb{C}$-vector subspace take two almost periodic functions $f , g$ of $G$. We have to prove that
$c_1 f + c_2 g$ is an almost periodic function of $G$ for every $c_1, c_2 \in \mathbb{C}$. It suffices to consider the case $c_1 = c_2 = 1$ since $c_1 f$
and $c_2 g$ are obviously almost periodic functions. Then the closures $K_f = \bar{ \{f_a: a \in G \}}$ and $K_g = \bar{{g_a: a \in G}}$ taken in the
uniform topology of $B(G)$ are compact. Hence $K_f + K_g$ is compact as well. Since $(f + g)a = f_a + g_a \in K_f + K_g$ for every
$a \in G$, we conclude that $f + g$ is almost periodic. A similar proof gives $f_g \in A(G)$. The stability of $A(G)$ under the complex
conjugation is obvious.


    To check that $A(G)$ is closed assume that $f \in B(G)$ can be uniformly approximated by almost periodic functions and pick
a sequence $(g^{(m)})$ of almost periodic functions of $G$ such that


    $\|f - g^{(m)}\| \leq 1/m.$ (1)


Then for every sequence $(f_{a_n})$ of translations of $f$ one can inductively define a sequence of subsequences of $(a_n)$ as follows.
For the first one the subsequence $(g^{(1)}_{a_{n_k}})$ of the sequence $(g^{(1)}_{a_n})$ uniformly converges in $B(G)$. Then pick a subsequence $a_{n_{k_s}}$
of ${a_{n_k}}$ such that the subsequence $(g^{(2)}_{a_{n_{k_s}}})$ of the sequence $(g^{(2)}_{a_{n_k}})$ uniformly converges in $B(G)$, etc. Finally take a diagonal
subsequence $a_\nu$ , e.g., $a_1, a_{n_1} ,a_{n_{k_1}},...$ such that each subsequence $a_n, a_{n_k} , a_{n_{k_{s}}}$, etc. of $a_n$ has a tail contained in the
subsequence $a_\nu$. Then for every $m$ the sequence $(g^{(m)}_{a_\nu})$ uniformly converges in $B(G)$. Therefore, by (1) also the sequence $(f_{a_\nu})$
uniformly converges in $B(G)$. 


    One can consider this notion for a topological abelian groups $G$, without involving the topology of $G$. In other words, we
keep using the notation $A(G)$ in this case too, but we intend $A(G) := A(G_d)$.


    We are now going to describe a different (internal) approach to almost periodic functions.


\textbf{Definition 6.2.} Let $G$ be an abelian group, $f : G \to \mathbb{C}$ a function. An element $a \in G$ is called an $\varepsilon$-almost period of $f$ for $\varepsilon > 0$
if $\| f - f_a\| \leq \varepsilon$.


    Let $T (f , \varepsilon) = {a \in G: a \text{ is an } \varepsilon \text{-almost period of } f }$. Then $-T (f , \varepsilon) = T (f , \varepsilon)$ and $T (f , \varepsilon /2) + T (f , \varepsilon /2) \subseteq T (f , \varepsilon)$ for
all $\varepsilon$. This proves of the following:


\textbf{Fact 6.3.} If $G$ is an abelian group and $f : G \to \mathbb{C}$ is a function, then $\{T (f , \varepsilon): \varepsilon > 0\}$ is a filter-base of neighborhoods of 0 in
a group topology $\mathcal{T}_f$ on $G$.


    Now we use the group topology $\mathcal{T}_f$ to find an equivalent description of almost periodicity of $f$ .


\textbf{Proposition 6.4.} Let $G$ be an abelian group. Then for every function $f : G \to \mathbb{C}$ the following are equivalent:


(a) $f$ is almost periodic;


(b) $\mathbb{T}_f$ is totally bounded.


\textbf{Proof.} Clearly, $\mathcal{T}_f$ is totally bounded iff for every $\varepsilon > 0$ the set $T (f , \varepsilon)$ is big, i.e., for every $\varepsilon > 0$ there exist $a_1,...,a_n \in G$
such that $G = \bigcup^n_{k=1} a_k + T (f , \varepsilon)$.


$(a) \to (b)$ Assume that $\mathcal{T}_f$ is not totally bounded. Then there exists some $\varepsilon > 0$ such that $T (f , 2\varepsilon)$ is not big, so
$T (f , \varepsilon) - T (f , \varepsilon) \subseteq T (f , 2\varepsilon)$ is not big either. Hence one can build inductively a sequence $(b_n)$ in $G$ such that the sets
$b_n + T (f , \varepsilon)$ are pairwise disjoint. By the almost periodicity of $f$ the sequence of translates $(f_{b_m}})$ admits a subsequence
that is Cauchy w.r.t. the uniform topology of $B(G)$. In particular, one can find two distinct indexes $n < m$ such that
$\| f_{b_m} - f_{b_n} \|=\| f_{b_m-b_n} - f \|  \varepsilon$, i.e., $b_m - b_n \in T (f , \varepsilon) \subseteq T (f , \varepsilon) - T (f , \varepsilon)$. Hence $(b_m + T (f , \varepsilon)) \cap (b_n + T (f , \varepsilon)) = \emptyset$,
a contradiction.


    $(b) → (a)$ By the completeness of $B(G)$, it suffices to check that every infinite sequence of translates $(f_{b_m})$ of $f$ admits
a subsequence that is Cauchy w.r.t. the uniform topology of $B(G)$.


    Assume for a contradiction that some sequence of translates (fbm ) admits no Cauchy subsequence. That is, for every
subsequence $(f_{bm_k})$ there exists an $\varepsilon > 0$ such that for some subsequence $m_{k_s}$ of $m_k$ one has


    $\| f_{b_{m_{k_s}}} - f_{b_{m_{k_t}}} \| = \| f_{b_{m_{k_s}}-b_{m_{k_t}}} - f \| \geq 2\varepsilon$ for all $s \neq t$. (2)


Since our hypothesis implies that $T(f , \varepsilon/2)$ is big, there exist $a_1,...,a_n \in G$ such that $G = \bigcup^n_{k=1} a_k + T(f , \varepsilon/2)$. Since
infinitely many $b_{m_{k_{s}}}$ will fall in the same $a_k + T (f , \varepsilon/2)$ for some $k$, we deduce that for distinct $s$ and $t$ with $b_{m_{k_{s}}} , b_{m_{k_{t}}} \in a_k + T (f , \varepsilon/2)$
one has


    $b_{m_{k_{s}}} - b_{m_{k_{t}}} \in T (f , \varepsilon/2) - T (f , \varepsilon/2) \subseteq T (f , \varepsilon)$, i.e., $\| f_{b_{m_{k_s}}}-b_{m_{k_{t}}} - f \| \leq \varepsilon$.


This contradicts (2).


    The next example provides a large supply of almost periodic functions.


\textbf{Example 6.5.} Let $G$ be an abelian group.


(a) If $\chi \in G^*$, then $T (\chi, \varepsilon) = \{ a \in G: |\chi(a)-1| < \varepsilon \} \supseteq U_G(\chi; \delta)$ for some $\delta$, hence $T (\chi, \varepsilon)$ is big. Consequently, $\chi$ is almost
periodic.


(b) It follows from (a) and Fact 6.1 that every linear combination of characters is an almost periodic function, i.e.,
$\mathfrak{X}(G) \subseteq A(G)$. Hence, $\mathfrak{A}(G) = \bar{\mathfrak{X}(G)} \subseteq A(G)$ by Fact 6.1.

\subsection{6.2. Around Bohr-von Neumann theorem}


    The main aim of this subsection is to show that a bounded function $f$ on an abelian group $G$ is almost periodic if and
only if $f$ can be uniformly approximated by linear combinations, with complex coefficients, of characters of $G$.


\textbf{Theorem 6.6} (Bohr-von Neumann theorem). $A(G) = \mathfrak{A}(G)$ for every abelian group $G$.


\textbf{Proof.} The inclusion $\mathfrak{A}(G) \subseteq A(G)$ was proved in Example 6.5 (b).
    

    Assume $f \in A(G)$. Fix an $\varepsilon > 0$. By Proposition 6.4 the set $T (f , \varepsilon /8)$ is big. Hence we can apply Corollary 3.5 to the
set $T (f , \varepsilon)$ containing $T (f , \varepsilon/8)_{(8)}$ and find $\chi_1,...,\chi_n \in G^*$, $\delta > 0$ such that $U_G (\chi_1,...,\chi_n; \delta) \subseteq T (f , \varepsilon)$. Now if $x, y \in G$
satisfy $x - y \in U_G (\chi_1,...,\chi_n; \delta)$, then $x - y \in T (f , \varepsilon)$, so $\| f_{x-y} - f \| \leqslant \varepsilon$. In particular, $| f (x) - f (y)| \leqslant \varepsilon$. Then $f$ satisfies
condition (b) of Proposition 4.6 with $H = G^*$. Hence $f \in \mathfrak{A}(G)$. 


    Note that Bohr-von Neumann theorem involves no topologies on the group $G$. For compact groups $G$, this theorem and
Corollary 5.5 give:


\textbf{Corollary 6.7.} Every continuous function on a compact abelian group is almost periodic.


    Theorem 6.9 below characterizes the continuous almost periodic functions of a topological abelian group $G$ as those
continuous functions that factorize through the Bohr compactification $b_G : G \to b_G$. We need first the following


    \textbf{Remark 6.8.} Let $G$ be a topological abelian group. Consider the Bohr compactification $b_G : G \to b_G$ and for every $f \in B(bG)$
let $\rho(f ) = f \circ b_G$ . This defines a $C$-algebra homomorphism $B(bG) \to B(G)$ such that $\rho(C(bG)) \subseteq C(G), \rho(A(bG)) \subseteq A(G)$
and $\rho((bG)^*) \subseteq G^*$. According to Theorem 3.9, the restriction $\rho\upharpoonright_{\hat{bG}} : \hat{bG} \to G$ is an isomorphism. It can be extended to an
isomorphism of the $\mathbb{C}$-algebras $\mathfrak{X}(bG)$ and $\mathfrak{X}(G)$ that preserves the norm.


    Now we show that the isomorphism $\mathfrak{X}(bG) \to \mathfrak{X}(G)$ can be extended to the $\mathbb{C}$-subalgebras of continuous almost periodic
functions:


    \textbf{Theorem 6.9.} Let $G$ be a topological abelian group. Then a continuous function $f : G \to \mathbb{C}$ is almost periodic iff there exists a
    continuous function $\tilde{f} : bG \to \mathbb{C}$ such that $f = \tilde{f} \circ b_G$.


\textbf{Proof.} Assume there exists a continuous function $\tilde{f} : bG \to \mathbb{C}$ such that $f = \tilde{f} \circ b_G$ . Then $\tilde{f}$ is almost periodic by
Corollary 6.7, so $f = \rho(\tilde{f}) \in A(G)$ by Remark 6.8.


    Now assume that $f \in A(G)$. Then by Theorem 6.6, $f$ can be uniformly approximated by functions from $\mathfrak{X}(G_d)$. According
to Corollary 4.3, we can assume that these functions are continuous, i.e., $f$ can be uniformly approximated by functions
from $\mathfrak{X}(G)$. Let $(g_\alpha)$ be a net in $\mathfrak{X}(G)$ uniformly converging to $f$ . By Remark 6.8 there exists a net $(g_{\tilde{\alpha}})$ in $\mathfrak{X}(bG)$, such that
$g_\alpha = \tilde{g}_\alpha \circ b_G = \rho(\tilde{g}\alpha)$ for every $\alpha$. Then $(\tilde{g}_\alpha)$ is a Cauchy net in $B(bG)$ since the norm is preserved by the isomorphism $\rho$.
By the completeness of $B(bG)$ there exists a limit $\tilde{f} \to B(bG)$ of $(\tilde{g}_\alpha)$ and $\tilde{f}$ is continuous by the continuity of each $\tilde{g}_\alpha$.
Finally, $f = \rho(\tilde{f} )$ follows from $g_\alpha = \rho(\tilde{g}_{\alpha})$ and the continuity of $\rho$.


    From this theorem it easily follows that a topological abelian group $G$ is maximally almost periodic precisely when the
continuous almost periodic functions of $G$ separate the points of $G$.


\subsection{6.3. Invariant means for almost periodic functions}


    The aim of this section is to build an invariant mean for almost periodic functions in the following sense:


\textbf{Definition 6.10.} Let $G$ be an abelian group. An invariant mean on $A(G)$ is a linear functional $\int : A(G) \to \mathbb{C}$ which is


(a) positive (i.e., if $f \in A(G)$ is real-valued and $f \geqslant 0$, then also $\int f \geqslant 0$);


(b) invariant (i.e., $\int f_a = \int f \text{ for every } f \in A(G)$ and $a \in G$, where $f_a(x) = f (x + a)$);


(c)$\int 1 = 1$.


    To define the invariant mean we shall exploit the fact that, according to Bohr-von Neumann theorem, $A(G)$ consists of all
uniform limits of linear combinations of characters of $G$, i.e., $A(G) = \bar{\mathfrak{X}(G)}^{B(G)}$. Of primary importance will be the invariant
subspace $A_0(G) = \bar{\mathfrak{X}_0(G)}^{B(G)}$ of $A(G)$. By Corollary 4.11, it is a closed hyperspace of $A(G)$, hence $A_0 (G)$ has co-dimension
one in $A(G)$ and


    $A(G) = A_0(G) \oplus C \cdot 1$. (3)


    This gives rise to a projection $I : A(G) \to \mathbb{C}$, $f \mapsto I(f )$, that turns out to be the invariant mean (see Theorem 6.13, this
explains the choice of the notation $I$).


    The next lemma is crucial for two reasons. First, it shows that every invariant mean on $A(G)$ must necessarily vanish
on $A_0(G)$, so this allows for the construction of the invariant mean as a linear functional $A(G) \to \mathbb{C}$ that vanishes on $A_0(G)$.
On the other hand, it plays a key role also in the proof of the uniqueness of the invariant mean on $A(G)$.


\textbf{Lemma 6.11.} If $G$ is an abelian group and $\int$ is an invariant mean on $A(G)$, then:


(a) $\int f = 0 \text{ for every } f \in \mathfrak{X}_0(G)$;

(b) $\int : A(G) \to \mathbb{C}$ is continuous;

(c) $\int f = 0 $for every $f \in A_0(G)$.


\textbf{Proof.} (a) More generally, the invariance of $\int$ gives $\int \varphi(x) \bar{\chi(x)} = 0$ when $\varphi,\chi \in G^*$ and $\varphi \neq \chi$. So $\int \varphi(x) = 0$ whenever
$\varphi \neq 1$.


    (b) It follows easily from the positivity axiom that $|\int f |  \sqrt{2} \cdot \| f \|$ for every $f \in A(G)$. Hence, (uniform) continuity of $\int$
can be derived from the inequality $| \int f - \int g| \leqslant |f - g| \leqslant \sqrt{2} \cdot \| f - g\|$ for $f , g \in A(G)$.


    Now (c) follows from (a) and (b). 


The constant $\sqrt{2}$ in the inequality $|\int f |  \sqrt{2} \cdot \| f \|$ established in the above proof may look somewhat unusual. Indeed, it
follows from Theorem 6.13 that there exists a unique invariant mean $I$ on $A(G)$ and $I$ satisfies the inequality with constant 1
in place of $\sqrt{2}$.


    The next lemma is needed for the proof of Theorem 6.13.


\textbf{Lemma 6.12.} Let $G$ be an abelian group, $g \in A_0(G)$ and let $M$ be a closed convex subset of $\mathbb{C}$. If $g(x) + c \in M$ for some $c \in \mathbb{C}$ and for
every $x \in G$, then $c \in M$.


\textbf{Proof.} Assume for contradiction that $c \notin M$. Since $M$ is closed there exists $\varepsilon > 0$ such that $c \notin M + D$, where $D$ is the closed
(so compact) ball with center 0 and radius $\varepsilon$. Let $h \in \mathfrak{X}_0(G)$ with $\|g - h\| \leqslant \varepsilon/2$. Since $M + D$ is still a closed convex set
of $\mathbb{C}$ and $h(x) + c \in M + D$, we conclude with Corollary 4.7 that $c \in M + D$, a contradiction. 


\textbf{Theorem 6.13.} For every abelian group $G$ the assignment $f \mapsto I(f )$ defines a unique invariant mean on $A(G)$ with $|I(f )| \leqslant \| f \|$ for
all $f \in A(G)$.


\emph{Proof.} Obviously, $I(1) = 1$. To establish positivity of the linear functional $f \mapsto I(f )$, apply Lemma 6.12 to the closed convex
subset of $\mathbb{C}$ consisting of all non-negative real numbers.


    To check invariance fix a function $f \in C(G)$. Let $f = f_0 + I(f )\cdot 1$, with $f_0 \in A_0(G)$. Since the action of $G$ on $\mathbb{C} \cdot 1$ is trivial
and $A_0(G)$ is invariant, we get $f_a = (f_0)_a + I(f ) \cdot 1$ with $(f_0)_a \in A_0(G)$. Hence $I(f_a) = I(f )$.


    The last assertion is true if $f = 0$, since then $I(f ) = 0$ and there is nothing to prove. Assume $f \neq 0$ and let $\varepsilon = \| f \|$.
Then $\| f \|=\| f_0 + I(f )\| \leqslant \varepsilon$ yields $|I(f )| \leqslant \varepsilon$ by Lemma 6.12.


    To prove the uniqueness part, we have to see that for every invariant mean $\int$ on $A(G)$ one has $\int f = I(f )$ for every
$f \in A(G)$. It follows from Lemma 6.11 and the definition of $I$, that $\int f = I(f ) = 0$ for every $f \in A_0(G)$. On the other hand,
by $\int 1 = 1 = I(1)$ and linearity, it follows that $\int$ and $I$ coincide on $\mathbb{C} \cdot 1$ too. Therefore, (3) yields $\int = I$.


    For a different proof of this theorem for arbitrary groups see [15, Theorems 18.8 and 18.9] or [19].


\section{Haar integral of the locally compact abelian groups}


    Let $G$ be a compact abelian group. A Haar integral on $G$ is a linear functional $\int : C(G) \to \mathbb{C}$ with the properties $(a)-(c)$
applied to functions $f \in C(G)$. This makes sense since every continuous function on a compact abelian group is almost
periodic. In other words, a Haar integral on $G$ is an invariant mean defined only on the subalgebra $C(G)$ of continuous
functions of $G$ according to Theorem 6.13. This proves the existence part of the following


\textbf{Theorem 7.1} For every compact abelian group $G$ the assignment $f \mapsto I(f ) (f \in C(G))$ defines a (unique) Haar integral on $G$.


\textbf{Proof.} The uniqueness of the Haar integral cannot be derived directly from Theorem 6.13 since the Haar integral is not
defined for all almost periodic functions. Nevertheless, if: $C(G) \to \mathbb{C}$ is a Haar integral on $G$, then one can still see
(arguing as in the proof of items (a) and (b) of Theorem 6.13) that  vanishes on $\mathfrak{X}_0(G)$ and $\int : C(G) \to \mathbb{C}$ is continuous.
From this one can conclude that  vanishes on the closure $C^0(G) := A_0(G) \cap C(G)$ of $\mathfrak{X}_0(G)$ in $B(G)$ as well. By (3), the
subspace $C^0(G)$ splits, i.e., $C(G) = C^0(G) \oplus C \cdot 1$. Since both $\int$ and $I$ vanish on $C^0(G)$, while they obviously coincide on $\mathbb{C} \cdot 1$,
we deduce that $\int$ coincides with the restriction of $I$ on $C(G)$. This ends up the proof of the uniqueness.


    Analogously, we can define a Haar integral on a locally compact (abelian) group $G$ as follows. Denote by $C_0(G)$ the
space of all continuous complex-valued functions on $G$ with compact support. A Haar integral on $G$ is a linear functional $I = \int_G : C_0(G) \to \mathbb{C}$
such that:


\begin{itemize}

    \item (i) $\int f \geqslant 0$ for any real-valued $f \in C_0(G)$ with $f \geqslant 0$;

    \item (ii) $\int f_a = \int f$ for any $f \in C_0(G)$ and any $a \in G$;

    \item (iii) there exists $f \in C_0(G)$ with $\int f \neq 0$.

\end{itemize}


    In the remaining part of this section we will show that every LCA group $G$ admits a Haar integral $\int_G$.


    We begin with a simple property of Haar integrals that will be useful later on.


\textbf{Lemma 7.2}. Let $I = \int_G$ be a Haar integral on a LCA group $G$. Then for any real-valued $h \in C_0(G)$ with $h \geqslant 0$ on G and $h(x) > 0$ for at
least one $x \in G$ we have $I(h) > 0$.


\textbf{Proof.} Let $h \in C_0(G)$ be a real-valued function such that $h \geqslant 0$ on $G$ and $h(x_0) > 0$ for some $x_0 \in G$. Then there exists a
neighborhood $V$ of $0$ in $G$ such that $h(x) \geqslant a = h(x_0)/2$ for all $x \in x_0 + V$ .


    By property (iii) of Haar integrals, there exists $f \in C_0(G)$ with $I(f) \neq 0$. Then $f = u + \mathbf{i}v$ for some real-valued
$u, v \in C_0(G)$, so we must have either $I(u) \neq 0$ or $I(v) \neq 0$. So, without loss of generality we may assume that $f$ is
realvalued. Setting $f_+(x) = \max\{f (x), 0\}$, $x \in G$ and $f_-(x) = \max\{- f (x), 0\}$, we get functions $f_+, f_- \in C_0(G)$ such that $f_+ \geqslant 0$
and $f_-\geqslant 0$ on $G$ and $f = f_+ - f_-$. Thus, either $I(f_+) \neq 0$ or $I(f_-) \neq 0$.


    So, we may assume that $f \geqslant 0$ and $I(f ) \neq 0$; then by (i) we must have $I(f ) > 0$. Since $f \in C_0(G)$, there exists a compact
$K \subset G$ with $f (x) = 0$ on $G / K$. So one can find a finite $F \subset G$ such that $K \subset F + V$. If $A = \max_{x \in G} f (x)$, then $A > 0$
and for every $g \in F$ we have $h_{x_0-g} (x) \geqslant a$ for all $x \in g + V$. Thus, $f(x) \leqslant \frac{A}{a} \sum_{g \in F} h_{x_0-g} (x)$ for all $x \in G$, and therefore
$0 < I(f ) \leqslant \frac{A}{a} |F |I(h)$, which shows that $I(h) > 0$. 


    The following three lemmas are the main steps in the proof of existence of Haar integrals.

\textbf{Lemma 7.3.} If $G$ is a discrete abelian group, then $G$ admits a Haar integral.


\textbf{Proof.} Setting


    $\int_G f = \sum_{x \in G} f(x)$. $f \in C_0 (G)$ 


one checks easily that $\int_G$ is a Haar integral on $G$.


\textbf{Lemma 7.4.} If $G \in \mathcal{L}$ and $H$ is a closed subgroup of $G$ such that both $H$ and $G/H$ admit a Haar integral, then also $G$ admits a Haar
integral.


\textbf{Proof.} Let $f \in C_0(G)$. Then $f_y \upharpoonright_H \in C_0(H)$ for every $y \in G$. Let $F(y) = \int_H f_y \upharpoonright_H$. Then $F : G \to \mathbb{C}$ is a continuous function.
Indeed, let $y_0 \in G$ and $\varepsilon > 0$. There exists a compact $K \subset G$ such that $f = 0$ on $G / K$. Let $U$ be an arbitrary compact
symmetric neighborhood of 0 in $G$. There exists $h \in C_0(G)$ such that $0 \leqslant h(x) \leqslant 1$ for all $x \in G$ and $h(x) = 1$ for all
$x \in y_0 + U + K$.


    Since $f$ is continuous and $U + K$ is compact, there exists a symmetric neighborhood $V$ of 0 in $G$ such that $V \subset U$ and


$|f(x) - f(y)| \leqslant \varepsilon$, $x,y \in U + K$, $x - y \in V$.  (1)


We will now show that $|F (y) - F (y_0)| \leqslant \varepsilon$ for all $y \in y_0 + V$. Given $y \in y_0 + V$ let us first check that


$|f(x - y) - f(x - y_0)| \leqslant \varepsilon h(x)$, $x \in G$  (2)


Indeed, if $x \in G$ is such that $f (x - y) = f (x - y_0) = 0$, then (2) is obviously true. Assume that either $f (x - y) \neq 0$ or
$f (x - y_0) \neq 0$. Then either $x - y \in K$ or $x - y_0 \in K$, so either $x \in y + K \subset y_0 + V + K$ or $x \in y_0 + K$. In both cases
$x \in y_0 + U + K$ and $x - y, x - y_0 \in U + K$. Moreover,


$(x - y) - (x - y_0) = y_0 - y \in y_0 - (y_0 + V) = V$,


so (1) and $h(x) = 1$ imply


$ f(x - y) - f(x - y_0) \leqslant \varepsilon \leqslant \varepsilon h(x)$.


This proves (2).


    From (2) it follows that $| f_y \upharpoonright_H - f_{y_0} \upharpoonright_H |  \varepsilon h \upharpoonright_H$ , so


    $F(y) - F(y_0)| \leqslant \int_H | f_y \upharpoonright_H - f_{y_0} \upharpoonright{H} | \leqslant \varepsilon \int_H h \upharpoonright_H $


This proves the continuity of $F$ at $y_0$.


    Next, for any $x, y \in G$ with $x - y \in H$ we have


$F(x) = \int_H f_x \upharpoonright_H = \int_H (f_y)_{x-y} \upharpoonright_H = \int_H f_y \upharpoonright_H = F(y)$


using the invariance of $\int_H$ in $H$. Then there exists a continuous function $\tilde{F} : G/H \to \mathbb{C}$ such that $F = \tilde{F} \circ p$, where
$p : G \to G/H$ is the natural projection. Moreover, $\tilde{F} \in C_0(G/H)$.


    Set $\int_G f := \int_{G/H} \tilde{F}$ for any $f \in C_0(G)$. It is now easy to check thatG is a Haar integral on $G$. Indeed, the linearity of $\int_G$
follows from that of $\int_{G/H}$ and the fact that $(\alpha f_1 + \beta f_2) \cong \alpha \tilde{F}_1 + \beta \tilde{F}_2$ for any $\alpha, \beta \in \mathbb{C}$ and any $f_1, f_2 \in C_0(G)$. If $f \geqslant 0$,
then $\tilde{F} \geqslant 0$, too, so $\int_G = \int_{G/H} \tilde{F} \geqslant 0$. To check invariance, notice that for any $x \in G$ we have $(f_x) \tilde{} = (\tilde{F} )_{p(x)}$, so


$\int_G f_x = \int_{G/H} (f_x) \tilde{} = \int_{G/H} \tilde{(F)}_{p(x)} = \int_{G/H} \tilde{F} = \int_G f$


    We will now show that $\int_G$ is non-trivial, i.e. it satisfies (iii). Take an arbitrary compact neighborhood $U$ of 0 in $G$.
There exists a real-valued $f \in C_0(G)$ with $f \geqslant 0$ on $G$ such that $f (x) \geqslant 1$ for all $x \in U$. Then $f \geqslant 1$ on $U \cap H$, so by
Lemma 7.2, $F (0) = \int_H f \upharpoonright H > 0$, which gives $\tilde{F} (0) > 0$. Moreover $f \geqslant 0$ implies $\tilde{F} \geqslant 0$ on $G/H$, so using Lemma 7.2 again,
$\int_G f = \int_{G/H} \tilde{F} > 0$.


    Thus, $\int_G$ is a Haar integral on $G$. 


    We are now ready to prove existence of Haar integrals on general LCA groups.


\textbf{Theorem 7.5.} Every locally compact abelian group admits a Haar integral.


\textbf{Proof.} Let $G$ be a LCA group. If $G$ is compact or discrete, then Theorem 7.1 or Lemma 7.3 apply. In case $G$ is compactly
generated, $G$ has a discrete subgroup $H$ such that $G/H$ is compact by Theorem 5.12. So both $H$ and $G/H$ admit a Haar
integral. It follows from Lemma 7.4 that $G$ admits a Haar integral, too.


    In the general case $G$ has an open subgroup $H$ which is compactly generated - just take the subgroup generated by an
arbitrary compact neighborhood of 0 in $G$. Such a subgroup $H$ is locally compact (and compactly generated), hence admits
a Haar integral by the above argument, while $G/H$ is discrete, so it also admits a Haar integral by Lemma 7.3. Finally,
Lemma 7.4 implies that $G$ admits a Haar integral, too.


\textbf{Acknowledgement}


Thanks are due to the referee for her/his careful reading and useful advise.


\textbf{References}
[1] G. Artico, V. Malykhin, U. Marconi, Some large and small sets in topological groups, Math. Pannon. 12 (2) (2001) 157-165.
[2] N. Bogoliouboff, Sur quelques propriétés arithmétiques des presque-périodes, Ann. Chaire Phys. Math., Kiev 4 (1939) 195-205.
[3] H. Bohr, Collected mathematical works, in: E. Følner, B Jessen (Eds.), Dansk Matemetitisk Forsening, 1952.
[4] H. Cartan, Sur le mesure de Haar, C. R. Acad. Sci. Paris 211 (1940) 759-762.
[5] W.W. Comfort, Ross, Topologies induced by groups of characters, Fund. Math. 55 (1964) 283-291.
[6] M. Cotlar, R. Ricabarra, On the existence of characters in topological groups, Amer. J. Math. 76 (1954) 375-388.
[7] D. Dikranjan, A. Orsatti, On an unpublished manuscript of Ivan Prodanov concerning locally compact modules and their dualities, Comm. Algebra 17
(1989) 2739-2771.
[8] D. Dikranjan, Iv. Prodanov, L. Stoyanov, Topological Groups: Characters, Dualities and Minimal Group Topologies, Pure Appl. Math., vol. 130, Marcel
Dekker Inc., New York, Basel, 1989.
[9] R. Ellis, H.B. Keynes, Bohr compactifications and a result of Følner, Israel J. Math. 12 (1972) 314-330.
[10] R. Engelking, General Topology, Heldermann Verlag, Berlin, 1989.
[11] E. Følner, A proof of the main theorem for almost periodic functions in an abelian group, Ann. of Math. (2) 50 (1949) 559-569.
[12] E. Følner, Generalization of a theorem of Bogolioúboff to topological abelian groups. With an appendix on Banach mean values in non-abelian groups,
Math. Scand. 2 (1954) 5e-18.
[13] L. Fuchs, Infinite Abelian Groups, vol. I, Academic Press, New York, London, 1973.
[14] E. Glasner, On minimal actions of Polish groups, Topology Appl. 85 (1998) 119-125.
[15] E. Hewitt, K. Ross, Abstract Harmonic Analysis, vol. I, Springer-Verlag, Berlin, Göttingen, Heidelberg, 1979.
[16] K.H. Hofmann, S. Morris, The Structure of Compact Groups. A Primer for the Student—A Handbook for the Expert, de Gruyter Stud. Math., vol. 25,
Walter de Gruyter & Co., Berlin, 1998, xviii+835 pp.
[17] P. Holm, On the Bohr compactification, Math. Ann. 156 (1964) 34-46.
[18] E.R. van Kampen, Locally bicompact abelian groups and their character groups, Ann. of Math. (2) 36 (1935) 448-463.
[19] W. Maak, Fastperiodische Funktionen, Springer-Verlag, Berlin, 1950.
[20] G.W. Mackey, Harmonic analysis as the exploitation of symmetry—A historical survey, Bull. Amer. Math. Soc. (N.S.) 3 (1) (1980) 543-698, part 1.
[21] S. Morris, Pontryagin Duality and the Structure of Locally Compact Abelian Groups, Cambridge Univ. Press, Cambridge, 1977.
[22] L. Nachbin, The Haar Integral, D. Van Nostrand Company, Princeton, 1965.
[23] G. Nobeling, H. Beyer, Allgemaine Approximationskriterien mit Anwendungen, Jahrsber. Deutsch. Math. Ver. 58 (1955) 54-72.
[24] V. Pestov, Forty-plus annotated questions about large topological groups, in: Elliott Pearl (Ed.), Open Problems in Topology II, Elsevier B.V., March,
2007, pp. 439–450.
[25] L. Pontryagin, Topological Groups, Gordon and Breach Science Publishers, Inc., New York, London, Paris, 1966, translated from Russian by Arlen Brown.
[26] Iv. Prodanov, Compact representations of continuous algebraic structures, in: General Topology and its Relations to Modern Analysis and Algebra, II,
Proc. Second Prague Topological Sympos., 1966, Academia, Prague, 1967, pp. 290-294 (in Russian).
[27] Iv. Prodanov, Elementary proof of the Peter-Weyl theorem, C. R. Acad. Bulgare Sci. 34 (3) (1981) 315-318 (in Russian).

\end{document}
